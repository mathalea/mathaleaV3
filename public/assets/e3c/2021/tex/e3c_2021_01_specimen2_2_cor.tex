\textbf{\large\textsc{Exercice 2 \hfill 5 points}}

\bigskip

Une collectivité locale octroie une subvention de \np{116610}~\euro{} pour le forage d'une nappe d'eau souterraine. Une entreprise estime que le forage du premier mètre coûte 130~\euro; le
forage du deuxième mètre coûte 52~\euro{} de plus que celui du premier mètre ; le forage du
troisième mètre coûte 52~\euro{} de plus que celui du deuxième mètre, etc.

Plus généralement, le forage de chaque mètre supplémentaire coûte 52~\euro{} de plus que celui
du mètre précédent.

Pour tout entier $n$ supérieur ou égal à 1, on note: $u_n$ le coût du forage du $n$-ième mètre en
euros et $S_n$ le coût du forage de $n$ mètres en euros; ainsi $u_1=130$.

\medskip

\begin{enumerate}
\item  $u_2=130+52=182$ et $u_3=182+52= 234$.
\item La suite $\left(u_n\right)$ est arithmétique de premier terme $u_1=130$ et de raison $r=52$.

%En déduire l'expression de $u_n$ en fonction de $n$, pour tout $n$ entier naturel non nul.

On en déduit que pour tout $n\geqslant 1$, on a: $u_n=u_1+(n-1)\times r$ donc $u_n=130+52(n-1)$.

\item $S_2= u_1+u_2 = 130+182=312$ et $S_3=u_1+u_2+u_3 = S_2+u_3=312+234=546$.

\item Afin de déterminer le nombre maximal de mètres que l'entreprise peut forer avec la subvention qui est octroyée, on considère la fonction Python suivante:

\begin{center}
\fbox{\tt
\begin{tabular}{l}
def nombre\_metre(S) :\\
\quad C = 130\\
\quad n = 1\\
\quad while C < S :\\
\qquad C = C + ...\\
\qquad n = n + 1\\
 \quad return n
\end{tabular}
}
\end{center}

On veut compléter cet algorithme de sorte que l'exécution de la fonction \verb!nombre_metre(S)! renvoie le nombre maximal de mètres que l'entreprise peut forer avec la subvention octroyée. %Justifier votre réponse.

La variable \texttt{C} contient le coût du forage de \texttt{n} mètres.

Pour tout $n\geqslant 1$, on a $S_{n+1}= u_1+u_2+\cdots + u_n + u_{n+1} = S_n+u_{n+1}$; or $u_n=130+52\left (n-1\right )$ donc $u_{n+1} =130+52n$. On a donc $S_{n+1} = S_n + 130 + 52n$.

Il faut donc ajouter $130+52n$ à la variable  \texttt{C} à chaque tour de boucle.

\begin{center}
\fbox{\tt
\begin{tabular}{l}
def nombre\_metre(S) :\\
\quad C = 130\\
\quad n = 1\\
\quad while C < S :\\
\qquad C = C + \blue 130 + 52*n\\
\qquad n = n + 1\\
 \quad return n
\end{tabular}
}
\end{center}

\item On admet que, pour tout entier naturel non nul, $S_n=26n^2 +104n$. %En déduire la valeur de $n$ que fournit la fonction Python donnée à la question 4. On expliquera la démarche utilisée.

On cherche le plus grand entier $n$ tel que $S_n \leqslant \np{116610}$.

On résout l'équation $26n^2 + 104n - \np{116610}=0$.

$\Delta = 104^2 - 4 \times 26 \times (-\np{116610}) = \np{12138256} = \np{3484}^2$ 

Deux solutions: $n'=\dfrac{-104+\np{3484}}{2\times 26}=65$ et $n''=\dfrac{-104-\np{3484}}{2\times 26}=-69<0$.

Avec $\np{116610}$~\euro{}, on pourra creuser 65 mètres.

\end{enumerate}

\vspace{0.5cm}

