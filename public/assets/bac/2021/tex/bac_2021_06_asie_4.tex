\textbf{EXERCICE -- A}

\medskip

\begin{tabular}{|l|}\hline
\textbf{Principaux domaines abordés}\\
-- convexité\\
-- fonction logarithme\\ \hline
\end{tabular}

\bigskip

\textbf{Partie I : lectures graphiques}

\medskip

$f$ désigne une fonction définie et dérivable sur $\R$.

On donne ci-dessous la courbe représentative de la fonction dérivée $f'$.

\begin{center}
\psset{xunit=0.9cm,yunit=3cm}
\begin{pspicture}(-7,-0.8)(7,1.2)
\multido{\n=-7+1}{15}{\psline[linestyle=dashed,linewidth=0.1pt](\n,-0.8)(\n,1.2)}
\multido{\n=-0.8+0.2}{11}{\psline[linestyle=dashed,linewidth=0.1pt](-7,\n)(7,\n)}
\psaxes[linewidth=1.25pt,labelFontSize=\scriptstyle]{->}(0,0)(-7,-0.8)(7,1.2)
\psplot[plotpoints=2000,linewidth=1.25pt,linecolor=red]{-7}{7}{x 2 mul 1 add x dup mul x add 2.5 add div}
\uput[u](0,1){\red Courbe de la fonction dérivée $f'$}
\end{pspicture}
\end{center}

\medskip

\emph{Avec la précision permise par le graphique, répondre aux questions suivantes}

\medskip

\begin{enumerate}
\item Déterminer le coefficient directeur de la tangente à la courbe de la fonction $f$ en $0$.
\item 
	\begin{enumerate}
		\item Donner les variations de la fonction dérivée $f'$.
		\item En déduire un intervalle sur lequel $f$ est convexe.
	\end{enumerate}
\end{enumerate}

\bigskip

\textbf{Partie II : étude de fonction}

\medskip

La fonction $f$ est définie sur $\R$ par 

\[f(x) = \ln \left(x^2 + x + \dfrac{5}{2}\right).\]

\medskip

\begin{enumerate}
\item Calculer les limites de la fonction $f$ en $+\infty$ et en $-\infty$.
\item Déterminer une expression $f'(x)$ de la fonction dérivée de $f$ pour tout $x \in \R$.
\item En déduire le tableau des variations de $f$. On veillera à placer les limites dans ce tableau.
\item 
	\begin{enumerate}
		\item Justifier que l'équation $f(x) = 2$ a une unique solution $\alpha$ dans l'intervalle $\left[-\dfrac{1}{2}~;~+ \infty\right[$. 
		\item Donner une valeur approchée de $\alpha$ à $10^{-1}$ près.
 	\end{enumerate}
\item La fonction $f'$ est dérivable sur $\R$. On admet que, pour tout $x \in  \R$,\,  $f''(x) = \dfrac{-2x^2 - 2x + 4}{\left(x^2 + x + \dfrac{5}{2}\right)^2}$.

Déterminer le nombre de points d'inflexion de la courbe représentative de $f$.
\end{enumerate}

\bigskip

