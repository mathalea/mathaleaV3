
\textbf{Commun à tous les candidats}

\medskip

%Au 1\up{er} janvier 2020, la centrale solaire de Big Sun possédait \np{10560} panneaux solaires.
%
%On observe, chaque année, que 2\,\% des panneaux se sont détériorés et nécessitent d'être retirés tandis que 250 nouveaux panneaux solaires sont installés.
%
%\bigskip

\textbf{Partie A - Modélisation à l'aide d'une suite}

\medskip

%On modélise l'évolution du nombre de panneaux solaires par la suite $\left(u_n\right)$ définie par $u_0 = \np{10560}$ et, pour tout entier naturel $n$,\, $u_{n+1} = 0,98u_n +250$, où $u_n$  est le nombre de panneaux solaires au 1\up{er} janvier de l'année $2020 +n$.

\medskip

\begin{enumerate}
\item 
	\begin{enumerate}
		\item  %Expliquer en quoi cette modélisation correspond à la situation étudiée.
Retrancher 2\,\% c'est multiplier par $1 - \dfrac{2}{100} = 1 - 0,02 = 0,98$.

D'une année sur l'autre on multiplie le nombre de panneaux par 0,98 puis on augment de nombre de panneaux de 250.
		\item  %On souhaite savoir au bout de combien d'années le nombre de panneaux solaires sera strictement supérieur à \np{12000}.
		
%À l'aide de la calculatrice, donner la réponse à ce problème.
Avec la calculatrice il suffit de taper \np{10560} Entrée puis $\times 0,98 + 50$.

Entrée donne $u_1 \approx \np{10599}$, les appuis successifs de Entrée donnent $u_2$, \, $u_3$, etc.

On obtient $u_{68} \approx \np{12009}$.

le nombre de panneaux dépassera \np{12000} au bout de 68 ans soit en 2088.
		\item  Recopier et compléter le programme en Python ci-dessous de sorte que la valeur
cherchée à la question précédente soit stockée dans la variable n à l'issue de l'exécution de ce dernier.
\begin{center}
\begin{tabular}{|l|}\hline
u $= \np{10560}$\\
n$ =0$\\
while  u $\leqslant \np{12000}$ :\\
\qquad u $= 0,98*\text{u} + 250$  \\
\qquad n $= \text{n} + 1$\\ \hline
\end{tabular}
\end{center}

	\end{enumerate}
\item %Démontrer par récurrence que, pour tout entier naturel $n$, on a $u_n \leqslant \np{12500}$.
\emph{Initialisation }: $u_0 = \np{10560} \leqslant \np{12500}$ : la proposition est vraie au rang 0.

\emph{Hérédité} : soit $n \in \N$ tel que $u_n \leqslant \np{12500}$ soit en multipliant par 0,98 :

$0,98u_n \leqslant 0,98 \times \np{12500}$ et en ajoutant 250 à chaque membre :

$0,98u_n + 250  \leqslant 0,98 \times \np{12500} + 250$ ou $u_{n+1}  \leqslant \np{12250} + 250 $ et finalement $u_{n+1} \leqslant \np{12500}$ : la proposition est vraie au r	ng $n + 1$.

La proposition est vraie au rang 0 et si elle est vraie au rang $n \in \N$ elle est vraie au rang $n + 1$ : d'après le principe de la récurrence la proposition $u_n \leqslant \np{12500}$ est vraie pur tout naturel $n \in \N$.
\item %Démontrer que la suite $\left(u_n\right)$ est croissante.
On a pour $n \in \N$, \, $u_{n+1} - u_n = 0,98u_n + 250 - u_n = 250  - 0,02u_n$.

Or d'après le résultat précédent :

$u_n \leqslant \np{12500} \Rightarrow 0,02u_n \leqslant 0,02 \times \np{12500}$ ou encore $0,02u_n \leqslant 250$ ou en ajoutant à chaque membre $- 0,02u_n$ :

$0 \leqslant 250 - 0,02u_n$ ; on a donc démontré que $u_{n+1} - u_n  \geqslant 0$, ce qui signifie que la suite $\left(u_n\right)$ est croissante.
\item %En déduire que la suite $\left(u_n\right)$ converge. Il n'est pas demandé, ici, de calculer sa limite.
La suite $\left(u_n\right)$ est croissante et majorée par \np{12500} : elle est donc convergente vers une limite $\ell$, telle que $\ell \leqslant \np{2500}$.
\item %On définit la suite $\left(v_n\right)$  par $v_n = u_n - \np{12500}$, pour tout entier naturel $n$.
	\begin{enumerate}
		\item %Démontrer que la suite $\left(v_n\right)$ est une suite géométrique de raison $0,98$ dont on précisera le premier terme.
		Quel que soit $n \in \N$, \, $v_{n+1} = u_{n+1} - \np{12500} = 0,98u_n + 250 - \np{12500}$, soit 
		
		$v_{n+1}  = 0,98u_n - \np{12250} = 0,98u_n - \np{12250}\times \frac{0,98}{0,98} = 0,98u_n - \np{12500} \times 0,98 = 0,98\left(u_n - \np{12500}\right)$ soit enfin $v_{n+1} = 0,98v_n$ : cette relation vraie quel que soit $n \in \N$ montre que la suite $\left(v_n\right)$ est une suite géométrique de raison $0,98$ de premier terme $v_0 = u_0 - \np{12500} = \np{10560} - \np{12500} = - \np{1940}$.
		\item %Exprimer, pour tout entier naturel $n$,\, $v_n$ en fonction de $n$.
		On sait qu'alors quel que soit $n \in \N$, \, $v_n = v_0 \times 0,98^n$, soit $v_n = - \np{1940} \times 0,98^n$.
		\item %En déduire, pour tout entier naturel $n$,\, $u_n$ en fonction de $n$.
Pour tout entier naturel $n$,\,$v_n = u_n - \np{12500} \iff u_n = v_n + \np{12500} = \np{12500} - \np{1940} \times 0,98^n$.
		\item %Déterminer la limite de la suite $\left(u_n\right)$.
		
Comme $0 < 0,98 < 1$, on sait que $\displaystyle\lim_{n \to + \infty} 0,98^n = 0$, donc 
		
$\displaystyle\lim_{n \to + \infty} u_n = \np{12500}$
		
Interpréter ce résultat dans le contexte du modèle.
	\end{enumerate}
\end{enumerate}

\bigskip

\textbf{Partie B - Modélisation à l'aide d'une fonction}

%\medskip
%
%Une modélisation plus précise a permis d'estimer le nombre de panneaux solaires de la centrale à l'aide de la fonction $f$ définie pour tout $x \in [0~;~ +\infty[$ par 
%
%$f(x) = \np{12500} - 500 \e^{-0,02x + 1,4}$, où $x$ représente le nombre d'années écoulées depuis le 1\up{er} janvier 2020.

\medskip

\begin{enumerate}
\item % Étudier le sens de variation de la fonction $f$.
La fonction $f$ est une somme de fonctions dérivables sur $[0~;~ +\infty[$ et sur cet intervalle :

$f'(x) = - 0,02 \times (- 500)\e^{-0,02x+1,4} = 10\e^{-0,02x+1,4}$.

Comme la fonction exponentielle est strictement positive, on en déduit que $f'(x) > 0$ : la fonction est donc strictement croissante sur $[0~;~ +\infty[$ 
\item %Déterminer la limite de la fonction $f$ en $+\infty$.
On sait que $\displaystyle\lim_{n \to + \infty} \e^{-0,02x+1,4} = 0$, donc $\displaystyle\lim_{n \to + \infty} 500 \e^{-0,02x+1,4} = 0$ et $\displaystyle\lim_{n \to + \infty} f(x) = \np{12500}$.
\item %En utilisant ce modèle, déterminer au bout de combien d'années le nombre de panneaux solaires dépassera \np{12000}.
Il faut résoudre l'inéquation :

$\np{12500} - 500 \e^{-0,02x+1,4} > \np{12000} \iff 500 > 500\e^{-0,02x+1,4} \iff 1 >\e^{-0,02x+1,4} \iff \e^0 \geqslant \e^{-0,02x+1,4}$, soit par croissance de la fonction exponentielle :

$0 > - 0,02x + 1,4 \iff 0,02x > 1,4$ et en multipliant chaque membre par 50 : 

$x > 70$ :il faut donc attendre 71 ans pour que le nombre de panneaux dépasse \np{12000}, soit en 2091.

\end{enumerate}

\bigskip

