
\medskip

\begin{tabular}{|l|}\hline
Principaux domaines abordés :\\
Géométrie de l'espace rapporté à un repère orthonormé.\\ \hline
\end{tabular}

\bigskip

On considère le cube ABCDEFGH donné en annexe. 

On donne trois points I, J et K vérifiant :

\[\vect{\text{EI}} = \dfrac{1}{4} \vect{\text{EH}},\qquad   \vect{\text{EJ}} = \dfrac{1}{4}  \vect{\text{EF}}, \qquad  \vect{\text{BK}} = \dfrac{1}{4}  \vect{\text{BF}}\]

Les points I, J et K sont représentés sur la \textbf{figure donnée en annexe, à compléter et à rendre avec la copie}.

On se place dans le repère orthonormé $\left(\text{A}~;~\vect{\text{AB}},~\vect{\text{AD}},~\vect{\text{AE}}\right)$.

\medskip

\begin{enumerate}
\item Donner sans justification les coordonnées des points I, J et K.
\item Démontrer que le vecteur $\vect{\text{AG}}$ est normal au plan (IJK).
\item Montrer qu'une équation cartésienne du plan (IJK) est $4x + 4y + 4z - 5 = 0$.
\item Déterminer une représentation paramétrique de la droite (BC).
\item En déduire les coordonnées du point L, point d'intersection de la droite (BC) avec le plan (IJK).
\item Sur la figure en annexe, placer le point L et construire l'intersection du plan (IJK) avec la face (BCGF).
\item Soit M$\left(\frac{1}{4}~;~1~;~0\right)$. Montrer que les points I, J, L et M sont coplanaires.
\end{enumerate}

\begin{center}
	\textbf{\Large ANNEXE À COMPLÉTER ET À RENDRE AVEC LA COPIE}
	

	\psset{unit=1cm}
	\begin{pspicture}(9.5,9.5)
	\psframe[linewidth=1.25pt](0.5,0.5)(6,6)%BCGF
	\psline[linewidth=1.25pt](6,0.5)(8.7,3.2)(8.7,8.7)(6,6)%CDHG
	\psline[linewidth=1.25pt](8.7,8.7)(3.2,8.7)(0.5,6)%HEF
	\psline[linewidth=1pt,linestyle=dashed](0.5,0.5)(3.2,3.2)(8.7,3.2)%BAD
	\psline[linewidth=1pt,linestyle=dashed](3.2,3.2)(3.2,8.7)%AE
	\uput[d](3.2,3.2){A} \uput[d](0.5,0.5){B} \uput[d](6,0.5){C} \uput[dr](8.7,3.2){D} 
	\uput[u](3.2,8.7){E} \uput[ul](0.5,6){F} \uput[u](6,6){G} \uput[ur](8.7,8.7){H} 
	\uput[u](4.575,8.7){I} \uput[ul](2.525,8.025){J} \uput[l](0.5,1.875){K}
	\psdots(3.2,3.2)(0.5,0.5)(6,0.5)(8.7,3.2)(3.2,8.7)(0.5,6)(6,6)(8.7,8.7)(2.525,8.025)(4.575,8.7)(0.5,1.875)
	\end{pspicture}
	\end{center}
	

\bigskip

