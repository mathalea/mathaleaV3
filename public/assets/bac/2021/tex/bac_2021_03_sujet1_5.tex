
\medskip

\begin{tabularx}{\linewidth}{|X|}\hline
\textbf{Principaux domaines abordés : Fonction logarithme; convexité}
\\ \hline
\end{tabularx}

\medskip

On considère la fonction $f$ définie sur l'intervalle $]0~;~ +\infty[$ par :

\[f(x) = x + 4 - 4 \ln (x) - \dfrac{3}{x}\]

où ln désigne la fonction logarithme népérien.

On note $\mathcal{C}$ la représentation graphique de $f$ dans un repère orthonormé.

\medskip

\begin{enumerate}
\item Déterminer la limite de la fonction $f$ en $+\infty$.
\item On admet que la fonction $f$ est dérivable sur $]0~;~ +\infty[$ et on note $f'$ sa fonction dérivée.

Démontrer que, pour tout nombre réel $x > 0$, on a :

\[f'(x) = \dfrac{x^2 - 4x + 3}{x^2}.\]

\item
	\begin{enumerate}
		\item Donner le tableau de variations de la fonction $f$ sur l'intervalle $]0~;~ +\infty[$. 
		
On y fera figurer les valeurs exactes des extremums et les limites de $f$ en $0$ et en $+ \infty$. 
		
On admettra que $\displaystyle\lim_{x \to 0} f(x) = - \infty$.
		\item Par simple lecture du tableau de variations, préciser le nombre de solutions de l'équation $f(x) = \dfrac{5}{3}$.
	\end{enumerate}
\item Étudier la convexité de la fonction $f$ c'est-à-dire préciser les parties de l'intervalle $]0~;~ +\infty[$ sur lesquelles $f$ est convexe, et celles sur lesquelles $f$ est concave. 

On justifiera que la courbe $\mathcal C$ admet un unique point d'inflexion, dont on précisera les coordonnées.
\end{enumerate}
