
\textbf{Commun à tous les candidats}
\medskip

%Une chaîne de fabrication produit des pièces mécaniques. On estime que 5\,\% des pièces
%produites par cette chaîne sont défectueuses.
%
%Un ingénieur a mis au point un test à appliquer aux pièces. Ce test a deux résultats possibles :
%
%\og positif \fg{} ou bien \og négatif \fg.
%
%On applique ce test à une pièce choisie au hasard dans la production de la chaîne.
%
%On note $p(E)$ la probabilité d'un évènement $E$.
%
%On considère les évènements suivants :
%
%\begin{itemize}
%\item $D$ : \og la pièce est défectueuse \fg{} ;
%\item $T$ : \og la pièce présente un test positif \fg {};
%\item $\overline{D}$ et $\overline{T}$ désignent respectivement les évènements contraires de $D$ et $T$.
%\end{itemize}
%
%Compte tenu des caractéristiques du test, on sait que :
%\begin{itemize}
%\item La probabilité qu'une pièce présente un test positif sachant qu'elle défectueuse est
%égale à $0,98$ ;
%\item la probabilité qu'une pièce présente un test négatif sachant qu'elle n'est pas
%défectueuse est égale à $0,97$.
%\end{itemize}
%\medskip
%\begin{center}
%\textbf{Les parties I et II peuvent être traitées de façon indépendante.}
%\end{center}

\medskip

\textbf{PARTIE I}

\medskip

\begin{enumerate}
\item On traduit la situation à l'aide d'un arbre pondéré:
\begin{center}
\pstree[treemode=R,nodesepB=3pt,levelsep=2.75cm]{\TR{}}
{\pstree{\TR{$D$~~}\taput{0,05}}
	{\TR{$T$}\taput{0,98}
	\TR{$\overline{T}$}\tbput{0,02}
	}
\pstree{\TR{$\overline{D}$~~}\tbput{0,95}}
	{\TR{$T$}\taput{0,03}
	\TR{$\overline{T}$}\tbput{0,97}
	}	
}
\end{center}

\item \begin{enumerate}
\item %Déterminer la probabilité qu'une pièce choisie au hasard dans la production de lachaîne soit défectueuse et présente un test positif.
On a: $P(D \cap T) = P(D) \times P_D(T) = 0,05 \times 0,98 = 0,049$.
\item %Démontrer que: $p(T) = \np{0.0775}$.
On a de même : $P\left(\overline{D} \cap T\right) = P\left(\overline{D}\right)  \times P_{\overline{D}}(T) = 0,95 \times 0,3 = \np{0,0285}$.

D'après la loi des probabilités totales :

$P(T) = P(D \cap T) + P\left(\overline{D} \cap T\right) = 0,049 + \np{0,0285} = \np{0,0775}$.
\end{enumerate}
\item %On appelle \textbf{valeur prédictive positive} du test la probabilité qu'une pièce soit défectueuse sachant que le test est positif. On considère que pour être efficace, un test doit avoir une valeur prédictive positive supérieure à $0,95$.

%Calculer la valeur prédictive positive de ce test et préciser s'il est efficace.
La valeur prédictive positive du test est égale à :

$P_T(D) = \dfrac{P(T \cap D)}{P(T)} = \dfrac{P(D \cap T)}{P(T)} = \dfrac{0,049}{\np{0,0775}} \approx \np{0,6322}$, soit 0,632 au millième près.

Comme $0,632 < 0,95$ on peut en déduire que le test n'est pas efficace.
\end{enumerate}
\medskip

\textbf{PARTIE II}

\medskip

%On choisit un échantillon de $20$ pièces dans la production de la chaîne, en assimilant ce choix à un tirage avec remise. On note $X$ la variable aléatoire qui donne le nombre de pièces défectueuses dans cet échantillon.
%
%On rappelle que: $p(D) = 0,05$.
%
%\medskip

\begin{enumerate}
\item %Justifier que $X$ suit une loi binomiale et déterminer les paramètres de cette loi.
Le choix de l'échantillon étant assimilé à un tirage avec remise et avec une probabilité constante de choisir un produit défectueux égale à $ 0,05$, on peut donc dire que la variable aléatoire $X$ suit une loi binomiale de paramètres $n = 10$ et $p  = 0,05$.
\item %Calculer la probabilité que cet échantillon contienne au moins une pièce défectueuse.

%On donnera un résultat arrondi au centième.
On a: $p(X = 0) = 0,05^0 \times 0,95^{20}$.

Donc la probabilité cherchée est $p(X \geqslant 1) = 1 -p(X =0) = 1 - 0,95^{20} \approx 0,642$, soit 0,64 au centième près.
\item %Calculer l'espérance de la variable aléatoire $X$ et interpréter le résultat obtenu.
On a: $E = n \times p = 20 \times 0,05 = 1$.

Cela signifie que sur un grand nombre de tirages d'échantillons on trouvera 1 pièce défectueuse sur 20 pièces tirées.
\end{enumerate}

\vspace{0,5cm}

