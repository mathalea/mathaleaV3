 
\smallskip

\begin{tabular}{|l|}\hline
Principaux domaines abordés : \\ 
\hspace{1cm}$\bullet~~$Fonction logarithme népérien\\
\hspace{1cm}$\bullet~~$Convexité\\ \hline
\end{tabular}

\medskip

Dans le plan muni d'un repère, on considère ci-dessous la courbe $\mathcal{C}_f$ représentative d'une fonction $f$, deux fois dérivable sur l'intervalle $]0~;~ +\infty[$. 

La courbe $\mathcal{C}_f$ admet une tangente horizontale $T$ au point A(1~;~4).

\begin{center}
\psset{unit=1.25cm}
\begin{pspicture*}(-0.5,-0.8)(6.4,4.4)
\psgrid[gridlabels=0pt,subgriddiv=5,gridcolor=gray](0,-1)(9,5)
\psaxes[linewidth=1.25pt]{->}(0,0)(0,-0.8)(6.4,4.4)
\psplot[plotpoints=2000,linewidth=1.25pt,linecolor=red]{0.1}{6.4}{x ln 1 add 4 mul x div}
\psline[linewidth=1.25pt](0,4)(6.4,4)\uput[u](5.8,4){$T$}
\uput[u](5.8,1.9){\red $\mathcal{C}_f$}
\uput[u](1,4){A}
\end{pspicture*}
\end{center}

\medskip

\begin{enumerate}
\item Préciser les valeurs $f(1)$ et $f'(1)$.
\end{enumerate}

On admet que la fonction $f$ est définie pour tout réel $x$ de l'intervalle $]0~;~ +\infty[$ par:

\[f(x) = \dfrac{a + b \ln x}{x} \,\, 
\text{où }\, a \text{ et}\, b \text{ sont deux nombres réels}.\]

\begin{enumerate}[resume]
\item Démontrer que, pour tout réel $x$ strictement positif, on a :

\[f'(x) = \dfrac{b - a - b\, \ln x}{x^2}.\]

\item En déduire les valeurs des réels $a$ et $b$.
\end{enumerate}

Dans la suite de l'exercice, on admet que la fonction $f$ est définie pour tout réel $x$ de l'intervalle $]0~;~ +\infty[$ par:

\[f(x) = \dfrac{4 + 4\ln x}{x}.\]

\begin{enumerate}[resume]
\item Déterminer les limites de $f$ en $0$ et en $+\infty$.
\item Déterminer le tableau de variations de $f$ sur l'intervalle $]0~;~ +\infty[$. 
\item Démontrer que, pour tout réel $x$ strictement positif, on a :

\[f''(x) = \dfrac{- 4 + 8\ln x}{x^3}.\]

\item Montrer que la courbe $\mathcal{C}_f$ possède un unique point d'inflexion B dont on précisera les coordonnées.
\end{enumerate}
