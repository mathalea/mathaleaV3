\textbf{\large\textsc{Exercice 1 \hfill 5 points}}

\bigskip

%\emph{Cet exercice est un questionnaire à choix multiples (QCM). Les cinq questions sont indépendantes. Pour chacune des questions, une seule des quatre réponses est exacte.\\
%Le candidat indiquera sur sa copie le numéro de la question et la lettre correspondant à la réponse exacte.\\
%Aucune justification n'est demandée. Une réponse exacte rapporte un point, une réponse fausse ou une absence de réponse ne rapporte ni n'enlève aucun point.}
%
%\bigskip

\textbf{Question 1}

\medskip

Pour $x$ pièces produites, le coût de fabrication $C(x)$, en milliers d'euros est donné par

$C(x)= 0,01x^3 - 0,135x^2 + 0,6x +15$, avec $x\in [0~;~30]$.

Pour 2 pièces produites, le coût de fabrication en euros est:

\medskip

{\renewcommand{\arraystretch}{1.5}
\begin{tabularx}{\linewidth}{|X|X|X|X|}
\hline
\textcolor{blue}{\textbf{a.}\quad $15,74$} &  \textbf{b.}\quad $157,4$ &  \textbf{c.}\quad $\np{1574}$ &  \textbf{d.}\quad $\np{15740}$\\
\hline
\end{tabularx}}

\medskip

\begin{tabular}{@{\hspace*{0.05\linewidth}} | p{0.93\linewidth}}
$C(2)=0,02\times 2^3 -0,135\times 2^2 +0,6\times 2 + 15 = 15,74$\\
\textbf{Réponse a.}
\end{tabular}

\bigskip

\textbf{Question 2}

\medskip

Soit $f$ une fonction polynôme du second degré donnée, pour tout nombre réel $x$ par

$f(x)= ax^2+bx+c$, où $a$, $b$, $c$ sont réels. On note $\Delta$ son discriminant. On donne ci-dessous $\mathcal{C}_f$ la courbe représentative de $f$ et on suppose qu'elle admet l'axe des abscisses comme tangente en un de ses points.

\begin{center}
\scalebox{0.7}{
\psset{xunit=1cm, yunit=0.5cm}
\def\xmin {-10.9}   \def\xmax {1.8}
\def\ymin {-15}   \def\ymax {3.2}
\begin{pspicture*}(\xmin,\ymin)(\xmax,\ymax)
\psset{yMaxValue=\ymax,yMinValue=\ymin}
\psgrid[unit=1cm,subgriddiv=1, gridlabels=0, gridcolor=lightgray](-11,-8)(2,3)
\psaxes[labelFontSize=\displaystyle,arrowsize=3pt 2, ticksize=-2pt 2pt, Dy=2]{->}(0,0)(\xmin,\ymin)(\xmax,\ymax)[$x$,-120][$y$,210] 
\def\f{-12 x x 10 add mul 25 add mul 25 div}% définition de la fonction
\psplot[plotpoints=2000,linewidth=1.25pt]{\xmin}{\xmax}{\f}%
\end{pspicture*}
}%% fin du scalebox
\end{center}

On peut affirmer que:

\medskip

{\renewcommand{\arraystretch}{1.5}
\begin{tabularx}{\linewidth}{|X|X|X|X|}
\hline
\textbf{a.}\quad $a<0$ et $\Delta<0$
& \textbf{b.}\quad $a>0$ et $\Delta=0$ 
& \textcolor{blue}{\textbf{c.}\quad $a<0$ et $\Delta=0$}
& \textbf{d.}\quad $a<0$ et $\Delta>0$\\
\hline
\end{tabularx}}

\medskip

\begin{tabular}{@{\hspace*{0.05\linewidth}} | p{0.93\linewidth}}
La parabole est tournée vers les $y$ négatifs donc $a<0$.\newline
La parabole est tangente à l'axe des abscisses donc $\Delta=0$.\\
\textbf{Réponse c.}
\end{tabular}

\bigskip

\textbf{Question 3}

\medskip

$\cos \left (x+\dfrac{\pi}{2}\right)$ est égal à :

\medskip

{%\renewcommand{\arraystretch}{3}
\begin{tabularx}{\linewidth}{|X|X|X|X|}
\hline
\textbf{a.}\quad $\cos(x)-\sin(x)$
& \textbf{b.}\quad $\cos \left (x-\dfrac{\pi}{2}\right )$\rule[-10pt]{0pt}{25pt}
& \textbf{c.}\quad $\sin(x)$
& \textcolor{blue}{\textbf{d.}\quad $-\sin(x)$}\\
\hline
\end{tabularx}}

\medskip

\begin{tabular}{@{\hspace*{0.05\linewidth}} | p{0.93\linewidth}}
Propriété du cercle trigonométrique.\\
\textbf{Réponse d.}
\end{tabular}



\textbf{Question 4}

\medskip

Dans le plan rapporté à un repère orthonormé, on donne les points A\,$(-7~;~4)$ et B\,$(1~;~-2)$.

Le cercle $\Gamma$ de diamètre [AB] admet comme équation dans ce repère :

\medskip

{\renewcommand{\arraystretch}{1.5}
\begin{tabularx}{\linewidth}{|X|X|}
\hline
\textbf{a.}\quad $(x+7)^2+(y-4)^2=100$
& \textcolor{blue}{\textbf{b.}\quad $(x+3)^2+(y-1)^2=25$}\\
\hline
\textbf{c.}\quad $(x+3)^2+(y-1)^2=100$
& \textbf{d.}\quad $(x+7)^2+(y-4)^2=25$\\
\hline
\end{tabularx}}

\medskip

\begin{tabular}{@{\hspace*{0.05\linewidth}} | p{0.93\linewidth}}
Un cercle de centre $\Omega$ et de rayon $R$ a pour équation $\left (x-x_{\Omega}\right )^2 + \left (y - y_{\Omega} \right )^2=R^2$

Le centre du cercle de diamètre [AB] a pour coordonnées $\left (\dfrac{-7+1}{2}~;~\dfrac{4-2}{2}\right ) = (-3~;~1)$.

Le diamètre AB vaut

 $\ds\sqrt{\left ( x_{\text B} - x_{\text A}\right )^2 + \left (y_{\text B} - y_{\text A} \right )^2 }=\ds\sqrt{\left ( 1-(-7)\right )^2 + \left (-2-4 \right )^2 }=\ds\sqrt{64+36}=\sqrt{100}=10$.

Donc le rayon $R$ du cercle est égal à 5 donc $R^2=25$.

\textbf{Réponse b.}
\end{tabular}

\bigskip

\textbf{Question 5}

\medskip

Dans le plan rapporté à un repère orthonormé, les droites $\mathcal{D}$ et $\mathcal{D}'$ d'équations cartésiennes respectives $3x+2y-1=0$  et $6x+4y+2=0$ sont :

\medskip

{\renewcommand{\arraystretch}{3}
\begin{tabularx}{\linewidth}{|X|X|X|X|}
\hline
\shortstack{\textbf{a.}~ sécantes et non\\~perpendiculaires}
& \shortstack{\textbf{b.}~ confondues\\\phantom{confondues}}
& \shortstack{\blue\textbf{c.}~ strictement\\\blue{}parallèles}
& \shortstack{\textbf{d.}~ perpendiculaires\\\phantom{confondues}}\\
\hline
\end{tabularx}}

\medskip

\begin{tabular}{@{\hspace*{0.05\linewidth}} | p{0.93\linewidth}}
Les deux droites $\mathcal{D}$ et $\mathcal{D}'$ ont respectivement pour vecteurs  directeurs 
$\vect{v}\,(-2~;~3)$ et $\vect{v'}\,(-4~;~6)$; or $\vect{v'} = 2\vect{v}$, donc les deux vecteurs sont colinéaires, ce qui entraîne que les deux droites sont parallèles.

Le point A de coordonnées $(1~;~-1)$ appartient à la droite $\mathcal{D}$ car 
$3x_{\text A} +2y_{\text A} -1 = 3-2-1=0$. Mais $6x_{\text A} +4y_{\text A} +2 = 6-4+2 \neq 0$ donc $\text A \not\in \mathcal{D}'$. 

Les deux droites sont parallèles non confondues donc elles sont strictement parallèles.

\textbf{Réponse c.}
\end{tabular}

\vspace{0.5cm}

