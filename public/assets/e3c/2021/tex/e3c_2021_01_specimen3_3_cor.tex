\textbf{\large Exercice 3 \hfill 5 points}

\bigskip

Dans une école d'ingénieurs, certains étudiants s'occupent de la gestion des associations
comme par exemple le BDS (bureau des sports).

Sur les cinq années d'études, le cycle \og licence \fg{}  dure les trois premières années, et les deux
dernières années sont celles du cycle de \og   spécialisation \fg{}.

On constate que, dans cette école, il y a 40\,\% d'étudiants dans le cycle \og licence \fg{} et 60\,\%
dans le cycle de \og spécialisation \fg.

\begin{list}{\textbullet}{}
\item Parmi les étudiants du cycle \og licence \fg, 8\,\% sont membres du BDS ;
\item Parmi les étudiants du cycle de \og spécialisation \fg, 10\,\% sont membres du BDS.
\end{list}

On considère un étudiant de cette école choisi au hasard, et on considère les évènements suivants:

\begin{list}{}{}
\item $L$: \og L'étudiant est dans le cycle  licence \fg; $\overline{L}$ est son évènement contraire.
\item $B$: \og L'étudiant est membre du BDS \fg; $\overline{B}$ est son évènement contraire.
\end{list}
 
La probabilité d'un évènement $A$ est notée $P(A)$.

\bigskip

\textbf{Partie A}

\medskip

\begin{enumerate}
\item On complète l'arbre pondéré modélisant la situation.

\begin{center}
\bigskip
  \pstree[treemode=R,nodesepA=0pt,nodesepB=4pt,levelsep=2.5cm,nrot=:U]{\TR{}}
 {
 	\pstree[nodesepA=4pt]{\TR{$L$}\naput{$0,4$}}
 	  { 
 		  \TR{$B$}\naput{$\blue 0,08$}
 		  \TR{$\overline{B}$}\nbput{$\blue 0,92$}	   
 	  }
 	\pstree[nodesepA=4pt]{\TR{$\overline{L}$}\nbput{$\blue 0,6$}}
 	  {
 		  \TR{$B$}\naput{$\blue 0,10$}
 		  \TR{$\overline{B}$}\nbput{$\blue 0,90$}	   
     }
}
\bigskip
\end{center}

\item La probabilité que l'étudiant choisi soit en cycle \og licence \fg{} et membre du BDS est:

$P(L\cap B)=0,4\times 0,08 = 0,032$.

\item $P(B)=  P(L\cap B) + P(\overline L\cap B) = 0,032 + 0,6\times 0,1 = 0,092$.
\end{enumerate}

\medskip

\textbf{Partie B}

\medskip

Le BDS décide d'organiser une randonnée en montagne. Cette sortie est proposée à tous les étudiants de cette école mais le prix qu'ils auront à payer pour y participer est variable. Il est de 60~\euro{ pour les étudiants qui ne sont pas membres du BDS, et de 20~\euro{} pour les étudiants qui sont membres du BDS.

On désigne par $X$ la variable aléatoire donnant la somme à payer pour un étudiant qui désire
faire cette randonnée.

\begin{enumerate}
\item Les valeurs prises par $X$ sont 20~\euro{} et 60~\euro{}.

\item% Donner la loi de probabilité de $X$, et calculer l'espérance de $X$.
La somme à payer est de 20~\euro{} si l'étudiant est membre du BDS, c'est-à-dire avec une probabilité de $0,092$, ou de 50~\euro{} si l'étudiant n'est pas membre du BDS, c'est-à-dire avec une probabilité de $1-0,092 = 0,908$. D'où la loi de probabilité de la variable aléatoire $X$:

\begin{center}
{\renewcommand{\arraystretch}{1.5}
\begin{tabular}{|c|*{2}{>{\centering\arraybackslash}p{1cm}|}}
\hline
$x_i$ en euro & 20 & 60\\
\hline
$p_i = P(X=x_i)$ & $0,092$  & $0,908$\\
\hline
\end{tabular}}
\end{center}

L'espérance mathématique de la variable aléatoire $X$ est:

$E(X) = \sum (x_i\times p_i) = 20\times 0,092 + 60\times 0,908 = 56,32$~\euro.

\end{enumerate}

\vspace{0,5cm}

