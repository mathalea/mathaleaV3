
\medskip

\begin{tabularx}{\linewidth}{|X|}\hline
\textbf{Principaux domaines abordés :
Fonction logarithme ; dérivation.}\\ \hline
\end{tabularx}

\bigskip

\textbf{Partie I: Étude d'une fonction auxiliaire}

\medskip

Soit $g$ la fonction définie sur $]0~;~+\infty[$ par :

\[g(x) = \ln(x) + 2x - 2.\]

\smallskip

\begin{enumerate}
\item Déterminer les limites de $g$ en $+\infty$ et $0$.
\item Déterminer le sens de variation de la fonction $g$ sur $]0~;~ +\infty[$.
\item Démontrer que l'équation $g(x) = 0$ admet une unique solution $\alpha$ sur $]0~;~ +\infty[$. 
\item Calculer $g(1)$ puis déterminer le signe de $g$ sur $]0~;~ +\infty[$.
\end{enumerate}

\bigskip

\textbf{Partie II : Étude d'une fonction } \boldmath $f$\unboldmath

\medskip

On considère la fonction $f$, définie sur $]0~;~ +\infty[$ par: 

\[f(x) = \left(2 - \dfrac{1}{x}\right)\left(\ln (x) - 1\right).\]

\begin{enumerate}
\item 
	\begin{enumerate}
		\item On admet que la fonction $f$ est dérivable sur $]0~;~ +\infty[$ et on note $f'$ sa dérivée.

Démontrer que, pour tout $x$ de $]0~;~ +\infty[$, on a : 

\[f'(x) = \dfrac{g(x)}{x^2}.\]

		\item Dresser le tableau de variation de la fonction $f$ sur $]0~;~ +\infty[$. Le calcul des limites n'est pas demandé.
	\end{enumerate}
\item  Résoudre l'équation $f(x) = 0$ sur $]0~;~ +\infty[$ puis dresser le tableau de signes de $f$ sur l'intervalle $]0~;~ +\infty[$.
\end{enumerate}

\bigskip

\textbf{Partie III : Étude d'une fonction \boldmath $F$\unboldmath{} admettant pour dérivée la fonction \boldmath $f$\unboldmath}

\medskip

On admet qu'il existe une fonction $F$ dérivable sur $]0~;~ +\infty[$ dont la dérivée $F'$ est la fonction $f$.

Ainsi, on a : $F' = f$.

On note $\mathcal{C}_F$ la courbe représentative de la fonction $F$ dans un repère orthonormé \Oij. On ne cherchera pas à déterminer une expression de $F(x)$.

\medskip

\begin{enumerate}
\item Étudier les variations de $F$ sur $]0~;~ +\infty[$.
\item La courbe $\mathcal{C}_F$ représentative  de $F$ admet-elle des tangentes parallèles à l'axe des abscisses ?

Justifier la réponse.
\end{enumerate}
