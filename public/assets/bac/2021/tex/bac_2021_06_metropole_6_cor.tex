\textbf{\large\textsc{Exercice 1} \hfill 4 points}

\textbf{Commun à tous les candidats}

\medskip

%\emph{Cet exercice est un questionnaire à choix multiples. Pour chacune des questions suivantes, une seule des quatre réponses proposées est exacte. \\
%Une réponse exacte rapporte un point. Une réponse fausse, une réponse multiple ou l'absence de réponse à une question ne rapporte ni n'enlève de point.\\
%Pour répondre, indiquer sur la copie le numéro de la question et la lettre de la réponse choisie.\\
%Aucune justification n'est demandée.}
%
%\medskip
%
%L'espace  est rapporté à un repère orthonormé \Oijk. 
%
%On considère :
%
%\setlength\parindent{9mm}
%\begin{itemize}
%\item[$\bullet~~$]La droite $\mathcal{D}$ passant par les points A$(1~;~1~;~ -2)$ et B$(-1~;~3~;~2)$. 
%\item[$\bullet~~$]La droite $\mathcal{D}'$ de représentation paramétrique :
%$\left\{\begin{array}{l c l}
%x &=& -4 +3t\\
%y&=&\phantom{-}6 - 3t\\
%z&=&\phantom{-}8 - 6t
%\end{array}\right.$\quad avec $t \in \R$.
%\item[$\bullet~~$]Le plan $\mathcal{P}$ d'équation cartésienne $x + my- 2z + 8 = 0$ où $m$ est un nombre réel.
%\end{itemize}
%\setlength\parindent{0mm}
%
%\medskip

\textbf{Question 1} : %Parmi les points suivants, lequel appartient à la droite $\mathcal{D}'$ ?
On voit que pour $t = 5$, les coordonnées du point de la droite $\mathcal{D}'$ sont $(11~;~- 9~;~- 22)$ soit les coordonnées de M$_2$.

\smallskip

\hfill\textbf{Réponse b.}

\medskip

\textbf{Question 2} : Un vecteur directeur de la droite $\mathcal{D}'$ est : $\vect{u}_3\begin{pmatrix}3\\- 3\\-6\end{pmatrix}$.

\smallskip

\hfill\textbf{Réponse c.}

\medskip

%\begin{center}

\textbf{Question 3} : %Les droites $\mathcal{D}$ et $\mathcal{D}'$ sont :
%\begin{center}
%\begin{tabularx}{\linewidth}{*{4}{X}}
%\textbf{a.~~}sécantes &\textbf{b.~~} strictement parallèles&\textbf{c.~~}non coplanaires&\textbf{d.~~}confondues
%\end{tabularx}
%\end{center}

Un vecteur directeur de la droite $\mathcal{D}$ est $\vect{\text{AB}}\begin{pmatrix}-2\\2\\4\end{pmatrix}$ colinéaire au vecteur $\dfrac{1}{2}\vect{\text{AB}}\begin{pmatrix}-1\\1\\2\end{pmatrix}$

Un vecteur directeur de la droite $\mathcal{D}'$ est $\vect{u_3}\begin{pmatrix}3\\- 3\\-6\end{pmatrix}$ colinéaire au vecteur $\dfrac{1}{3}\vect{u_3}\begin{pmatrix}1\\- 1\\-2\end{pmatrix}$ ou encore colinéaire au vecteur $-\dfrac{1}{3}\vect{u_3}\begin{pmatrix}-1\\+ 1\\2\end{pmatrix}$.

Les droites $\mathcal{D}$ et $\mathcal{D}'$ ayant des vecteurs directeurs colinéaires au même vecteur sont donc parallèles.

De plus en remplaçant $t$ par $\dfrac{5}{3}$ dans l'équation paramétrique de $\mathcal{D}'$ on obtient $x = 1,~\, y = 1$ et $z = - 2$.

Les droites sont parallèles et ont un point commun : elles sont donc confondues.

\smallskip

\hfill\textbf{Réponse d.}

\medskip

\textbf{Question 4} : %La valeur du réel $m$ pour laquelle la droite $\mathcal{D}$ est parallèle au plan $\mathcal{P}$ est :
%\begin{center}
%\begin{tabularx}{\linewidth}{*{4}{X}}
%\textbf{a.~~}$m =-1$ &\textbf{a.~~} $m = 1$&\textbf{c.~~}$m = 5$&\textbf{d.~~}$m =-2$
%\end{tabularx}
%\end{center}
$\mathcal{P}$ a pour vecteur normal $\vect{p}\begin{pmatrix}1\\m\\-2\end{pmatrix}$.

$\mathcal{D}$ est parallèle au plan $\mathcal{P}$ si $\vect{\text{AB}}$ et $\vect{p}$ sont orthogonaux, soit :

$\vect{\text{AB}} \cdot \vect{p} = 0 \iff -2 \times 1 + 2 \times m + 4 \times (- 2) = 0 \iff - 2 + 2m - 8 = 0 \iff 2m = 10 \iff m = 5$. 

\smallskip

\hfill\textbf{Réponse c.}

\medskip

\bigskip

