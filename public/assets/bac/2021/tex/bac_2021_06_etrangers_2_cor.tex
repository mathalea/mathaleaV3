\textbf{\large\textsc{Exercice 2} \hfill 5 points}

\textbf{Commun à tous les candidats}

\medskip

Dans tout cet exercice, les probabilités seront arrondies, si nécessaire, à $10^{-3}$.

D'après une étude, les utilisateurs réguliers de transports en commun représentent 17\,\% de la population française. 

Parmi ces utilisateurs réguliers, 32\,\% sont des jeunes âgés de 18 à 24 ans. (Source : TNS-Sofres)

\bigskip

\textbf{Partie A}

\medskip
 
On interroge une personne au hasard et on note :

\setlength\parindent{9mm}
\begin{itemize}
\item $R$ l'évènement : \og La personne interrogée utilise régulièrement les transports en commun \fg.
\item $J$ l'évènement : \og La personne interrogée est âgée de 18 à 24 ans \fg.
\end{itemize}
\setlength\parindent{0mm}

%\medskip

\begin{enumerate}
\item On représente la situation à l'aide de cet arbre pondéré:

\begin{center}
\bigskip
\pstree[treemode=R,nodesepB=3pt,nodesepA=0pt,levelsep=3.5cm,nrot=:U]{\TR{}}
{\pstree[nodesepA=3pt]{\TR{$R$}\naput{$0,17$}}
		{\TR{$J$}\naput{$0,32$}
		\TR{$\overline{J}$}\nbput{$\blue 1-0,32=0,68$}
		}
\pstree[nodesepA=3pt]{\TR{$\overline{R}$~~}\nbput{$\blue 1-0,17=0,83$}}
		{\TR{$J$}
		\TR{$\overline{J}$}
		}
}
\end{center}
\item $P(R \cap J)= 0,17 \times 0,32 = \np{0,0544}$

\item D'après cette même étude, les jeunes de 18 à 24 ans représentent 11\,\% de la population française, donc $P(J)=0,11$.

La probabilité que la personne interrogée soit un jeune de 18 à 24 ans n'utilisant pas régulièrement les transports en commun est $P\left (\overline{R} \cap J\right )$.

D'après la formule des probabilités totales:

$P(J)= P(R\cap J) + P\left (\overline{R} \cap J\right )$ donc $P\left (\overline{R} \cap J\right ) = P(J) - P(R\cap J) = 0,11-0,0544 = \np{0,0556}$

soit $0,056$ à $10^{-3}$ près.


\item $P_{\overline{R}}(J)=\dfrac{P\left (\overline{R}\cap J\right )}{P\left (\overline{R}\right )} = \dfrac{0,056}{0,83} \approx \np{0,0675}$

La proportion de jeunes de 18 à 24 ans parmi les utilisateurs non réguliers des transports en commun est donc d'environ $6,75\,\%$.
\end{enumerate}

\bigskip

\textbf{Partie B}

\medskip

Lors d'un recensement sur la population française, un recenseur interroge au hasard 50 personnes en une journée sur leur pratique des transports en commun. 
La population française est suffisamment importante pour assimiler ce recensement à un tirage avec remise.
Soit $X$ la variable aléatoire dénombrant les personnes utilisant régulièrement les transports en commun parmi les 50 personnes interrogées.

%\medskip

\begin{enumerate}
\item %Déterminer, en justifiant, la loi de $X$ et préciser ses paramètres.
\begin{list}{\textbullet}{}
\item On interroge une personne au hasard et il n'y a que deux possibilités: elle utilise régulièrement les transports en commun, avec une probabilité $p=0,17$, ou pas, avec une probabilité de $1-p=0,83$.
\item On réalise $n=50$ fois ce questionnement de façon identique.
\end{list}

Donc la variable aléatoire $X$ qui donne le nombre de personnes utilisant régulièrement les transports en commun parmi les 50 personnes interrogées suit la loi binomiale de paramètres $n=50$ et $p=0,17$.

\item $P(X = 5) = \ds\binom{50}{5}\times 0,17^5 \times (1-0,17)^{50-5}\approx 0,069$
% et interpréter le résultat.

Il y a donc une probabilité de $0,069$ que, sur 50 personnes interrogées, exactement 5 prennent régulièrement les transports en commun.

\item Le recenseur indique qu'il y a plus de 95\,\% de chance pour que, parmi les $50$ personnes interrogées, moins de $13$ d'entre elles utilisent régulièrement les transports en commun.

Autrement dit, le recenseur affirme que $P(X<13) \geqslant 0,95$.

Or $P(X<13)= P(X \leqslant 12) \approx  0,929 < 0,95$ donc cette affirmation est fausse.

\item Le nombre moyen de personnes utilisant régulièrement les transports en commun parmi les $50$ personnes interrogées est
$E(X)=np=50\times 0,17 = 8,5$.
\end{enumerate}

\bigskip

