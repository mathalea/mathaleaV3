\textbf{\large Exercice B}

\medskip

\begin{tabularx}{\linewidth}{|X|}\hline
\textbf{Principaux domaines abordés :
Fonction logarithme ; dérivation.}\\ \hline
\end{tabularx}

\bigskip

\textbf{Partie I: étude d'une fonction auxiliaire}

\medskip

Soit $g$ la fonction définie sur $]0~;~+\infty[$ par: $g(x) = \ln(x) + 2x - 2$.

\smallskip

\begin{enumerate}
\item On détermine les limites de $g$ en $+\infty$ et $0$.

$\left.
\begin{array}{@{} r}
\ds\lim_{x\to +\infty} \ln (x) = +\infty\\
\ds\lim_{x\to +\infty} 2x -2= +\infty
\end{array}
\right \rbrace
\implies
\ds\lim_{x\to +\infty} g(x) = +\infty$

$\left.
\begin{array}{@{} r}
\ds\lim_{x\to 0\atop x>0} \ln (x) = -\infty\\
\ds\lim_{x\to 0} 2x-2 = -2
\end{array}
\right \rbrace
\implies
\ds\lim_{x\to 0\atop x>0} g(x) = -\infty$

\item La fonction $g$ est dérivable sur $]0~;~ +\infty[$, et $g'(x)=\dfrac{1}{x}+2>0$; donc la fonction $g$ est strictement croissante sur $]0~;~ +\infty[$.

\item %Démontrer que l'équation $g(x) = 0$ admet une unique solution $\alpha$ sur $]0~;~ +\infty[$. 
On établit le tableau des variations de la fonction $g$:

\begin{center}
{\renewcommand{\arraystretch}{1.3}
\def\esp{\hspace*{3cm}}
\psset{nodesep=3pt,arrowsize=2pt 3}  % paramètres
$\begin{array}{|c|l *2{c}|}
\hline
 x & 0   & \esp & +\infty \\
 \hline
  & \vline\;\vline\:  &    & \Rnode{max}{+\infty}   \\
g(x) & \vline\;\vline\: &  &  \\
 &   \vline\;\vline\:  \Rnode{min}{-\infty} & & 
\ncline{->}{min}{max}
\rput*(-2,0.68){\Rnode{zero}{\blue 0}}
\rput(-2,1.7){\Rnode{alpha}{\blue \alpha}}
\ncline[linestyle=dotted, linecolor=blue]{alpha}{zero}\\
\hline
\end{array}$
}
\end{center}

D'après ce tableau de variations, on peut dire que l'équation $g(x) = 0$ admet une unique solution $\alpha$ sur $]0~;~ +\infty[$. 

\item $g(1)=0$ donc $\alpha=1$.% puis déterminer le signe de $g$ sur $]0~;~ +\infty[$.

On en déduit que $g(x)<0$ sur $]0~;~1[$, et que $g(x) >0$ sur $]1~;~+\infty[$.

\end{enumerate}

\bigskip

\textbf{Partie II : étude d'une fonction } \boldmath $f$\unboldmath

\medskip

On considère la fonction $f$, définie sur $]0~;~ +\infty[$par: $f(x) = \left(2 - \dfrac{1}{x}\right)[\ln (x) - 1]$.

\begin{enumerate}
\item 
	\begin{enumerate}
		\item On admet que la fonction $f$ est dérivable sur $]0~;~ +\infty[$ et on note $f'$ sa dérivée.

%Démontrer que, pour tout $x$ de $]0~;~ +\infty[$, on a :  $f'(x) = \dfrac{g(x)}{x^2}$.

Pour tout $x$ de $]0~;~ +\infty[$, on a:

$f'(x) = \left (\dfrac{1}{x^2}\right )\left (\ln(x)-1\right ) + \left (2-\dfrac{1}{x}\right )\left (\dfrac{1}{x}\right )
= \dfrac{\ln(x)-1 +2x-1}{x^2}
= \dfrac{\ln(x)+2x-2}{x^2}
=\dfrac{g(x)}{x^2}$

		\item %Dresser le tableau de variation de la fonction $f$ sur $]0~;~ +\infty[$. Le calcul des limites n'est pas demandé.
Sur $]0~;~ +\infty[$, $x^2>0$ donc $f'(x)$ est du signe de $g(x)$ qui s'annule pour $x=1$.

$f(1) = \left (2-\dfrac{1}{1}\right )\left (\ln(1)-1\right ) = -1$

On dresse le tableau de variations de $f$:

\begin{center}
{\renewcommand{\arraystretch}{1.3}
\psset{nodesep=3pt,arrowsize=2pt 3}  % paramètres
\def\esp{\hspace*{1.5cm}}% pour modifier la largeur du tableau
\def\hauteur{0pt}% mettre au moins 20pt pour augmenter la hauteur
$\begin{array}{|c| l *3{c} c|}
\hline
 x & 0 & \esp & 1 & \esp & +\infty \\
 \hline
g(x) & \vline\;\vline\; &  \pmb{-} & \vline\hspace{-2.7pt}0 & \pmb{+} & \\  
\hline
f'(x) &  \vline\;\vline\;  &  \pmb{-} & \vline\hspace{-2.7pt}0 & \pmb{+} & \\  
\hline
  & \vline\;\vline\;  \Rnode{max1}{}  &  &  &  & \Rnode{max2}{}   \\
f(x) & \vline\;\vline\;  &  & & &  \rule{0pt}{\hauteur} \\
 &  \vline\;\vline\;  & &   \Rnode{min}{-1} & & \rule{0pt}{\hauteur}
\ncline{->}{max1}{min} \ncline{->}{min}{max2}\\
\hline
\end{array}$
}
\end{center}		
		
	\end{enumerate}
\item  %Résoudre l'équation $f(x) = 0$ sur $]0~;~ +\infty[$ puis dresser le tableau de signes de $f$ sur l'intervalle $]0~;~ +\infty[$.
$f(x)=0 \iff \left (2-\dfrac{1}{x}\right )\left ( \ln(x)-1\right )=0
\iff 2-\dfrac{1}{x}=0 \text{ ou } \ln(x)-1=0\\[5pt]
\phantom{f(x)=0}
\iff 2=\dfrac{1}{x} \text{ ou } \ln(x)=1
\iff x=\dfrac{1}{2} \text{ ou } x=\e$

L'équation $f(x) = 0$ admet donc deux solutions sur $]0~;~ +\infty[$: $x=\dfrac{1}{2}$ et $x=\e$.

On complète le tableau de variations de $f$ en intégrant les solutions de l'équation $f(x)=0$:

\begin{center}
{\renewcommand{\arraystretch}{1.3}
\psset{nodesep=3pt,arrowsize=2pt 3}  % paramètres
\def\esp{\hspace*{1.5cm}}% pour modifier la largeur du tableau
\def\hauteur{0pt}% mettre au moins 20pt pour augmenter la hauteur
$\begin{array}{|c| l *3{c} c|}
\hline
 x & 0 & \esp & 1 & \esp & +\infty \\
\hline
  & \vline\;\vline\; \Rnode{max1}{}  &  &  &  & \Rnode{max2}{}   \\
f(x) &  \vline\;\vline\;  &  & & &  \rule{0pt}{\hauteur} \\
 &  \vline\;\vline\;  &  &   \Rnode{min}{-1} & & \rule{0pt}{\hauteur}
\ncline{->}{max1}{min} \ncline{->}{min}{max2}
\rput*(-3.7,0.55){\Rnode{zero}{\blue 0}}
\rput(-3.7,1.78){\Rnode{alpha}{\blue \frac{1}{2}}}
\ncline[linestyle=dotted, linecolor=blue]{alpha}{zero}
\rput*(-1.3,0.65){\Rnode{zero2}{\red 0}}
\rput(-1.3,1.78){\Rnode{beta}{\red \e}}
\ncline[linestyle=dotted, linecolor=red]{beta}{zero2}
\\
\hline
\end{array}$
}
\end{center}

On en déduit le tableau de signes de la fonction $f$ sur $]0~;~ +\infty[$:

\begin{center}
{
\renewcommand{\arraystretch}{1.5}
\def\esp{\hspace*{2cm}}
$\begin{array}{|c |l *{6}{c} |} 
\hline
x  & 0 & \esp & \frac{1}{2} & \esp & \e & \esp & +\infty \\
\hline
f(x) & \vline\;\vline\; & \pmb{+} &  \vline\hspace{-2.7pt}{0} & \pmb{-} & \vline\hspace{-2.7pt}{0} & \pmb{+} &\\
\hline
\end{array}$
}
\end{center}

\end{enumerate}

\bigskip

\textbf{Partie III : étude d'une fonction \boldmath $F$\unboldmath{} admettant pour dérivée la fonction \boldmath $f$\unboldmath}

\medskip

On admet qu'il existe une fonction $F$ dérivable sur $]0~;~+\infty[$ dont la dérivée $F'$ est la fonction $f$.
Ainsi, on a : $F' = f$.
On note $\mathcal{C}_F$ la courbe représentative de la fonction $F$ dans un repère orthonormé \Oij. %On ne cherchera pas à  déterminer une expression de $F(x)$.

\begin{enumerate}
\item %à‰tudier les variations de $F$ sur $]0~;~ +\infty[$.
Par définition $F'= f$, donc le signe de $F'(x)$ est celui de $f(x)$. On en déduit les variations de la fonction $F$ sur $]0~;~ +\infty[$:

\begin{center}
{
\renewcommand{\arraystretch}{1.5}
\def\esp{\hspace*{2cm}}
$\begin{array}{|c |l *{6}{c} |} 
\hline
x  & 0 & \esp & \frac{1}{2} & \esp & \e & \esp & +\infty \\
\hline
F'(x)=f(x) & \vline\;\vline\; & \pmb{+} &  \vline\hspace{-2.7pt}{0} & \pmb{-} & \vline\hspace{-2.7pt}{0} & \pmb{+} &\\
\hline
F & \vline\;\vline\; & F \text{ croissante} &  \vline\hspace{-2.7pt}{\phantom 0} & F \text{ décroissante} & \vline\hspace{-2.7pt}{\phantom 0} & F \text{ croissante} &\\
\hline
\end{array}$
}
\end{center}
\item Le coefficient directeur de la tangente en $x=a$ à  la courbe $\mathcal{C}_F$ représentative  de $F$ est $F'(a)$ soit $f(a)$. Pour que $\mathcal{C}_F$ admette des tangentes parallèles à l'axe des abscisses, il faut trouver des valeurs de $x$ pour lesquelles $F'(x)=0$ c'est-à -dire $f(x)=0$.

D'après les questions précédentes, on peut dire  $\mathcal{C}_F$ admet deux tangentes parallèles à  l'axe des abscisses, en $x=\frac{1}{2}$ et en $x=\e$.
\end{enumerate}
