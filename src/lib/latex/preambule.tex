% pb avec les symboles mathématiques particuliers dans le "titre" de l'exercice -> Redéfinition de la commande \exo
% Pb de "dimension too large" à la première compilation
% Pb des fontes amssymb chargé alors que fourier-otf les inclue déjà -> suppression de amssymb
% Imposition d'une compilation en lualatex ?
% babel mis à la fin du préambule
% suppression du package enumerate : à voir si d'éventuels pb de compilation
% remplacement de lmodern par fourier par cohérence entre les deux compilations (luatex/reste)
% suppression du package graphics (obsolete par rapport à graphicx)
% remplacement du package amsmath par mathtools. Et donc suppression de amsfonts : pas possible pour \Leadto
% remplacement du package numprint par siunitx ?
% le theme s'écrit après la fin du préambule et non dans le fichier.tex ?
% package mdframed pose soucis. Remplacement par une macro tikz ?
% transformation automatique des \degree en \degres de french.

\usepackage{mathtools,amssymb,amsfonts,mathrsfs} 

\usepackage{unicode-math}
\setmainfont{TeX Gyre Schola}
\setmathfont{TeX Gyre Schola Math}

%\newfontfamily\myfontScratch[]{FreeSans}

\setmathfont{STIX Two Math}[
  range={\blacktriangledown}%\mdlgblklozenge,\mdblksquare,\blacktriangle,\bigstar
]

%%% COULEURS %%%
\usepackage[table,svgnames]{xcolor}
\definecolor{nombres}{cmyk}{0,.8,.95,0}
\definecolor{gestion}{cmyk}{.75,1,.11,.12}
\definecolor{gestionbis}{cmyk}{.75,1,.11,.12}
\definecolor{grandeurs}{cmyk}{.02,.44,1,0}
\definecolor{geo}{cmyk}{.62,.1,0,0}
\definecolor{algo}{cmyk}{.69,.02,.36,0}
\definecolor{correction}{cmyk}{.63,.23,.93,.06}
\arrayrulecolor{couleur_theme} % Couleur des filets des tableaux

\usepackage[autolanguage,np]{numprint}

\newcommand\degree{\degres}

\usepackage[most]{tcolorbox}
\tcbuselibrary{documentation}

\usepackage{hyperref}
\usepackage{multicol} 
\usepackage{calc}

\usepackage{enumitem}
\usepackage{graphicx}
\usepackage{tabularx}

\setlength{\parindent}{0mm}
\renewcommand{\arraystretch}{1.5}
\renewcommand{\labelenumi}{\textbf{\theenumi{}.}}
\renewcommand{\labelenumii}{\textbf{\theenumii{}.}}
\setlength{\fboxsep}{3mm}

\setlength{\headheight}{14.5pt}

\setlength{\premulticols}{0pt}

\newcounter{ExoMA}

\newlength{\parindentMA}%
\setlength{\parindentMA}{\parindent}
\addtolength{\parindentMA}{30pt}

\usepackage{simplekv}

\setKVdefault[Theme]{Coopmath,Libres=false,CANs=false}
\defKV[Theme]{Libre=\setKV[Theme]{Coopmath=false}\setKV[Theme]{Libres}}
\defKV[Theme]{CAN=\setKV[Theme]{Coopmath=false}\setKV[Theme]{CANs}}

\usepackage{fancyhdr}%,lastpage}
\fancyhead[C]{}%
\fancyfoot{}%

\NewDocumentCommand\Theme{o m m m m}{%
  \useKVdefault[Theme]%
  \setKV[Theme]{#1}%
  \ifboolKV[Theme]{CANs}{%
    \usepackage[a4paper,left=1.5cm,right=1.5cm,top=2cm,bottom=2cm]{geometry}%
    \fancypagestyle{premierePage}{%
      \fancyhead[C]{\textsc{Course aux nombres}}%
      \fancyfoot[R]{\scriptsize Coopmaths.fr -- CC-BY-SA}%
      \setlength{\headheight}{14.5pt}%
    }%
    \pagestyle{premierePage}
    \colorlet{couleur_theme}{black}%
    \colorlet{couleur_numerotation}{couleur_theme}%
    \let\EXO\EXOCan\let\endEXO\endEXOCan%
  }{%
    \usepackage[a4paper,left=1.5cm,right=1.5cm,top=4cm,bottom=2cm]{geometry}
    \renewcommand{\headrulewidth}{0pt}
    \renewcommand{\footrulewidth}{0pt}
    \pagestyle{fancy}
    \ifboolKV[Theme]{Libres}{%
      \let\EXO\EXOlibre\let\endEXO\endEXOlibre%
      \colorlet{couleur_theme}{black}%
      \colorlet{couleur_numerotation}{couleur_theme}%
      \fancyhead[L]{}
      \fancyhead[R]{}
    }{%
      \theme{#2}{#3}{#4}{#5}%
      \let\EXO\EXOcoop\let\endEXO\endEXOcoop%
    }%
  }%
}%

\NewDocumentEnvironment{EXOCan}{m m +b}{%
  #3
}

\NewDocumentEnvironment{EXOcoop}{m m +b}{%
  \begin{tcolorbox}[%
    enhanced,
    breakable,
    colback=white,
    colframe=white,%couleur_theme,
    coltitle=black,
    title=\IfNoValueTF{#1}{}{\textmd{#1}},%
    overlay unbroken and first={%
      \IfNoValueTF{#2}{}{%
        \node[%
        fill=white,
        anchor=east,
        xshift=10pt,
        text=black
        ]
        at (frame.north east)
        {\scriptsize #2};
      }
      \node[anchor=west,xshift=-20pt,yshift=-15pt] at (frame.north west) {\stepcounter{ExoMA}\numb{\arabic{ExoMA}}};
    }
    ]
    #3
  \end{tcolorbox}
}

\NewDocumentEnvironment{EXOlibre}{m m +b}{%
  \begin{tcolorbox}[%
    enhanced,
    breakable,
    colback=white,
    colframe=white,%couleur_theme,
    coltitle=black,
    title=\stepcounter{ExoMA}\textbf{Exercice \arabic{ExoMA}},%
    ]
    \IfNoValueTF{#1}{}{#1\par}%
    #3
  \end{tcolorbox}%
}

%%%%%%%%%%%%%%%%%%%%%%%%%%%%%%%%%%%%%%%%%%
% SPÉCIFIQUE SUJETS CAN                  %
%%%%%%%%%%%%%%%%%%%%%%%%%%%%%%%%%%%%%%%%%%

\usepackage{longtable}

\tikzset{
  mybox/.style={
    rectangle,
    drop shadow, 
    inner sep=17pt,
    draw=gray,
    shade,
    top color=gray,
    every shadow/.append style={fill=gray!40}, 
    bottom color=gray!20
    }
  }
  
  \newcommand\MyBox[2][]{%
    \tikz\node[mybox,#1] {#2}; 
  }
  % Un compteur pour les questions CAN
  \newcounter{nbEx}
  % Pour travailler avec les compteurs
  \usepackage{totcount}
  \regtotcounter{nbEx}  

  % Une checkmark !
  \def\checkmark{\tikz\fill[scale=0.4](0,.35) -- (.25,0) -- (1,.7) -- (.25,.15) -- cycle;}  
  % Repiqué sans vergogne dans lemanuel TikZ pour l'impatient
  \def\arete{3}   \def\epaisseur{5}   \def\rayon{2}

  \newcommand{\ruban}{(0,0)
    ++(0:0.57735*\arete-0.57735*\epaisseur+2*\rayon)
    ++(-30:\epaisseur-1.73205*\rayon)
    arc (60:0:\rayon)   -- ++(90:\epaisseur)
    arc (0:60:\rayon)   -- ++(150:\arete)
    arc (60:120:\rayon) -- ++(210:\epaisseur)
    arc (120:60:\rayon) -- cycle}

  \newcommand{\dureeCan}{[Temps total à modifier juste après le \textbackslash begin\{document\}]}  

  \newcommand{\titreSujetCan}{\textbf{[Ligne ci-dessous à modifier juste après le \textbackslash begin\{document\}]\\Sujet niveau NN - Mois Année}}

  \newcommand{\mobiusCan}{
    % Repiqué sans vergogne dans lemanuel TikZ pour l'impatient
    \begin{tikzpicture}[very thick,top color=white,bottom color=gray,scale=1.2]
      \shadedraw \ruban;
      \shadedraw [rotate=120] \ruban;
      \shadedraw [rotate=-120] \ruban;
      \draw (-60:4) node[scale=5,rotate=30]{CAN};
      \draw (180:4) node[scale=3,rotate=-90]{MathALEA};
      \clip (0,-6) rectangle (6,6); % pour croiser
      \shadedraw  \ruban;
      \draw (60:4) node [gray,xscale=2.5,yscale=2.5,rotate=-30]{CoopMaths};
    \end{tikzpicture} 
  }
  
  \newcommand{\pageDeGardeCan}[1]{
    % #1 --> nom du compteur pour le nombre de questions

    %\vspace*{10mm}
    \textsc{Nom} : \makebox[.35\linewidth]{\dotfill} \hfill \textsc{Prénom} : \makebox[.35\linewidth]{\dotfill}

    \vspace{10mm}
    \textsc{Classe} : \makebox[.33\linewidth]{\dotfill} \hfill
    \MyBox{\Large\textsc{Score} : \makebox[.15\linewidth]{\dotfill} / \total{#1}}      
    \par\medskip \hrulefill \par
    \checkmark \textit{\textbf{Durée :  \dureeCan~minutes}}

    \smallskip
    \checkmark \textit{L'épreuve comporte \total{#1} questions.}

    \smallskip  
    \checkmark \textit{L'usage de la calculatrice et du brouillon sont interdits.}

    \smallskip
    \checkmark \textit{Il n'est pas permis d'écrire des calculs intermédiaires.}
    \par \hrulefill \par\vspace{5mm}
    \begin{center}
      \textsc{\titreSujetCan}
      \par\vspace{5mm}
      \mobiusCan
    \end{center}
  }

  \newlength{\Largeurcp}
  
  % Structure globale pour les tableaux des livrets CAN
  \NewDocumentEnvironment{TableauCan}{+b}{%
    % #1 --> corps de tableau
    \setlength{\Largeurcp}{0.35\textwidth-8\tabcolsep}
    \renewcommand*{\arraystretch}{2.5}
    \begin{spacing}{1.1}
      \begin{longtable}{|>{\columncolor{gray!20}\centering}m{0.05\textwidth}|>{\centering}m{0.45\textwidth}|>{\centering}m{\Largeurcp}|>{\centering}p{0.1\textwidth}|}%
        \hline
        \rowcolor{gray!20}\#&Énoncé&Réponse&Jury\tabularnewline \hline
        % \endfirsthead
        % \hline
        % \rowcolor{gray!20}\#&Énoncé&Réponse&Jury\tabularnewline \hline
        % \endhead
        #1
      \end{longtable}
    \end{spacing}
    \renewcommand*{\arraystretch}{1}
  }

%%% MISE EN PAGE %%%

\usepackage{setspace}

\usepackage{tkz-tab,tkz-fct}
\usepackage{tkz-euclide}
\usepackage{pgf}%
\usepackage{pgfplots}
\pgfplotsset{compat=1.17}
\usetikzlibrary{babel,arrows,calc,fit,patterns,plotmarks,shapes.geometric,shapes.misc,shapes.symbols,shapes.arrows,shapes.callouts, shapes.multipart, shapes.gates.logic.US,shapes.gates.logic.IEC, er, automata,backgrounds,chains,topaths,trees,petri,mindmap,matrix, calendar, folding,fadings,through,positioning,scopes,decorations.fractals,decorations.shapes,decorations.text,decorations.pathmorphing,decorations.pathreplacing,decorations.footprints,decorations.markings,shadows}

\spaceskip=2\fontdimen2\font plus 3\fontdimen3\font minus3\fontdimen4\font\relax %Pour doubler l'espace entre les mots
\newcommand\numb[1]{ % Dessin autour du numéro d'exercice
  \begin{tikzpicture}[overlay,scale=.8]%,yshift=-.3cm
    \draw[fill=couleur_numerotation,couleur_numerotation](-.3,0)rectangle(.5,.8);
    \draw[line width=.05cm,couleur_numerotation,fill=white] (0,0)--(.5,.5)--(1,0)--(.5,-.5)--cycle;
    \draw (0.5,0) node[anchor=center,couleur_numerotation]{\large\bfseries#1};
    \draw (-.4,.8) node[white,anchor=north west]{\bfseries EX}; 
  \end{tikzpicture}
}

% echelle pour le dé
\def\globalscale {0.04}
% abscisse initiale pour les chevrons
\def\xini {3}

\newcommand\theme[4]{%
  \fancyhead[C]{%
    % Tracé du dé
    \begin{tikzpicture}[y=0.80pt, x=0.80pt, yscale=-\globalscale, xscale=\globalscale,remember picture, overlay,transform canvas={xshift=-0.5\linewidth,yshift=2.7cm},fill=couleur_theme]%,xshift=17cm,yshift=9.5cm
      %%%% Arc supérieur gauche%%%%
      \path[fill](523,1424)..controls(474,1413)and(404,1372)..(362,1333)..controls(322,1295)and(313,1272)..(331,1254)..controls(348,1236)and(369,1245)..(410,1283)..controls(458,1328)and(517,1356)..(575,1362)..controls(635,1368)and(646,1375)..(643,1404)..controls(641,1428)and(641,1428)..(596,1430)..controls(571,1431)and(538,1428)..(523,1424)--cycle;
      %%%% Dé face supérieur%%%%
      \path[fill](512,1272)..controls(490,1260)and(195,878)..(195,861)..controls(195,854)and(198,846)..(202,843)..controls(210,838)and(677,772)..(707,772)..controls(720,772)and(737,781)..(753,796)..controls(792,833)and(1057,1179)..(1057,1193)..controls(1057,1200)and(1053,1209)..(1048,1212)..controls(1038,1220)and(590,1283)..(551,1282)..controls(539,1282)and(521,1278)..(512,1272)--cycle;
      %%%% Dé faces gauche et droite%%%%
      \path[fill](1061,1167)..controls(1050,1158)and(978,1068)..(900,967)..controls(792,829)and(756,777)..(753,756)--(748,729)--(724,745)..controls(704,759)and(660,767)..(456,794)..controls(322,813)and(207,825)..(200,822)..controls(193,820)and(187,812)..(187,804)..controls(188,797)and(229,688)..(279,563)..controls(349,390)and(376,331)..(391,320)..controls(406,309)and(462,299)..(649,273)..controls(780,254)and(897,240)..(907,241)..controls(918,243)and(927,249)..(928,256)..controls(930,264)and(912,315)..(889,372)..controls(866,429)and(848,476)..(849,477)..controls(851,479)and(872,432)..(897,373)..controls(936,276)and(942,266)..(960,266)..controls(975,266)and(999,292)..(1089,408)..controls(1281,654)and(1290,666)..(1290,691)..controls(1290,720)and(1104,1175)..(1090,1180)..controls(1085,1182)and (1071,1176)..(1061,1167)--cycle;
      %%%% Arc inférieur bas%%%%
      \path[fill](1329,861)..controls(1316,848)and(1317,844)..(1339,788)..controls(1364,726)and(1367,654)..(1347,591)..controls(1330,539)and(1338,522)..(1375,526)..controls(1395,528)and(1400,533)..(1412,566)..controls(1432,624)and(1426,760)..(1401,821)..controls(1386,861)and(1380,866)..(1361,868)..controls(1348,870)and(1334,866)..(1329,861)--cycle;
      %%%% Arc inférieur gauche%%%%
      \path[fill](196,373)..controls(181,358)and(186,335)..(213,294)..controls(252,237)and(304,190)..(363,161)..controls(435,124)and(472,127)..(472,170)..controls(472,183)and(462,192)..(414,213)..controls(350,243)and(303,283)..(264,343)..controls(239,383)and(216,393)..(196,373)--cycle;
    \end{tikzpicture}
    \begin{tikzpicture}[remember picture,overlay]
      \node[anchor=north east,inner sep=0pt] at ($(current page.north east)+(0,-.8cm)$) {};
      \node[anchor=east, fill=white] at ($(current page.north east)+(-18.8,-2.3cm)$) {\footnotesize \bfseries{MathALEA}};
    \end{tikzpicture}
    \begin{tikzpicture}[line cap=round,line join=round,remember picture, overlay, shift={(current page.north west)},yshift=-8.5cm]
      \fill[fill=couleur_theme] (0,5) rectangle (21,6);
      \fill[fill=couleur_theme] (\xini,6)--(\xini+1.5,6)--(\xini+2.5,7)--(\xini+1.5,8)--(\xini,8)--(\xini+1,7)-- cycle;
      \fill[fill=couleur_theme] (\xini+2,6)--(\xini+2.5,6)--(\xini+3.5,7)--(\xini+2.5,8)--(\xini+2,8)--(\xini+3,7)-- cycle;  
      \fill[fill=couleur_theme] (\xini+3,6)--(\xini+3.5,6)--(\xini+4.5,7)--(\xini+3.5,8)--(\xini+3,8)--(\xini+4,7)-- cycle;   
      \node[color=white] at (10.5,5.5) {\LARGE \bfseries{ \MakeUppercase{#4}}};
    \end{tikzpicture}
    \begin{tikzpicture}[remember picture,overlay]
      \node[anchor=north east,inner sep=0pt] at ($(current page.north east)+(0,-.8cm)$) {};
      \node[anchor=east, fill=white] at ($(current page.north east)+(-2,-1.5cm)$) {\Huge \textcolor{couleur_theme}{\bfseries{\#}} \bfseries{#2} \textcolor{couleur_theme}{\bfseries \MakeUppercase{#3}}};
    \end{tikzpicture}
  }%
  \fancyfoot[R]{%
    \begin{tikzpicture}[remember picture,overlay]
      \node[anchor=south east] at ($(current page.south east)+(-2,0.25cm)$) {\scriptsize {\bfseries \href{https://coopmaths.fr/}{Coopmaths.fr} -- \href{http://creativecommons.fr/licences/}{CC-BY-SA}}};
    \end{tikzpicture}
    \begin{tikzpicture}[line cap=round,line join=round,remember picture, overlay,xscale=0.5,yscale=0.5, shift={(current page.south west)},xshift=35.7cm,yshift=-6cm]
      \fill[fill=couleur_theme] (\xini,6)--(\xini+1.5,6)--(\xini+2.5,7)--(\xini+1.5,8)--(\xini,8)--(\xini+1,7)-- cycle;
      \fill[fill=couleur_theme] (\xini+2,6)--(\xini+2.5,6)--(\xini+3.5,7)--(\xini+2.5,8)--(\xini+2,8)--(\xini+3,7)-- cycle;  
      \fill[fill=couleur_theme] (\xini+3,6)--(\xini+3.5,6)--(\xini+4.5,7)--(\xini+3.5,8)--(\xini+3,8)--(\xini+4,7)-- cycle;  
    \end{tikzpicture}
  }
  \fancyfoot[C]{}
  \colorlet{couleur_theme}{#1}
  \colorlet{couleur_numerotation}{couleur_theme}
  \def\iconeobjectif{icone-objectif-#1}
  \def\urliconeomethode{icone-methode-#1}
}%

%%%%%%% NOTATIONS DES ENSEMBLES %%%%%%%
\newcommand\R{\mathbb{R}}
\newcommand\N{\mathbb{N}}
\newcommand\D{\mathbb{D}}
\newcommand\Z{\mathbb{Z}}
\newcommand\Q{\mathbb{Q}}

\newcommand*\tikzfootMA{%
  \begin{tikzpicture}[remember picture,overlay,shift=(current page.south west)]
  \begin{scope}[x={(current page.south east)},y={(current page.north west)}]
    \draw[correction,line width=4pt,dashed,dash pattern= on 10pt off 10pt] ($(0,0)+(5mm,1.5cm)$) rectangle ($(1,1)+(-5mm,-3.75cm)$); 
  \end{scope}
\end{tikzpicture} 
}

\newenvironment{Correction}{%
  \setcounter{ExoMA}{0}%
  \cfoot{\tikzfootMA}
}{}%

\newcommand\version[1]{
  \fancyhead[R]{
    \begin{tikzpicture}[remember picture,overlay]
    \node[anchor=north east,inner sep=0pt] at ($(current page.north east)+(-.5,-.5cm)$) {\large \textcolor{couleur_theme}{\bfseries V#1}};
    \end{tikzpicture}
  }
}

%%%%% NOMBRES PREMIERS %%%%%
\usepackage{xlop} % JM pour les opérations
\opset{voperator=bottom}
%%% Table des nombres premiers  %%%%
\newcount\primeindex
\newcount\tryindex
\newif\ifprime
\newif\ifagain
\newcommand\getprime[1]{%
  \opcopy{2}{P0}%
  \opcopy{3}{P1}%
  \opcopy{5}{try}
  \primeindex=2
  \loop
  \ifnum\primeindex<#1\relax
  \testprimality
  \ifprime
  \opcopy{try}{P\the\primeindex}%
  \advance\primeindex by1
  \fi
  \opadd*{try}{2}{try}%
  \ifnum\primeindex<#1\relax
  \testprimality
  \ifprime
  \opcopy{try}{P\the\primeindex}%
  \advance\primeindex by1
  \fi
  \opadd*{try}{4}{try}%
  \fi
  \repeat
}
\newcommand\testprimality{%
  \begingroup
  \againtrue
  \global\primetrue
  \tryindex=0
  \loop
  \opidiv*{try}{P\the\tryindex}{q}{r}%
  \opcmp{r}{0}%
  \ifopeq \global\primefalse \againfalse \fi
  \opcmp{q}{P\the\tryindex}%
  \ifoplt \againfalse \fi
  \advance\tryindex by1
  \ifagain
  \repeat
  \endgroup
}

%%% Décomposition en nombres premiers %%%
\newcommand\primedecomp[2][nil]{%
  \begingroup
  \opset{#1}%
  \opcopy{#2}{NbtoDecompose}%
  \opabs{NbtoDecompose}{NbtoDecompose}%
  \opinteger{NbtoDecompose}{NbtoDecompose}%
  \opcmp{NbtoDecompose}{0}%
  \ifopeq
  Je refuse de décomposer zéro.
  \else
  \setbox1=\hbox{\opdisplay{operandstyle.1}%
    {NbtoDecompose}}%
  {\setbox2=\box2{}}%
  \count255=1
  \primeindex=0
  \loop
  \opcmp{NbtoDecompose}{1}\ifopneq
  \opidiv*{NbtoDecompose}{P\the\primeindex}{q}{r}%
  \opcmp{0}{r}\ifopeq
  \ifvoid2
  \setbox2=\hbox{%
    \opdisplay{intermediarystyle.\the\count255}%
    {P\the\primeindex}}%
  \else
  \setbox2=\vtop{%
    \hbox{\box2}
    \hbox{%
      \opdisplay{intermediarystyle.\the\count255}%
      {P\the\primeindex}}}
  \fi
  \opcopy{q}{NbtoDecompose}%
  \advance\count255 by1
  \setbox1=\vtop{%
    \hbox{\box1}
    \hbox{%
      \opdisplay{operandstyle.\the\count255}%
      {NbtoDecompose}}
  }%
  \else
  \advance\primeindex by1
  \fi
  \repeat
  \hbox{\box1
    \kern0.5\opcolumnwidth
    \opvline(0,0.75){\the\count255.25}
    \kern0.5\opcolumnwidth
    \box2}%
  \fi
  \endgroup
}

\usepackage{scratch3}
\usepackage{cancel}

\usepackage[tikz]{bclogo}

%%%%%%% PSTRICKS %%%%%%%
\usepackage{pstricks,pst-plot,pst-tree,pstricks-add}
\usepackage{pst-eucl}% permet de faire des dessins de géométrie simplement
\usepackage{pst-text}
\usepackage{pst-node,pst-all}
\usepackage{pst-func,pst-math,pst-bspline,pst-3dplot}  %%% POUR LE BAC %%%

\renewcommand{\pstEllipse}[5][]{% arc d'ellipse pour le sujet de Polynésie septembre 2013
  \psset{#1}
  \parametricplot{#4}{#5}{#2\space t cos mul #3\space t sin mul}
}

%%%%% TRACÉS DANS UN REPÈRE %%%%%
\newcommand{\vect}[1]{\overrightarrow{\,\mathstrut#1\,}}
\def\Oij{$\left(\text{O}~;~\vect{\imath},~\vect{\jmath}\right)$}
\def\Oijk{$\left(\text{O}~;~\vect{\imath},~\vect{\jmath},~\vect{k}\right)$}
\def\Ouv{$\left(\text{O}~;~\vect{u},~\vect{v}\right)$}

\newcommand{\e}{\mathrm{\,e\,}} %%% POUR LE BAC %%% le e de l'exponentielle
\newcommand{\ds}{\displaystyle} %%% POUR LE BAC %%%

%%%%% COMMANDES SPRECIFIQUES %%%%%
\usepackage{esvect} %%% POUR LE BAC %%%
\newcommand{\vvt}[1]{\vv{\text{#1}}} %%% POUR LE BAC %%%
\newcommand{\vectt}[1]{\overrightarrow{\,\mathstrut\text{#1}\,}} %%% POUR LE BAC %%%

\usepackage{multirow} % fusionner plusieurs lignes de tableau
\usepackage{diagbox} % des diagonales dans une cellule de tableau
\usepackage{forest} % arbre en proba

%%%%% PROBABILITÉS %%%%%
% Structure servant à avoir l'événement et la probabilité.
\def\getEvene#1/#2\endget{$#1$}
\def\getProba#1/#2\endget{$#2$}

\usepackage{pifont} % symboles
\newcommand{\textding}[1]{\text{\ding{#1}}}

\usepackage{booktabs} % tableaux de qualité

\usepackage[french]{babel}
\usepackage{eurosym}        

%%%%%%%%%%%%%%%%%%%%%%%%
%%% Fin du préambule %%%
%%%%%%%%%%%%%%%%%%%%%%%%