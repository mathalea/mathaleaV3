\textbf{EXERCICE -- B}

\medskip

\begin{tabular}{|l|}\hline
\textbf{Principaux domaines abordés}\\
-- Étude de fonction, fonction exponentielle\\
-- Équations différentielles\\ \hline
\end{tabular}

\bigskip

\textbf{Partie I}

\medskip

%Considérons l'équation différentielle

\[y'= -0,4y + 0,4\]

%où $y$ désigne une fonction de la variable $t$, définie et dérivable sur $[0~;~ + \infty[$.

\medskip

\begin{enumerate}
\item 
	\begin{enumerate}
		\item %Déterminer une solution particulière constante de cette équation différentielle.
$y = K$, avec $K \in \R$ est solution de l'équation, si, avec $y' = 0$,
		
$0 = -0,4K + 0,4 \iff 0,4K = 0,4 \iff K = 1$
		\item %En déduire l'ensemble des solutions de cette équation différentielle.
\starredbullet~On sait que les solutions de l'équation différentielle $y' = - 0,4y$ sont les fonctions définies par : $t \longmapsto y = C\text{e}^{-0,4t}$, avec $C \in \R$ ;
		
\starredbullet~Les solutions de l'équation $y' = - 0,4u + 0,4$ sont donc les fonctions :

\[t \longmapsto y = 1 + C\text{e}^{-0,4t},\, \text{avec }\,  C \in \R.\]

\item %Déterminer la fonction $g$, solution de cette équation différentielle, qui vérifie $g(0) = 10$.
$g$ définie par $g(t) = 1 + C\text{e}^{-0,4t}$ vérifie :

$g(0) = 10 \iff 1 + C\text{e}^{-0,4\times 0} = 10 \iff 1 + C = 10 \iff C = 9$.

On a donc $g(t) = 1 + 9\text{e}^{-0,4t}$
	\end{enumerate}
\end{enumerate}

\bigskip

\textbf{Partie II}

\medskip

%Soit $p$ la fonction définie et dérivable sur l'intervalle $[0~;~+ \infty[$ par 
%
%\[p(t) = \dfrac{1}{g(t)} = \dfrac{1}{1 + 9\e^{-0,4t}}.\]

\smallskip

\begin{enumerate}
\item %Déterminer la limite de $p$ en $+ \infty$.
On sait que $\displaystyle\lim_{t \to + \infty} \e^{-0,4t} = 0$, donc $\displaystyle\lim_{t \to + \infty} p(t) = 1$. 
\item %Montrer que $p'(t) = \dfrac{3,6\e^{-0,4t}}{ \left(1 + 9\e^{-0,4t}\right)^2}$ pour tout $t \in  [0~;~+ \infty[$.
$g$ somme de fonctions dérivables sur $\R$ est dérivable et sur cet intervalle :

$g'(t) = - 0,4 \times 9\e^{-0,4t} = - 3,6\e^{-0,4t}$.

Or $p(t) = \dfrac{1}{g(t)} \Rightarrow p'(t) = - \dfrac{g'(t)}{(g(t))^2} = - \dfrac{- 3,6\e^{-0,4t}}{\left(1 + 9\e^{-0,4t} \right)^2} = \dfrac{3,6\e^{-0,4t}}{\left(1 + 9\e^{-0,4t} \right)^2}$  pour tout $t \in  [0~;~+ \infty[$.
\item  
	\begin{enumerate}
		\item %Montrer que l'équation $p(t) = \dfrac{1}{2}$ admet une unique solution $\alpha$ sur $[0~;~+ \infty[$.
Le résultat précédent montre que, comme $3,6 > $,\, $\e^{-0,4t} > $ quel que soit le réel $y$, \, $\left(1 + 9\e^{-0,4t} \right)^2 > 0$, \, $p'(t) > 0$ sur $[0~;~+ \infty[$ : la fonction $p$ est strictement croissante sur cet intervalle.

Or $p(0) = \dfrac{1}{1 + 9} = \dfrac{1}{10} = 0,1$ ,  et $\displaystyle\lim_{t \to + \infty} p(t) = \dfrac{1}{1} = 1$.

Par application du théorème des valeurs intermédiaires comme $\dfrac{1}{2} \in [0~;~1]$, il existe un réel unique $\alpha \in [0~;~+ \infty[$ tel que $p(\alpha) = \dfrac{1}{2}$.
		\item %Déterminer une valeur approchée de $\alpha$ à $10^{-1}$ près à l'aide d'une calculatrice.
La calculatrice donne :

$p(5) \approx 0,45$ et $p(6) \approx 0,55$, donc $5 < \alpha < 6$ ;

$p(5,4) \approx 0,491$ et $p(5,5) \approx 0,501$, donc $5,4 < \alpha < 5,5$ ;

$p(5,49) \approx 0,499$ et $p(5,50) \approx 0,501$, donc $5,49 < \alpha < 5,50$.

Conclusion $\alpha \approx 5,5$ à $10^{-1}$ près.
	\end{enumerate}
\end{enumerate}

\bigskip

\textbf{Partie III}

\medskip

\begin{enumerate}
\item %$p$ désigne la fonction de la partie II.

%Vérifier que $p$ est solution de l'équation différentielle $y' = 0,4y(1 - y)$ avec la condition initiale 
%$y(0) = \dfrac{1}{10}$ où $y$ désigne une fonction définie et dérivable sur $[0~;~ + \infty[$.

\starredbullet~$p'(t) = \dfrac{3,6\e^{-0,4t}}{\left(1 + 9\e^{-0,4t} \right)^2}$ d'après la question 2.

\starredbullet~$0,4p(t)(1 - p(t)) = 0,4 \times \dfrac{1}{1 + 9\text{e}^{-0,4t}} \times \left(1 - \dfrac{1}{1 + 9\text{e}^{-0,4t}} \right)^2 = 0,4\times \dfrac{9\text{e}^{-0,4t}}{1 + 9\text{e}^{-0,4t}} = \dfrac{3,6\e^{-0,4t}}{\left(1 + 9\e^{-0,4t} \right)^2}$, donc $p$ est solution de l'équation différentielle.

De plus on a vu que $p(0) = \dfrac{1}{10}$.
\item %Dans un pays en voie de développement, en l'année 2020, 10\,\% des écoles ont accès à internet. 

%Une politique volontariste d'équipement est mise en œuvre et on s'intéresse à l'évolution de la proportion des écoles ayant accès à internet. 
%
%On note $t$ le temps écoulé, exprimé en année, depuis l'année 2020.
%
%La proportion des écoles ayant accès à internet à l'instant $t$ est modélisée par $p(t)$.
%
%Interpréter dans ce contexte la limite de la question II 1 puis la valeur approchée de $\alpha$ de la question II 3. b. ainsi que la valeur $p(0)$.
$\bullet~~$$\displaystyle\lim_{t \to + \infty} p(t) = 1$ signifie qu'à long terme toutes les écoles auront accès à internet.

$\bullet~~$$\alpha \approx 5,5$ signifie qu'au bout de 5 ans et demi la moitié des écoles aura accès à internet.

$\bullet~~$$p(0) = 0,1$ signifie qu'en 2020, 10\,\% des écoles ont accès à internet.
\end{enumerate}

%%%%%%%%%%%%%%% sujet du 8 juin %%%%%%%%%
