
\smallskip

\begin{tabular}{|l|}\hline
Principaux domaines abordés :\\
Fonction exponentielle; dérivation; convexité\\ \hline
\end{tabular}

\begin{center}\textbf{Partie 1}\end{center}

%On donne ci-dessous, dans le plan rapporté à un repère orthonormé, la courbe représentant la
%fonction dérivée $f'$ d'une fonction $f$ dérivable sur $\R$.
%
%À l'aide de cette courbe, conjecturer, en justifiant les réponses : 
%
%\medskip

\begin{enumerate}
\item% Le sens de variation de la fonction $f$ sur $\R$.
\begin{list}{\textbullet}{D'après la courbe représentant la fonction dérivée $f'$:}
\item la fonction $f'$ est positive sur $]-\infty~;~1[$ donc la fonction $f$ est croissante sur cet intervalle;
\item la fonction $f'$ est négative sur $]1~;~+\infty[$ donc la fonction $f$ est décroissante sur cet intervalle.
\end{list}

\item %La convexité de la fonction $f$ sur $\R$.
\begin{list}{\textbullet}{D'après la courbe représentant la fonction dérivée $f'$:}
\item la fonction $f'$ est décroissante sur $]-\infty~;~0[$ donc la fonction $f$ est concave sur cet intervalle; 
\item la fonction $f'$ est croissante sur $]0~;~+\infty[$ donc la fonction $f$ est convexe sur cet intervalle.
\end{list}
\end{enumerate}

%\begin{center}
%\psset{unit=1cm,arrowsize=2pt 3}
%\begin{pspicture*}(-3.25,-2)(5.25,4.5)
%\psgrid[gridlabels=0pt,subgriddiv=1,gridwidth=0.15pt](-2,-1)(4,4.5)
%\psaxes[linewidth=1.25pt](0,0)(-2,-1.25)(4,4.5)
%\psaxes[linewidth=1.25pt]{->}(0,0)(1,1)
%\psplot[plotpoints=2000,linewidth=1.25pt,linecolor=red]{-2}{4}{1 x add 2.71828 x exp div  neg}
%\rput(1,-1.5){Courbe représentant la \textbf{dérivée} $f'$ de la fonction $f$.}
%\end{pspicture*}
%\end{center}

\medskip

\begin{center}\textbf{Partie 2}\end{center}

\smallskip

On admet que la fonction $f$ mentionnée dans la Partie 1 est définie sur $\R$ par :
$f(x) = (x + 2)\e^{-x}.$
%
%On note $\mathcal{C}$ la courbe représentative de $f$ dans un repère orthonormé \Oij.
%
%On admet que la fonction $f$ est deux fois dérivable sur $\R$, et on note $f'$ et $f''$ les fonctions dérivées première et seconde de $f$ respectivement.

\medskip

\begin{enumerate}
\item Pour tout nombre réel $x$, $f(x) = (x + 2)\e^{-x}=x\e^{-x}+2\e^{-x}
= \dfrac{x}{\e^{x}}+ 2\e^{-x}$.

%En déduire la limite de $f$ en $+ \infty$.

D'après le cours: $\displaystyle\lim_{x \to ++ \infty} \dfrac{\e^{x}}{x}= +\infty$ donc  $\displaystyle\lim_{x \to + \infty} \dfrac{x}{\e^{x}}= 0$.

De plus $\displaystyle\lim_{x \to + \infty} \e^{-x}= 0$ donc $\displaystyle\lim_{x \to + \infty} f(x) = 0$.

%Justifier que la courbe $\mathcal{C}$ admet une asymptote que l'on précisera.

On en déduit que la courbe $\mathcal{C}$ admet la droite d'équation $y=0$, c'est-à-dire l'axe des abscisses, comme asymptote horizontale en $+\infty$.

On admet que $\displaystyle\lim_{x \to - \infty} f(x) = - \infty$.
\item 
	\begin{enumerate}
		\item% Montrer que, pour tout nombre réel $x$,\, $f'(x) = (- x - 1)\e^{-x}$.
$f'(x)=1\times \e^{-x} + (x+2)\times (-1)\e^{-x} = (1-x-2) \e^{-x}=(-x-1)\e^{-x}$. 		
		
		\item% Étudier les variations sur $\R$ de la fonction $f$ et dresser son tableau de variations.
Pour tout $x$, $\e^{-x}>0$ donc $f'(x)$ est du signe de $-x-1$; donc $f'(x)$ s'annule et change de signe en $x=-1$.

$f(-1) = (-1+2)\e^{1}=\e$; on établit le tableau de variations de $f$ sur $\R$:

\begin{center}
{\renewcommand{\arraystretch}{1.5}
\psset{nodesep=3pt,arrowsize=2pt 3}%  paramètres
\def\esp{\hspace*{2.5cm}}% pour modifier la largeur du tableau
\def\hauteur{0pt}% mettre au moins 20pt pour augmenter la hauteur
$\begin{array}{|c|l*4{c}|}
\hline
x & -\infty  & \esp & -1 & \esp & +\infty \\ 
\hline
-x-1 &    &   \pmb{+} & \vline\hspace{-2.7pt}0 & \pmb{-} & \\ 
\hline
f'(x) &    &   \pmb{+} & \vline\hspace{-2.7pt}0 & \pmb{-} & \\ 
\hline
 &  &  &   \Rnode{max}{\e}  &  &  \rule{0pt}{\hauteur} \\  
f(x) &   &     &  &  &  \rule{0pt}{\hauteur} \\ 
 &  \Rnode{min1}{-\infty} &   &  &  &   \Rnode{min2}{0} \rule{0pt}{\hauteur}    
 \ncline{->}{min1}{max} 
 \ncline{->}{max}{min2} 
 \\ 
\hline
\end{array} $
}
\end{center}	
	
		\item% Montrer que l'équation $f(x) = 2$ admet une unique solution $\alpha$ sur l'intervalle $[-2~;~-1]$ dont on donnera une valeur approchée à $10^{-1}$ près.
Sur l'intervalle $[-2~;~-1]$, la fonction $f$ est strictement croissante et continue car dérivable sur cetintervalle. $f(-2)=0<2$ et $f(-1)=\e>2$	donc, d'après le corollaire du théorème des valeurs intermédiaires, l'équation $f(x)=2$ admet une solution unique sur l'intervalle $[-2~;~-1]$.
		
		
	\end{enumerate}
\item% Déterminer, pour tout nombre réel $x$, l'expression de $f''(x)$ et étudier la convexité de la fonction~$f$. 
$f''(x)=(-1)\times \e^{-x} + (-x-1)\times (-1)\e^{-x} = (-1+x+1) \e^{-x}=x\e^{-x}$

$\e^{-x}>0$ pour tout $x$, donc $f''(x)$ est du signe de $x$.

\begin{list}{\textbullet}{}
\item Sur $]-\infty~;~0[$, $f''(x)<0$ donc la fonction $f$ est concave.
\item Sur $]0~;~+\infty[$, $f''(x)>0$ donc la fonction $f$ est convexe.
\item En $x=0$, la dérivée seconde s'annule et change de signe donc le point A d'abscisse 0 de $\mathcal{C}$ est le point d'inflexion de cette courbe.
\end{list}


%Que représente pour la courbe $\mathcal{C}$ son point A d'abscisse $0$ ?
\end{enumerate}
