
\textbf{Commun à tous les candidats}

\medskip

\begin{center}\textbf{Partie I}\end{center}

On considère la fonction $f$ définie sur $\R$ par 

\[f(x) = x - \text{e}^{-2x}.\]

On appelle $\Gamma$ la courbe représentative de la fonction $f$ dans un repère orthonormé \Oij.

\medskip

\begin{enumerate}
\item ~%Déterminer les limites de la fonction $f$ en $- \infty$ et en $+ \infty$.
$\bullet~~$On a $\displaystyle\lim_{x \to - \infty} x = - \infty$ et $\displaystyle\lim_{x \to - \infty}\text{e}^{-2x} = + \infty$ et donc $\displaystyle\lim_{x \to - \infty}-\text{e}^{-2x} = - \infty$ ; donc par somme de limites :

$\displaystyle\lim_{x \to - \infty}f(x) = - \infty$.

$\bullet~~$On a $\displaystyle\lim_{x \to + \infty} x  = + \infty$ et $\displaystyle\lim_{x \to + \infty}\text{e}^{-2x} = 0$ ; donc par somme de limites  $\displaystyle\lim_{x \to - \infty}f(x) = + \infty$.
\item %Étudier le sens de variation de la fonction $f$ sur $\R$ et dresser son tableau de variation.
la fonction $f$ est dérivable comme somme de fonctions dérivables sur $\R$ et sur cet intervalle :

$f'(x) = 1 - (- 2)\text{e}^{-2x} = 1 + 2\text{e}^{-2x}$.

On sait que quel que soit $x \in \R$, \, $\text{e}^{-2x} > 0$, donc $ 1 + 2\text{e}^{-2x} > 1 > 0$.
La dérivée est positive donc la fonction $f$ est strictement croissante de moins l'infini à plus l'infini.
\item %Montrer que l'équation $f(x) = 0$ admet une unique solution $\alpha$ sur $\R$, dont on donnera une valeur approchée à $10^{-2}$ près.
La fonction $f$ est continue car dérivable et strictement croissante; comme $0 \in \R$, d'après le corollaire du théorème des valeurs intermédiaires, il existe un réel unique $\alpha \in \R$ telle que $f(\alpha) = 0$.

La calculatrice donne  :

$f(0) = - 1$ et $f(1) \approx 0,865$, donc $0 < \alpha < 1$ ;

$f(0,4) \approx - 0,05$ et $f(0,5) \approx 0,13$, donc $0,4 < \alpha < 0,5$ ;

$f(0,42) \approx - 0,01$ et $f(0,43) \approx 0,007$, donc $0,42 < \alpha < 0,43$.
\item %Déduire des questions précédentes le signe de $f(x)$ suivant les valeurs de $x$.
On a donc :

$\bullet~~$sur $]- \infty~;~\alpha[, \, f(x) < 0$ ;

$\bullet~~$sur $]\alpha~;~+ \infty[, \, f(x) > 0$ ;

$\bullet~~$et $f(\alpha) = 0$.
\end{enumerate}

\begin{center}\textbf{Partie II}\end{center}


%Dans le repère orthonormé \Oij, on appelle $\mathcal{C}$ la courbe représentative de la fonction $g$ définie sur $\R$ par:
%
%\[g(x) = \text{e}^{-x}.\]
%
%La courbes $\mathcal{C}$ et la courbe $\Gamma$ (qui représente la fonction $f$ de la Partie I) sont tracées sur le \textbf{graphique donné en annexe qui est à compléter et à rendre avec la copie.}

%\smallskip
%
%Le but de cette partie est de déterminer le point de la courbe $\mathcal{C}$ le plus proche de l'origine O du repère et d'étudier la tangente à $\mathcal{C}$ en ce point.
%
%\medskip

\begin{enumerate}
\item %Pour tout nombre réel $t$, on note $M$ le point de coordonnées $\left(t~;~\text{e}^{-t}\right)$ de la courbe $\mathcal{C}$.

%On considère la fonction $h$ qui, au nombre réel $t$, associe la distance O$M$.

%On a donc: $h(t) = \text{O}M$, c'est-à-dire :

\[h(t) = \sqrt{t^2 + \text{e}^{-2t}}\]

	\begin{enumerate}
		\item %Montrer que, pour tout nombre réel $t$,
Soit $u(t) = t^2 + \text{e}^{-2t}$, donc $h(t) = \sqrt{u(t)}$ fonction dérivable car composée de deux fonctions \og racine \fg{}  et $h$ dérivables.

Donc $h'(t) = \dfrac{u'(t)}{2\sqrt{u(t)}} = \dfrac{2t - 2\text{e}^{-2t}}{2\sqrt{t^2 + \text{e}^{-2t}}} = \dfrac{t - \text{e}^{-2t}}{\sqrt{t^2 + \text{e}^{-2t}}} = \dfrac{f(t)}{\sqrt{t^2 + \text{e}^{-2t}}}$.
%\[h'(t) = \dfrac{f(t)}{\sqrt{t^2 + \text{e}^{-2t}}}.\]

%où $f$ désigne la fonction étudiée dans la \textbf{Partie I}.
		\item %Démontrer que le point A de coordonnées $\left(\alpha~;~\text{e}^{-\alpha}\right)$ est le point de la courbe $\mathcal{C}$ pour lequel la longueur O$M$ est minimale.
		
Le dénominateur étant positif, le signe de $h'(t)$ est celui du numérateur soit $f(t)$ dont on a vu le signe dans la partie I.

Donc :

$\bullet~~$sur $]- \infty~;~\alpha[, \, f(t) < 0$ donc $h'(t) < 0$ : la fonction est strictement décroissante sur cet intervalle ;

$\bullet~~$sur $]\alpha~;~+ \infty[, \, f(x) > 0$ donc $h'(t) > 0$ : la fonction est strictement croissante sur cet intervalle ;

$\bullet~~$et $f(\alpha) = 0$, donc $h(\alpha)$ est le minimum de la fonction $h$.

La distance O$M$ est donc minimale pour $t = \alpha$ et l'ordonnée de $M$ est alors $\text{e}^{- \alpha}$.

Le point de la courbe le plus proche de l'origine est donc le point A$\left(\alpha~;~\text{e}^{-\alpha}\right)$.

%Placer ce point sur le \textbf{graphique donné en annexe, à rendre avec la copie}.
$\alpha$ est l'abscisse du point d'intersection de $\Gamma$ avec l'axe des abscisses. Il suffit de tracer la parallèle à l'axe des ordonnées passant par ce point, elle coupe $\mathcal{C}$ au point A
	\end{enumerate}
\item %On appelle $T$ la tangente en A à la courbe $\mathcal{C}$.
	\begin{enumerate}
		\item %Exprimer en fonction de $\alpha$ le coefficient directeur de la tangente $T$.
Le coefficient directeur de la tangente $T$ au point d'abscisse $\alpha$ est $g'(\alpha) = - \text{e}^{- \alpha}$		
%On rappelle que le coefficient directeur de la droite (OA) est égal à $\dfrac{\text{e}^{-\alpha}}{\alpha}$.
%
%On rappelle également le résultat suivant qui pourra être utilisé sans démonstration:
%
%\emph{Dans un repère orthonormé du plan, deux droites $D$ et $D'$ de coefficients directeurs respectifs $m$ et $m'$ sont perpendiculaires si, et seulement si le produit $mm'$ est égal à $-1$.}

		\item %Démontrer que la droite (OA) et la tangente $T$ sont perpendiculaires. 
D'après le rappel le produit des coefficients directeurs est $- \text{e }^{-\alpha} \times \dfrac{\text{e}^{-\alpha}}{\alpha} = -\dfrac{\text{e }^{-2\alpha}}{\alpha}$.

or on sait que $f(\alpha) = 0 \iff \alpha - \text{e}^{-2\alpha} = 0 \iff \alpha = \text{e}^{-2\alpha} \iff \dfrac{\text{e }^{-2\alpha}}{\alpha} = 1$, donc finalement le produit des coefficients directeurs est égal à $- 1$. La droite (OA) et la tangente $T$ sont perpendiculaires.

Voir à la fin.
%Tracer ces droites sur le \textbf{graphique donné en annexe, à rendre avec la copie.}
	\end{enumerate}
\end{enumerate}

\begin{center}

	\textbf{\large Annexe à compléter et à rendre avec la copie}
	
	\psset{unit=3.25cm,comma=true}
	\begin{pspicture*}(-1,-0.75)(3.25,2.6)
	\psgrid[gridlabels=0pt,subgriddiv=2,gridwidth=0.08pt](-1,-1)(4,3)
	\psaxes[linewidth=1.25pt,Dx=0.5,Dy=0.5]{->}(0,0)(-0.99,-0.75)(3.25,2.6)
	\uput[l](1.5,1.5){\red $\Gamma$}\uput[ur](-0.75,2){\blue $\mathcal{C}$}
	\psplot[plotpoints=2000,linecolor=red,linewidth=1.25pt]{-1}{3.25}{x 1 2.71828 2 x mul exp div sub}
	\psplot[plotpoints=2000,linecolor=blue,linewidth=1.25pt]{-1}{3.25}{1 2.71828  x  exp div}
	\psline[linestyle=dashed,linewidth=1.25pt](0.43,0)(0.43,0.65)(0,0.65)
	\uput[ur](0.43,0.65){\blue A}\uput[ul](0.43,0){\red $\alpha$}
	\psplotTangent[linecolor=blue]{0.43}{4}{2.71828 x neg exp}
	\psplot[plotpoints=200,linewidth=1.25pt]{0}{0.43}{x 1.5128 mul}
	\uput[dl](0,0){O}
	\end{pspicture*}
	\end{center}

	
