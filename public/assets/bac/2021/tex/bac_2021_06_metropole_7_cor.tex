\textbf{\large\textsc{Exercice 2} \hfill 6 points}

\textbf{Commun à tous les candidats}

\medskip

Dans cet exercice, les résultats des probabilités demandées seront, si nécessaire, arrondis au millième.

%La leucose féline est une maladie touchant les chats ; elle est provoquée par un virus.
%
%Dans un grand centre vétérinaire, on estime à 40\,\% la proportion de chats porteurs de la maladie.
%
%On réalise un test de dépistage de la maladie parmi les chats présents dans ce centre vétérinaire.
%
%Ce test possède les caractéristiques suivantes.
%
%\setlength\parindent{9mm}
%\begin{itemize}
%\item[$\bullet~~$] Lorsque le chat est porteur de la maladie, son test est positif dans 90\,\% des cas.
%\item[$\bullet~~$] Lorsque le chat n'est pas porteur de la maladie, son test est négatif dans 85\,\% des cas.
%\end{itemize}
%\setlength\parindent{0mm}
%
%On choisit un chat au hasard dans le centre vétérinaire et on considère les évènements suivants :
%
%\setlength\parindent{9mm}
%\begin{itemize}
%\item[$\bullet~~$] $M$ : \og Le chat est porteur de la maladie\fg{} ;
%\item[$\bullet~~$] $T$ : \og Le test du chat est positif\fg ;
%\item[$\bullet~~$] $\overline{M}$ et $\overline{T}$ désignent les évènements contraires des évènements $M$ et $T$ respectivement.
%\end{itemize}
%\setlength\parindent{0mm}

\medskip

\begin{enumerate}
\item 
	\begin{enumerate}
		\item On traduit la situation par un arbre pondéré.
		
\begin{center}
\pstree[treemode=R,nodesepB=3pt,levelsep=2.75cm]{\TR{}}
{\pstree{\TR{$M$~~}\taput{0,4}}
	{\TR{$T$}\taput{0,9}
	\TR{$\overline{T}$}\tbput{0,1}
	}
\pstree{\TR{$\overline{M}$~~}\tbput{0,6}}
	{\TR{$T$}\taput{0,15}
	\TR{$\overline{T}$}\tbput{0,85}
	}	 
}		
\end{center}

\medskip
	
		\item %Calculer la probabilité que le chat soit porteur de la maladie et que son test soit positif.
Il faut trouver $P(M \cap T) = P(M) \times P_M(T) = 0,4 \times 0,9 = 0,36$.
		\item %Montrer que la probabilité que le test du chat soit positif est égale à $0,45$.
On a de même $P\left(\overline{M} \cap T\right) = P\left(\overline{M}\right) \times P_{\overline{M}}(T) = 0,6 \times  0,15 = 0,09$.

D'après la loi des probabilités totales :

$P(T) = P(M \cap T) + P\left(\overline{M} \cap T\right) = 0,36 + 0,09 = 0,45$.
		\item %On choisit un chat parmi ceux dont le test est positif. Calculer la probabilité qu'il soit porteur de la maladie.
		Il faut trouver $P_T(M) = \dfrac{P(M \cap T)}{P(T)} = \dfrac{0,36}{0,45} = \dfrac{36}{45} = \dfrac{9 \times 4}{9 \times 5} = \dfrac{4}{5} = \dfrac{8}{10} = 0,8$.
	\end{enumerate}
\item %On choisit dans le centre vétérinaire un échantillon de $20$ chats au hasard. On admet que l'on peut assimiler ce choix à un tirage avec remise.

%On note $X$ la variable aléatoire donnant le nombre de chats présentant un test positif dans l'échantillon choisi.
	\begin{enumerate}
		\item %Déterminer, en justifiant, la loi suivie par la variable aléatoire $X$.
On suppose que le nombre de chats est assez important pour que l'on puisse assimiler le choix des 20 chats à un tirage avec remise.
		
La variable $X$ suit donc une loi binomiale de paramètres $n = 20$ et de probabilité $p = 0,45$ trouvé à la question \textbf{1. c.}.
		\item %Calculer la probabilité qu'il y ait dans l'échantillon exactement $5$ chats présentant un test positif.
On a $p\left(X = 5\right) = \binom{20}{5} \times 0,45^5 \times (1 - 0,45)^{20-5} = \np{15504}\times 0,45^5  \times 0,55^{15} \approx \np{0,0365}$ soit environ 0,037.

		\item %Calculer la probabilité qu'il y ait dans l'échantillon au plus $8$ chats présentant un test positif.
La calculatrice donne $P(X < 9) \approx 0,414$.
		\item %Déterminer l'espérance de la variable aléatoire $X$ et interpréter le résultat dans le contexte de l'exercice.
On sait que l'espérance $E = n \times p = 20 \times 0,45 = 9$.
		
Cela signifie que sur un grand nombre d'échantillons il y aura en moyenne 9 chats positifs par échantillon de 20.
	\end{enumerate}		
\item %Dans cette question, on choisit un échantillon de $n$ chats dans le centre, qu'on assimile encore à un tirage avec remise. On note $p_n$ la probabilité qu'il y ait au moins un chat présentant un test positif dans cet échantillon.
	\begin{enumerate}
		\item %Montrer que $p_n = 1 - 0,55^n$.
On a encore une loi binomiale de paramètres $n$ et de probabilité d'être positif de $0,45$.

On a $P(X = 0) = \binom{n}{0}\times 0,45^0 \times 0,55^n = 0,55^n$.

Donc $p_n = 1 - P(X = 0) =  1 - 0,55^n$.
		\item
%\begin{minipage}{6cm}Décrire le rôle du programme ci-contre écrit en langage Python, dans lequel la variable $n$ est un entier naturel et la variable $P$ un nombre réel.
%\end{minipage} \hfill \begin{minipage}{6cm}
%\begin{tabular}{|l|}\hline
%def seuil():\\
%\qquad $n=0$\\
%\qquad $P=0$\\
%\qquad while $P< 0,99$ :\\
%\qquad \quad $n = n+1$\\
%\qquad \quad $P=1 - 0,55**n$\\
%\qquad return $n$\\ \hline
%\end{tabular}
%\end{minipage}
En partant de $n=0$, le programme calcule $p_n$ et augmente la taille de l'échantillon de 1 tant que $p_n < 0,99$.

		\item %Déterminer, en précisant la méthode employée, la valeur renvoyée par ce programme.
		On cherche donc $n$ tel que $1 - 0,55^n \geqslant 0,99 \iff 0,01 \geqslant 0,55^n$, d'où par croissance de la fonction logarithme népérien : $\ln 0,01 \geqslant n \ln 0,55 \iff \dfrac{\ln 0,01}{\ln 0,55} \leqslant n$ (car $\ln 001 < 0$).
		
		Or $\dfrac{\ln 0,01}{\ln 0,55} \approx 7,7$.
		
Conclusion : le programme renvoie la valeur 8.
	\end{enumerate}
\end{enumerate}

\bigskip

