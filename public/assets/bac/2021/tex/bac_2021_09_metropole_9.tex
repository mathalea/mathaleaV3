
\medskip

\begin{tabular}{|l|}\hline
Principaux domaines abordés :\\
Suites numériques ; raisonnement par récurrence.\\ \hline
\end{tabular}

\medskip

On considère les suites $\left(u_n\right)$ et $\left(v_n\right)$ définies par:

\[u_0 = 16 \quad ;\quad  v_0 = 5 \;;\]

et pour tout entier naturel $n$ :

\renewcommand\arraystretch{2}
\[\left\{\begin{array}{l c l}
u_{n+1}&=&\dfrac{3u_n + 2v_n}{5}\\
v_{n+1}&=& \dfrac{u_n + v_n}{2}
\end{array}\right.\]
\renewcommand\arraystretch{1}

\smallskip

\begin{enumerate}
\item Calculer $u_1$ et $v_1$.
\item On considère la suite $\left(w_n\right)$ définie pour tout entier naturel $n$ par : $w_n = u_n - v_n$.
	\begin{enumerate}
		\item Démontrer que la suite $\left(w_n\right)$ est géométrique de raison 0,1.
		
En déduire, pour tout entier naturel $n$, l'expression de $w_n$ en fonction de $n$.
		\item Préciser le signe de la suite $\left(w_n\right)$  et la limite de cette suite.
	\end{enumerate}
\item 
	\begin{enumerate}
		\item Démontrer que, pour tout entier naturel $n$, on a : $u_{n+1} - u_n = - 0,4 w_n$.
		\item En déduire que la suite $\left(u_n\right)$ est décroissante.

\end{enumerate}
		
On peut démontrer de la même manière que la suite $\left(v_n\right)$ est croissante. On admet ce résultat, et on remarque qu'on a alors: pour tout entier naturel $n$,\, $v_n \geqslant v_0 = 5$.
\begin{enumerate}[resume]
		\item Démontrer par récurrence que, pour tout entier naturel $n$, on a : $u_n \geqslant 5$. 
		
En déduire que la suite $\left(u_n\right)$ est convergente. On appelle $\ell$ la limite de $\left(u_n\right)$.
	\end{enumerate}
\end{enumerate}
	
On peut démontrer de la même manière que la suite $\left(v_n\right)$ est convergente. On admet ce résultat, et on appelle $\ell'$ la limite de $\left(v_n\right)$.
\begin{enumerate}[resume]
\item 
	\begin{enumerate}
		\item Démontrer que $\ell = \ell'$.
		\item On considère la suite $\left(c_n\right)$ définie pour tout entier naturel $n$ par : $c_n = 5u_n + 4v_n$.
		
Démontrer que la suite $\left(c_n\right)$ est constante, c'est-à-dire que pour tout entier naturel $n$, on a : $c_{n+1} = c_n$. 

En déduire que, pour tout entier naturel $n$ ,\, $c_n = 100$.
		\item Déterminer la valeur commune des limites  $\ell$ et $\ell'$.
	\end{enumerate}
\end{enumerate}

\bigskip

