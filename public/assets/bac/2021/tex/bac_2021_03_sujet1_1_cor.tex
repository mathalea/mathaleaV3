\textbf{\large Exercice 1 \hfill Commun à tous les candidats \hfill 5 points}

\medskip

Dans une école de statistique, après étude des dossiers des candidats, le recrutement se fait de deux façons :

\setlength\parindent{1cm}
\begin{itemize}
\item[$\bullet~~$] 10\,\% des candidats sont sélectionnés sur dossier. Ces candidats doivent ensuite passer un oral à l'issue duquel 60\,\% d'entre eux sont finalement admis à l'école.
\item[$\bullet~~$] Les candidats n'ayant pas été sélectionnés sur dossier passent une épreuve écrite à l'issue de laquelle 20\,\% d'entre eux sont admis à l'école.
\end{itemize}
\setlength\parindent{0cm}

%\bigskip

\begin{center}
\textbf{Partie I}
\end{center}

On choisit au hasard un candidat à ce concours de recrutement. On notera:

\setlength\parindent{1cm}
\begin{itemize}
\item[$\bullet~~$] $D$ l'évènement \og le candidat a été sélectionné sur dossier \fg{} ;
\item[$\bullet~~$] $A$ l'évènement \og le candidat a été admis à l'école \fg{} ;
\item[$\bullet~~$] $\overline{D}$ et $\overline{A}$ les évènements contraires des évènement $D$ et $A$ respectivement.
\end{itemize}
\setlength\parindent{0cm}

\medskip

\begin{enumerate}
\item On traduit la situation par un arbre pondéré:

\begin{center}
{%\bigskip
\psset{levelsep=3cm,nodesepB=4pt, treesep=10mm}
\pstree[treemode=R,nodesepA=0pt]% R pour Right
       {\TR{}}
       {
       \pstree[nodesepA=4pt]{\TR{$D$}\naput{$0,1$}}
	                        {
	                        \TR{$A$}\naput{$0,6$}
			                \TR{$\overline{A}$}\nbput{$0,4$}
	                        }
       \pstree[nodesepA=4pt]{\TR{$\overline{D}$}\nbput{$0,9$}}
	                        {
	                        \TR{$A$}\naput{$0,2$}
			                \TR{$\overline{A}$}\nbput{$0,8$}
	                        }
      }
}
\bigskip
\end{center}

\item La probabilité que le candidat soit sélectionné sur dossier et admis à l'école est:

$P(D\cap A)=0,1\times 0,6=0,06$.

\item La probabilité de l'évènement  $A$ est $P(A)$.

D'après la formule des probabilités totales:

$P(A) = P(D\cap A) + P(\overline{D}\cap A) = 0,06 + 0,9\times 0,2 = 0,24$.

\item On choisit au hasard un candidat admis à l'école. La probabilité que son dossier n'ait pas été sélectionné est:

$P_A(\overline{D}) = \dfrac{P(\overline{D}\cap A)}{P(A)} = \dfrac{0,18}{0,24}=0,75$.
\end{enumerate}

%\bigskip

\begin{center}
\textbf{Partie II}
\end{center}

\begin{enumerate}
\item On admet que la probabilité pour un candidat d'être admis à l'école est égale à $0,24$.

On considère un échantillon de sept candidats choisis au hasard, en assimilant ce choix à un tirage au sort avec remise. On désigne par $X$ la variable aléatoire dénombrant les candidats admis à l'école parmi les sept tirés au sort.
	\begin{enumerate}
		\item On admet que la variable aléatoire $X$ suit une loi binomiale. %Quels sont les paramètres de cette loi?
		
La probabilité pour un candidat d'être admis à l'école est égale à $p=0,24$, et on choisit un échantillon de 7 candidats donc $n=7$.

La variable aléatoire $X$ suit donc la loi binomiale $\mathcal{B}\left (7\,;\,0,24\right )$.
		
		\item La probabilité qu'un seul des sept candidats tirés au sort soit admis à l'école est:
		
$P(X=1)= \ds\binom{7}{1}\times 0,24^1 \times (1-0,24)^{7-1} \approx 0,32$.		
		
		\item La probabilité qu'au moins deux des sept candidats tirés au sort soient admis à cette école est: $P(X\geqslant 2) = 1-P(X\leqslant 1) = 1-0,47=0,53$.		
		
	\end{enumerate}
	
\item  Un lycée présente $n$ candidats au recrutement dans cette école, où $n$ est un entier naturel non nul.

On admet que la probabilité pour un candidat quelconque du lycée d'être admis à l'école est égale à $0,24$ et que les résultats des candidats sont indépendants les uns des autres.

	\begin{enumerate}
		\item La variable aléatoire $Y$ qui donne le nombre d'admis parmi les $n$ candidats présentés suit la loi binomiale $\mathcal{B}\left (n\;;\;0,24\right )$.
		
		 La probabilité qu'aucun candidat issu de ce lycée ne soit admis à l'école est:
		
$P(Y=0) = \ds\binom{n}{0}\times 0,24^0 \times 0,76^n=0,76^n$.		
		
		\item On cherche à partir de quelle valeur de l'entier $n$ la probabilité qu'au moins un élève de ce lycée soit admis à l'école est supérieure ou égale à $0,99$.
		
On veut donc que $P(Y\geqslant 1) \geqslant 0,99$ c'est-à-dire $1-P(Y=0) \geqslant 0,99$ ou encore\\ $P(Y=0) \leqslant 0,01$. On résout l'inéquation d'inconnue $n$: $0,76^n \leqslant 0,01$:		
		
$0,76^n \leqslant 0,01 \iff \ln\left (0,76^n\right ) \leqslant \ln\left (0,01\right )
\iff n \times \ln\left (0,76\right ) \leqslant \ln\left (0,01\right )
\iff n \geqslant \dfrac{\ln\left (0,01\right )}{\ln\left (0,76\right )}$

Or $\dfrac{\ln\left (0,01\right )}{\ln\left (0,76\right )}\approx 16,8$ donc c'est à partir de 17 élèves que la probabilité qu'au moins un élève de ce lycée soit admis à l'école est supérieure ou égale à $0,99$.
\end{enumerate}
\end{enumerate}

\bigskip

