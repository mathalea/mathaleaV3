
\medskip

La population d'une ville A augmente chaque année de 2\,\%. 

La ville A avait \np{4600} habitants en 2010.

La population d'une ville B augmente de 110 habitants par année.

La ville B avait \np{5100} habitants en 2010.

Pour tout entier $n$, on note $u_n$ le nombre d'habitants de la ville A et $v_n$ le nombre d'habitants de la ville B à la fin de l'année $2010 + n$.

\medskip

\begin{enumerate}
\item Calculer le nombre d'habitants de la ville A et le nombre d'habitants de la ville B à la fin de l'année 2011
\item Quelle est la nature des suites $\left(u_n\right)$ et $\left(v_n\right)$ ?
\item Donner l'expression de $u_n$ en fonction de $n$, pour tout entier naturel $n$ et calculer le nombre d'habitants de la ville A en 2020.
\item Donner l'expression de $v_n$ en fonction de $n$, pour tout entier naturel $n$ et calculer le nombre d'habitants de la ville B en 2020.
\item Reproduire et compléter sur la copie l'algorithme ci-dessous qui permet de déterminer au bout de combien d'années la population de la ville A dépasse celle de la ville B.

\begin{center}
\renewcommand{\arraystretch}{1.2}
\begin{tabular}{|l|}\hline
def année () : \hspace*{3cm}\\
\quad $u =\np{4600}$\\
\quad $v =\np{5100}$\\
\quad $n = 0$\\
\quad while \ldots\\
\qquad $u = \ldots$\\
\qquad $v = \ldots$\\
\qquad $n = \ldots$\\
\quad return $n$\\ \hline
\end{tabular}
\end{center}
\end{enumerate}

\bigskip

