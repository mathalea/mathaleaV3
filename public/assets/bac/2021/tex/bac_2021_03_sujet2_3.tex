
\medskip

Dans l'espace rapporté à un repère orthonormé \Oijk, on considère les points:

\text{A} de coordonnées (2~;~0~;~0), B de coordonnées (0~;~3~;~0) et C de coordonnées
(0~;~0~;~1).

\psset{unit=1cm}
\begin{center}
\begin{pspicture}(-0.5,-0.5)(12,6)
\pspolygon[fillstyle=solid,fillcolor=gray!20,linestyle=dashed](0,0)(2.4,4.5)(10.4,2)
%\psgrid
\psline(0,0)(8,0)(10.4,2)(10.4,4.5)(8,2.5)(8,0)
\psline(8,2.5)(0,2.5)(0,0)
\psline(0,2.5)(2.4,4.5)(10.4,4.5)
\psline[linestyle=dashed](0,0)(2.4,2)(2.4,4.5)
\psline[linestyle=dashed](2.4,2)(10.4,2)
\uput[dl](0,0){A}\uput[u](2.4,4.5){C}\uput[r](10.4,2){B}\uput[d](2.5,2){O}
\end{pspicture}
\end{center}

\medskip

L'objectif de cet exercice est de calculer l'aire du triangle ABC.

\medskip

\begin{enumerate}
\item 
	\begin{enumerate}
		\item Montrer que le vecteur $\vect{n}\begin{pmatrix}3\\2\\6\end{pmatrix}$ est normal au plan (ABC).
		\item En déduire qu'une équation cartésienne du plan (ABC) est : $3x + 2y + 6z - 6 = 0$.
	\end{enumerate}
\item  On note $d$ la droite passant par O et orthogonale au plan (ABC). 
	\begin{enumerate}
		\item Déterminer une représentation paramétrique de la droite $d$.
		\item Montrer que la droite $d$ coupe le plan (ABC) au point H de coordonnées $\left(\frac{18}{49}~;~\frac{12}{49}~;~\frac{36}{49}\right)$.
		\item Calculer la distance OH.
	\end{enumerate}
\item  On rappelle que le volume d'une pyramide est donné par: $V = \dfrac{1}{3}\mathcal{B}h$, où $\mathcal{B}$ est l'aire d'une
base et $h$ est la hauteur de la pyramide correspondant à cette base.

En calculant de deux façons différentes le volume de la pyramide OABC, déterminer l'aire du triangle ABC.
\end{enumerate}

\bigskip


