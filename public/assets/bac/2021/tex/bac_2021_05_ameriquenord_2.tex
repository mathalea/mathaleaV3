 
\medskip


Un biologiste s'intéresse à l'évolution de la population d'une espèce animale sur une île du Pacifique.

Au début de l'année 2020, cette population comptait $600$ individus. On considère que l'espèce sera menacée d'extinction sur cette île si sa population devient inférieure ou égale à 20 individus.

\smallskip

Le biologiste modélise le nombre d'individus par la suite $\left(u_n\right)$ définie par : 

\[\left\{\begin{array}{l c l}
u_0		&=	&0,6\\
u_{n+1}	&=	&0,75 u_n\left(1 - 0,15 u_n\right)
\end{array}\right.\]

où pour tout entier naturel $n$, $u_n$ désigne le nombre d'individus, en milliers, au début de l'année $2020 + n$.

\medskip

\begin{enumerate}
\item Estimer, selon ce modèle, le nombre d'individus présents sur l'île au début de l'année 2021 puis au début de l'année 2022.
\end{enumerate}

Soit $f$ la fonction définie sur l'intervalle [0~;~1] par 

\[f(x) = 0,75x (1 - 0,15x).\]

\begin{enumerate}[resume]
\item Montrer que la fonction $f$ est croissante sur l'intervalle [0~;~1] et dresser son tableau de variations.
\item Résoudre dans l'intervalle [0~;~1] l'équation $f(x) = x$.
\end{enumerate}

On remarquera pour la suite de l'exercice que, pour tout entier naturel $n$,\, $u_{n+1} = f\left(u_n\right)$.

\begin{enumerate}[resume]
\item 
	\begin{enumerate}
		\item Démontrer par récurrence que pour tout entier naturel $n$,\, $0 \leqslant u_{n+1} \leqslant  u_n \leqslant 1$.
		\item En déduire que la suite $\left(u_n\right)$ est convergente.
		\item Déterminer la limite $\ell$ de la suite $\left(u_n\right)$.
	\end{enumerate}
\item Le biologiste a l'intuition que l'espèce sera tôt ou tard menacée d'extinction.
	\begin{enumerate}
		\item Justifier que, selon ce modèle, le biologiste a raison.
		\item Le biologiste a programmé en langage Python la fonction \textbf{menace()} ci-dessous:

\begin{center}
\fbox{
\begin{tabular}{l}%\hline
def menace()\\
\quad u = 0,6\\
\quad n = 0\\
\quad while u > 0,02\\
\hspace{1cm} u = 0,75*u*(1-0,15*u)\\
\hspace{1cm} n = n+1\\
\quad return n\\ %\hline
\end{tabular}
}
\end{center}

Donner la valeur numérique renvoyée lorsqu'on appelle la fonction menace(). 

Interpréter ce résultat dans le contexte de l'exercice.
	\end{enumerate}
\end{enumerate}

\bigskip

