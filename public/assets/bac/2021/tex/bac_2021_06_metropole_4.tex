
\vspace{0.75cm}
\begin{tabular}[]{|l|}
\hline
Principaux domaines abordés :\\

Géométrie de l'espace rapporté  à un repère orthonormé ; orthogonalité dans l'espace\\
\hline
\end{tabular}

\vspace{0.5cm}

\begin{minipage}[]{7.5cm}
Dans un repère orthonormé \Oijk {} on considère

\begin{itemize}
\item [$\bullet$] le point A de coordonnées (1~;~3~;~2),

 \item[$\bullet$]le vecteur $\vv{u}$ de coordonnées $\begin{pmatrix} 1\\1\\0\\\end{pmatrix}$

\item[$\bullet$] la droite $d$ passant par l'origine O du repère et admettant pour vecteur directeur $\vv{u}$. 
\end{itemize}

Le but de cet exercice est de déterminer le point de $d$ le plus proche du point A et d’étudier quelques propriétés de ce point.

On pourra s’appuyer sur la figure ci-contre pour raisonner au fur et à mesure des questions.
\end{minipage}
\begin{minipage}[]{7cm}
\psset{coorType=2,unit=1.5cm}
\begin{pspicture}(-2,-5)(4,3)
\pstThreeDCoor[xMin=0,xMax=3,yMin=-1.5,yMax=3.5,zMin=-1.5,zMax=2.5,IIIDticks]
\pstThreeDDot(1,3,2)\pstThreeDDot(1,3,0)\pstThreeDDot(2.5,2.5,0)\pstThreeDPut(2.5,2,0){$M_0$}
\pstThreeDPut(1,3.1,2.2){A}\pstThreeDPut(1,3,-0.3){A'}
\pstThreeDLine[linewidth=1.25pt]{->}(0,0,0)(1,1,0)\pstThreeDLine[linewidth=0.5pt]{-}(-2,-2,0)(4,4,0)
\pstThreeDLine[linewidth=1pt]{-}(2.5,2.5,0)(1,3,2)
\pstThreeDLine[linewidth=1pt]{->}(0,0,0)(1,3,2)
\pstThreeDLine[linewidth=1pt]{->}(0,0,0)(1,0,0)
\pstThreeDLine[linewidth=1pt]{-}(1,3,0)(1,3,2)
\pstThreeDLine[linewidth=1pt]{-}(2.5,2.5,0)(1,3,0)
\pstThreeDLine[linewidth=1pt,linestyle=dashed]{->}(0,0,0)(1,3,0)
\pstThreeDLine[linewidth=1pt]{->}(0,0,0)(0,0,1)
\pstThreeDLine[linewidth=1pt]{->}(0,0,0)(0,1,0)
\pstThreeDLine[linewidth=1pt,linestyle=dashed]{-}(0,0,0)(0,2.5,0)
\pstThreeDPut(1,-0.25,0){$\vv{\imath}$}\pstThreeDPut(0,1,0.15){$\vv{\jmath}$}
\pstThreeDPut(0,-0.2,1){$\vv{k}$}\pstThreeDPut(0,-0.1,-0.15){O}
\pstThreeDPut(0.9,0.65,0){$\vv{u}$}
\end{pspicture}

\end{minipage}


 \begin{enumerate}
\item  Déterminer une représentation paramétrique de la droite $d$. 

\item Soit $t$ un nombre réel quelconque, et $M$ un point de la droite $d$, le point $M$ ayant pour coordonnées $(t~;~t~;~0)$. 
\begin{enumerate}
\item  On note A$M$ la distance entre les points A et $M$. Démontrer que: 

\[AM^2 = 2t^2 - 8t+ 14.\] 

\item Démontrer que le point $M_0$ de coordonnées $(2~;~2~;~0)$ est le point de la droite $d$ pour lequel la distance $AM$ est minimale. 

On admettra que la distance $AM$ est minimale lorsque son carré $AM^2$ est minimal. 
\end{enumerate}
\item Démontrer que les droites $(AM_0)$ et $d$ sont orthogonales. 

\item On appelle $A'$ le projeté orthogonal du point $A$ sur le plan d'équation cartésienne $z = 0$. Le point $A'$ admet donc pour coordonnées $(1~;~3~;~0)$. 

Démontrer que le point $M_0$ est le point du plan $(AA'M_0)$ le plus proche du point O, origine du repère. 

\item Calculer le volume de la pyramide $OM_0A'A$. 

On rappelle que le volume d'une pyramide est donné par: $V = \frac{1}{3}\mathcal{B}h$, où $\mathcal{B}$ est l'aire  

d'une base et $h$ est la hauteur de la pyramide correspondant à cette base. 
\end{enumerate}

 \vspace{0.75cm}
 
