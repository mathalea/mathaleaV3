\textbf{\large Exercice 1 \hfill 5 points}

\bigskip

Cet exercice est un QCM (\textbf{Q}uestionnaire à \textbf{C}hoix \textbf{M}ultiples). Pour chacune des questions posées, une seule des quatre réponses est exacte. Recopier le numéro de la question et la réponse choisie. Aucune justification n'est demandée. Une réponse exacte rapporte 1 point, une réponse fausse ou l'absence de réponse ne rapporte ni n'enlève de point. Une réponse multiple ne rapporte aucun point.

\bigskip

\textbf{Question 1}

\medskip

On considère la fonction $f$ définie sur $\R$ par : $f(x)= (x+1)\e^{x}$.

La fonction dérivée $f'$ de $f$ est donnée sur $\R$ par:

\begin{center}
{\renewcommand{\arraystretch}{1.5}
\begin{tabularx}{\linewidth}{|X|X|X|X|}
\hline
\textbf{a.}\quad $f'(x)=\e^{x}$ & \textbf{b.}\quad $f'(x)=(x+2)\e^{x}$ 
& \textbf{c.}\quad $f'(x)=-x\e^{x}$   & \textbf{d.}\quad $f'(0)=0$\\
\hline
\end{tabularx}}
\end{center}

\medskip

\textbf{Question 2}

\medskip

Pour tous réels $a$ et $b$, le nombre $\dfrac{\e^{a}}{\e^{-b}}$ est égal à:

\begin{center}
{\renewcommand{\arraystretch}{1}
\begin{tabularx}{\linewidth}{|X|X|X|X|}
\hline
\textbf{a.}\quad $\e^{a-b}$ & \textbf{b.}\quad $\e^{\frac{a}{-b}}$ 
& \textbf{c.}\quad $\dfrac{\e^{b}}{\e^{-a}}$  \rule[-10pt]{0pt}{27pt} & \textbf{d.}\quad $\e^{a}-\e^{-b}$\\
\hline
\end{tabularx}}
\end{center}

\medskip

\textbf{Question 3}

\medskip

Soit $\left(u_n\right)$  une suite arithmétique telle que $u_3= \dfrac{9}{2}$ et $u_6=3$.

Alors le premier terme $u_0$ et la raison $R$ de la suite sont :

\begin{center}
{\renewcommand{\arraystretch}{1}
\begin{tabularx}{0.8\linewidth}{|X|X|}
\hline
\textbf{a.}\quad $u_0=6$ et $R=-\dfrac{1}{2}$ \rule[-10pt]{0pt}{27pt}& \textbf{b.}\quad $u_0=\dfrac{1}{2}$ et $R=6$ \\
\hline
\textbf{c.}\quad $u_0=6$ et $R=\dfrac{1}{2}$  \rule[-10pt]{0pt}{27pt} & \textbf{d.}\quad $u_0=\dfrac{3}{2}$ et $R=\dfrac{1}{2}$\\
\hline
\end{tabularx}}
\end{center}

\medskip

\textbf{Question 4}

\medskip

On considère le programme écrit en langage Python ci-dessous.

\begin{center}
\fbox{\tt
\begin{tabular}{l}
s = 0\\
for i in range(51) :\\
\quad s = s + i
\end{tabular}
}
\end{center}

Quelle est la valeur contenue dans la variable \texttt{s} après exécution du programme ?

\begin{center}
{\renewcommand{\arraystretch}{1.5}
\begin{tabularx}{\linewidth}{|X|X|X|X|}
\hline
\textbf{a.}\quad $51$ & \textbf{b.}\quad $\np{1326}$ 
& \textbf{c.}\quad $\np{1275}$   & \textbf{d.}\quad $\np{2500}$\\
\hline
\end{tabularx}}
\end{center}

\medskip

\textbf{Question 5}

\medskip

La valeur exacte de la somme  $S= 1+ \dfrac{1}{2} + \left (\dfrac{1}{2}\right )^2 + \cdots + \left ( \dfrac{1}{2}\right )^{15}$ est:

\begin{center}
{\renewcommand{\arraystretch}{1.7}
\begin{tabularx}{\linewidth}{|X|X|X|X|}
\hline
\textbf{a.}\quad $\np{1,750030518}$ & \textbf{b.}\quad $2 - \left ( \frac{1}{2}\right )^{15}$ 
& \textbf{c.}\quad $2 - \left ( \frac{1}{2}\right )^{14}$   & \textbf{d.}\quad $\np{1,999969482}$\\
\hline
\end{tabularx}}
\end{center}


