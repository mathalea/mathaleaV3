\textbf{EXERCICE -- B}

\medskip

\begin{tabular}{|l|}\hline
\textbf{Principaux domaines abordés}\\
-- Suites, étude de fonction\\
-- Fonction logarithme\\ \hline
\end{tabular}

\medskip

Soit la fonction $f$ définie sur l'intervalle $]1~;~ +\infty[$ par 

\[f(x) = x - \ln (x - 1).\]

On considère la suite $\left(u_n\right)$ de terme initial $u_0 = 10$ et telle que $u_{n+1} = f\left(u_n\right)$ pour tout entier naturel $n$.

\bigskip

\textbf{Partie I :}

\medskip

La feuille de calcul ci-dessous a permis d'obtenir des valeurs approchées des premiers termes de la suite $\left(u_n\right)$.

\begin{center}
\begin{tabularx}{0.6\linewidth}{|c|*{2}{>{\centering \arraybackslash}X|}}\hline
&A &B\\ \hline
1 &$n$&$u_n$\\ \hline
2 &0&10\\ \hline
3& 1&\np{7,80277542}\\ \hline
4& 2&\np{5,88544474}\\ \hline
5& 3&\np{4,29918442}\\ \hline
6& 4&\np{3,10550913}\\ \hline
7& 5&\np{2,36095182}\\ \hline
8& 6&\np{2,0527675}\\ \hline
9& 7&\np{2,00134509}\\ \hline
10& 8&\np{2,0000009}\\ \hline
\end{tabularx}
\end{center}

\medskip

\begin{enumerate}
\item Quelle formule a été saisie dans la cellule B3 pour permettre le calcul des valeurs approchées de $\left(u_n\right)$ par recopie vers le bas ?
\item À l'aide de ces valeurs, conjecturer le sens de variation et la limite de la suite $\left(u_n\right)$.
\end{enumerate}

\bigskip

\textbf{Partie II :}

\medskip

On rappelle que la fonction $f$ est définie sur l'intervalle $]1~;~ +\infty[$ par 

\[f(x) = x - \ln (x - 1).\]

\medskip

\begin{enumerate}
\item Calculer $\displaystyle\lim_{x \to 1} f(x)$. On admettra que $\displaystyle\lim_{x \to + \infty} f(x) = + \infty$.
\item  
	\begin{enumerate}
		\item Soit $f'$ la fonction dérivée de $f$. Montrer que pour tout $x \in ]1~;~ +\infty[$,\,  $f'(x) = \dfrac{x - 2}{x - 1}$.
		\item En déduire le tableau des variations de $f$ sur l'intervalle $]1~;~ +\infty[$, complété par les limites.
		\item Justifier que pour tout $x
\geqslant  2$,\,  $f(x) \geqslant  2$.
	\end{enumerate}
\end{enumerate}

\bigskip

\textbf{Partie III :}

\medskip

\begin{enumerate}
\item En utilisant les résultats de la partie II, démontrer par récurrence que $u_n \geqslant  2$ pour tout entier naturel $n$.
\item Montrer que la suite $\left(u_n\right)$ est décroissante.
\item En déduire que la suite $\left(u_n\right)$ est convergente. On note $\ell$ sa limite.
\item On admet que $\ell$ vérifie $f(\ell) = \ell$. Donner la valeur de $\ell$.
\end{enumerate}
