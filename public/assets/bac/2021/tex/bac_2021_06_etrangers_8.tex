
\textbf{Commun à tous les candidats}

\medskip

ABCDEFGH est un cube. I est le centre de la face ADHE et J est un point du segment [CG].

Il existe donc $a \in [0~;~1]$ tel que $\vect{\text{CJ}} =a \vect{\text{CG}}$.

On note $(d)$ la droite passant par I et parallèle à (FJ).

On note K et L les points d'intersection de la droite $(d)$ et des droites (AE) et (DH).

On se place dans le repère $\left(\text{A}~;~\vect{\text{AB}},\,\vect{\text{AD}},\, \vect{\text{AE}}\right)$. 

\bigskip

\textbf{Partie A : Dans cette partie} \boldmath $a = \dfrac{2}{3}$ \unboldmath

\begin{center}
\psset{unit=1cm}
\begin{pspicture}(-0.5,0)(9.5,7.5)
\pspolygon[fillstyle=solid,fillcolor=lightgray](2,3.65)(4.2,4.05)(8.9,5.6)(6.8,5.2)%KLJF
\psframe(2,0.5)(6.8,5.2)%ABFE
\psline(6.8,0.5)(8.9,2.4)(8.9,7.2)(6.8,5.2)%BCGF
\psline(8.9,7.2)(4.2,7.2)(2,5.1)%GHE
\psline[linestyle=dashed](2,0.5)(4.2,2.4)(4.2,7.2)%ADH
\psline[linestyle=dashed](8.9,2.4)(4.2,2.4)
\psdots(2,0.5)(6.8,0.5)(8.9,2.4)(4.2,2.4)%ABCD
\uput[dl](2,0.5){A} \uput[d](6.8,0.5){B} \uput[r](8.9,2.4){C} \uput[ul](4.2,2.4){D} 
\psdots(2,5.2)(6.8,5.2)(8.9,7.2)(4.2,7.2)(8.9,2.4)(8.9,4)(8.9,5.6)(3.1,3.85)%EFGHPJI
\uput[l](2,5.2){E} \uput[u](6.8,5.2){F} \uput[ur](8.9,7.2){G} \uput[u](4.2,7.2){H} \uput[r](8.9,4){P} 
\uput[r](8.9,5.6){J}\uput[d](3.1,3.85){I}
\psline(2,3.65)(6.8,5.2)(8.9,5.6)(4.2,4.05)%KFJL
\psline(-0.2,3.25)(2,3.65)(4.2,4.05)%KL
\psdots(2,3.65)(4.2,4.05)%KL
\uput[ul](2,3.65){K}\uput[dr](4.2,4.05){L}
\psline[linestyle=dashed](4.2,4.05)(8.9,4.9)
\psdots[dotstyle=+,dotangle=45,dotscale=1.8](8.9,3.2)(8.9,4.8)(8.9,6.4)
\end{pspicture}
\end{center}
\medskip

\begin{enumerate}
\item Donner les coordonnées des points F{}, I et J.
\item Déterminer une représentation paramétrique de la droite $(d)$.
\item 
	\begin{enumerate}
		\item Montrer que le point de coordonnées $\left(0~;~ 0~;~\dfrac{2}{3}\right)$ est le point K.
		\item Déterminer les coordonnées du point L, intersection des droites $(d)$ et (DH).

	\end{enumerate}
\item  
	\begin{enumerate}
		\item Démontrer que le quadrilatère FJLK est un parallélogramme. 
		\item Démontrer que le quadrilatère FJLK est un losange.
		\item Le quadrilatère FJLK est-il un carré ?

	\end{enumerate}
\end{enumerate}

\bigskip

\textbf{Partie B : Cas général}

\medskip

On admet que les coordonnées des points K et L sont : K$\left(0~;~0~;~1- \dfrac{a}{2}\right)$ et L$\left(0~;~1~;~\dfrac{a}{2}\right)$.

On rappelle que $a \in [0~;~1]$.

\medskip

\begin{enumerate}
\item Déterminer les coordonnées de J en fonction de $a$.
\item Montrer que le quadrilatère FJLK est un parallélogramme.
\item Existe-t-il des valeurs de $a$ telles que le quadrilatère FJLK soit un losange ? Justifier.
\item Existe-t-il des valeurs de $a$ telles que le quadrilatère FJLK soit un carré ? Justifier.
\end{enumerate}

\bigskip

