
\bigskip

\textbf{Partie A :}

\medskip

Dans un pays, une maladie touche la population avec une probabilité de $0,05$. 

On possède un test de dépistage de cette maladie.

On considère un échantillon de $n$ personnes ($n \geqslant 20$) prises au hasard dans la population assimilé à un tirage avec remise.

On teste l'échantillon suivant cette méthode : on mélange le sang de ces $n$ individus, on teste le mélange. 

Si le test est positif, on effectue une analyse individuelle de chaque personne.

Soit $X_n$ la variable aléatoire qui donne le nombre d'analyses effectuées.

\medskip

\begin{enumerate}
\item Montrer $X_n$ prend les valeurs 1 et $(n + 1)$. 
\item Prouver que $P\left(X_n = 1\right) = 0,95^n$.

Établir la loi de $X_n$ en recopiant sur la copie et en complétant le tableau suivant: 

\begin{center}
\begin{tabularx}{0.6\linewidth}{|*{3}{>{\centering \arraybackslash}X|}}\hline
$x_i$& 1 &$n + 1$\\ \hline
$P\left(X_n = x_i\right)$&&\\ \hline
\end{tabularx}
\end{center}

\item Que représente l'espérance de $X_n$ dans le cadre de l'expérience ?

Montrer que $E\left(X_n\right) =n + 1 - n \times  0,95^n$. 
\end{enumerate}

\bigskip

\textbf{Partie B :}

\medskip

\begin{enumerate}
\item On considère la fonction $f$ définie sur $[20~;~ +\infty[$ par $f(x) = \ln (x) + x \ln (0,95)$.

Montrer que $f$ est décroissante sur $[20~;~ +\infty[$.
\item On rappelle que $\displaystyle\lim_{x \to + \infty} \dfrac{\ln x}{x} = 0$. Montrer que $\displaystyle\lim_{x \to + \infty} f(x) = - \infty$.
\item Montrer que $f(x) = 0$ admet une unique solution $a$ sur $[20~;~ +\infty[$. 

Donner un encadrement à 0,1 près de cette solution.
\item En déduire le signe de $f$ sur $[20~;~ +\infty[$. 
\end{enumerate}

\bigskip

\textbf{Partie C :}

\medskip

On cherche à comparer deux types de dépistages. 

La première méthode est décrite dans la partie A, la seconde, plus classique, consiste à tester tous les individus. 

La première méthode permet de diminuer le nombre d'analyses dès que $E\left(X_n\right)  < n$.

En utilisant la partie B, montrer que la première méthode diminue le nombre d'analyses pour des échantillons comportant $87$ personnes maximum.

\bigskip

