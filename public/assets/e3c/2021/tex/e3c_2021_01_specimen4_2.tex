\textbf{\large Exercice 2 \hfill 5 points}

\bigskip

Un rameur est une machine d'exercice physique simulant les mouvements d'une personne qui fait de l'aviron.

Il est souvent utilisé pour l'entraînement sportif afin d'améliorer sa condition physique.

La courbe ci-dessous représente la puissance (en Watt) en fonction du temps (en dixième de seconde) développée par un rameur débutant.

\medskip

\textbf{Partie A} : Répondre par lecture graphique aux deux questions suivantes

\medskip

\begin{enumerate}
\item Quelle est la puissance maximale atteinte par ce rameur ?
\item Pendant combien de temps la puissance développée reste-t-elle au-dessus de 100 Watts ?
\end{enumerate}

\begin{center}
\psset{xunit=2cm, yunit=0.05cm,labelFontSize=\scriptstyle}
\def\xmin {-0.8}   \def\xmax {4.6}
\def\ymin {-10}   \def\ymax {170}
\begin{pspicture*}(\xmin,\ymin)(\xmax,\ymax)
\psgrid[unit=1cm,subgriddiv=1,gridlabels=0,gridcolor=lightgray](-2,-1)(10,9)
\psaxes[arrowsize=3pt 2, ticksize=-2pt 2pt,Dx=0.5,Dy=20]{->}(0,0)(\xmin,\ymin)(\xmax,\ymax) 
\def\f{-8 x mul 32 add 2.7183 x exp mul} % définition de la fonction
\psplot[linecolor=blue,plotpoints=2000,linewidth=1.25pt]{0.2}{4}{\f}
\end{pspicture*}
\end{center}

\medskip

\textbf{Partie B} : Modélisation par une fonction

\medskip

On suppose que la courbe est la courbe représentative de la fonction $f$ définie sur l'intervalle $[0,2~;~4]$ par :

\[f(x)=(-8x+32)\e^{x}.\]

On note $f'$ la fonction dérivée de $f$. On admet que pour tout réel $x$ de l'intervalle $[0,2~;~4]$,

\[f'(x)=(-8x+24)\e^{x}.\]

\begin{enumerate}
\item Étudier le signe de $f'(x)$  puis en déduire les variations de $f$ sur $[0,2~;~4]$.
\item Déterminer la valeur exacte du maximum de la fonction $f$.

On suppose que le sportif améliore sa meilleure performance de 5\,\% tous les mois. Combien de mois d'entrainement seront-ils nécessaires pour qu'il dépasse les 200 W ?
\end{enumerate}


