\textbf{\large\textsc{Exercice 3} \hfill 5 points}

\textbf{Commun à tous les candidats}

\medskip

Soit la suite $\left(u_n\right)$ définie par : $u_0 = 1$ et, pour $n$,
$ u_{n+1}  = \dfrac{4u_n}{u_n + 4}.$

\smallskip

\begin{enumerate}
\item
%\parbox{0.7\linewidth}{
%La copie d'écran ci-contre présente les valeurs, calculées à l'aide d'un tableur, des
%termes de la suite $\left(u_n\right)$ pour $n$ variant de 0 à 12, ainsi que celles du quotient 
%$\dfrac{4}{u_n}$,  (avec, pour les valeurs de $u_n$, affichage de deux
%chiffres pour les parties décimales).
%
%À l'aide de ces valeurs, conjecturer l'expression de $\dfrac{4}{u_n}$  en fonction de $n$.
%
%Le but de cet exercice est de démontrer cette conjecture (question \textbf{5.}), et d'en déduire la limite de la suite $\left(u_n\right)$ (question \textbf{6.}).}\hfill
%\parbox{0.25\linewidth}{
%$\begin{array}{|*{3}{c|}}\hline
%n&u_n&\dfrac{4}{u_n}\\ \hline
%0 &1,00 &4\\ \hline
%1 &0,80 &5\\ \hline
%2 &0,67 &6\\ \hline
%3 &0,57 &7\\ \hline
%4 &0,50 &8\\ \hline
%5 &0,44 &9\\ \hline
%6 &0,40 &10\\ \hline
%7 &0,36 &11\\ \hline
%8 &0,33 &12\\ \hline
%9 &0,31 &13\\ \hline
%10 &0,29& 14 \\ \hline
%11 &0,27& 15 \\ \hline
%12 &0,25 &16\\ \hline
%\end{array}$}
On peut conjecturer que quel que soit $n$, \, $\dfrac{4}{u_n} = n + 4$.
\item On veut démontrer par récurrence que, pour tout entier naturel $n$, on a : $u_n > 0$.

\emph{Initialisation} : $u_0 = 1 > 0$ : la proposition est vraie au rang $0$.

\emph{Hérédité} : supposons qu'il existe $n \in \N$ tel que $u_n > 0$.

Alors $u_n + 4  >4 > 0$ donc : $ \dfrac{1}{u_n + 4} >0$ et donc $ \dfrac{4}{u_n + 4} >0$

Or $u_n>0$ (hypothèse de récurrence) donc $\dfrac{4u_n}{u_n + 4}>0$.

Soit finalement : $u_{n+1} >0$.; la proposition est vraie au rang $n + 1$.

La proposition est vraie au rang $0$ et si elle est vraie au rang $n \in \N$, elle est vraie au rang $n + 1$.

D'après le principe de récurrence on a démontré que pour tout $n \in \N$, \, $u_n > 0$.
\item %Démontrer que la suite $\left(u_n\right)$ est décroissante.
Quel que soit $n \in \N$, \, $u_{n+1} - u_n = \dfrac{4u_n}{u_n + 4} - u_n = \dfrac{4u_n - u_n^2 - 4u_n}{u_n + 4} = - \dfrac{u_n^2}{u_n + 4}$.

Comme $u_n + 4 > 4 > 0$ d'où l'inverse $\dfrac{1}{u_n + 4} > 0$ et comme $u_n^2 > 0$, $\dfrac{u_n^2}{u_n + 4} > 0$ et finalement 

$- \dfrac{u_n^2}{u_n + 4} < 0$.

On a donc quel que soit $n \in \N$, \, $u_{n+1} - u_n < 0$ : la suite $\left(u_n\right)$ est décroissante.
\item %Que peut-on conclure des questions \textbf{2.} et \textbf{3.} concernant la suite $\left(u_n\right)$ ?
La suite $\left(u_n\right)$ est décroissante et minorée par $0$ : d'après le théorème de la convergence monotone, la suite est convergente vers un réel $\ell \geqslant 0$

\item %On considère la suite $\left(v_n\right)$ définie pour tout entier naturel $n$ par : $v_n = \dfrac{4}{u_n}$.

%Démontrer que $\left(v_n\right)$ est une suite arithmétique. 

%Préciser sa raison et son premier terme. 

%En déduire, pour tout entier naturel $n$, l'expression de $v_n$ en fonction de $n$.
On a pour $n \in \N$, \\$v_{n+1} - v_n = \dfrac{4}{u_{n+1}} - \dfrac{4}{u_n} = \dfrac{4}{\frac{4u_n}{u_n + 4}} - \dfrac{4}{u_n} = \dfrac{4\left(u_n + 4 \right)}{4u_n} - \dfrac{4}{u_n} = \dfrac{u_n + 4}{u_n} - \dfrac{4}{u_n} = \dfrac{u_n + 4 - 4}{u_n} = \dfrac{u_n}{u_n} = 1$.

$v_{n+1} - v_n = 1$, quel que soit $n \in \N$ montre que la suite $\left(v_n\right)$ est une suite arithmétique de raison $r = 1$.

Son premier terme est $v_0 = \dfrac{4}{u_0} = \dfrac{4}{1} = 4$.

\item %Déterminer, pour tout entier naturel $n$, l'expression de $u_n$ en fonction de $n$.
On sait qu'alors pour $n \in \N$, \, $v_n = v_0 + nr = 4 + n$.

Or $v_n = \dfrac{4}{u_n} \iff u_n = \dfrac{4}{v_n}$ donc $u_n = \dfrac{4}{4 + n}$ quel que soit $n \in \N$.

$u_n = \dfrac{1}{1 + \frac{n}{4}}$, donc comme $\displaystyle\lim_{n \to + \infty} 1 + \frac{n}{4} = + \infty$, \, $\displaystyle\lim_{n \to + \infty} u_n = 0$.

%En déduire la limite de la suite $\left(u_n\right)$.
La limite de la suite $\left(u_n\right)$ est donc $0$.
\end{enumerate}

\bigskip

