\textbf{\large Exercice 4 \hfill 5 points}

\bigskip

Le plan est rapporté à un repère orthonormé \Oij{} d'unité 1 cm.

On considère la droite $\mathcal D$ d'équation $x+3y-5=0$.

\medskip

\begin{enumerate}
\item Montrer que le point A de coordonnées $(2~;~1)$ appartient à la droite $\mathcal D$ et tracer la droite $\mathcal D$ dans le repère \Oij.
\item Montrer que la droite $\mathcal D'$ passant par le point B de coordonnées $(4~;~2)$ et perpendiculaire à la droite $\mathcal D$, admet pour équation $3x-y-10=0$.
\item Soit H le projeté orthogonal de B sur la droite $\mathcal D$.

Déterminer, par le calcul, les coordonnées de H.
\item On considère le cercle $\mathcal C$ de diamètre [AB] et on note $\Omega$ son centre.
	\begin{enumerate}
		\item Déterminer une équation de $\mathcal C$ ; préciser son rayon et les coordonnées de $\Omega$.
		\item Le point H appartient-il à $\mathcal C$ ? Justifier.
	\end{enumerate}
\end{enumerate}

