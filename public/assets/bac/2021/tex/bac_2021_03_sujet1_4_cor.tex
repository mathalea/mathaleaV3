\textbf{\large Exercice A}

\medskip

\begin{tabularx}{\linewidth}{|X|}\hline
\textbf{Principaux domaines abordés : Suites numériques; raisonnement par récurrence ; suites géométriques.}\\ \hline
\end{tabularx}

\medskip

La suite $\left(u_n\right)$ est définie sur $\N$ par $u_0 = 1$ et pour tout $n$, 
$u_{n+1} = \dfrac{3}{4}u_n + \dfrac{1}{4}n + 1$.


%\smallskip

\begin{enumerate}
\item %Calculer, en détaillant les calculs, $u_1$ et $u_2$ sous forme de fraction irréductible.
Pour $n=0$, $u_1=u_{0+1}=\dfrac{3}{4}u_0 + \dfrac{1}{4}\times 0 +1 =\dfrac{3}{4}\times 1  +1 = \dfrac{7}{4}$.

Pour $n=1$, $u_1=u_{1+1}=\dfrac{3}{4}u_1 + \dfrac{1}{4}\times 1+1 = \dfrac{3}{4}\times \dfrac{7}{4}+\dfrac{1}{4} +1 = \dfrac{41}{16}$. 

\end{enumerate}

\medskip

\parbox{0.5\linewidth}{L'extrait, reproduit ci-contre, d'une feuille de calcul réalisée avec un tableur présente les valeurs des premiers termes de la suite $\left(u_n\right)$.} \hfill
\parbox{0.35\linewidth}{
\begin{tabularx}{\linewidth}{|c|*{2}{>{\centering \arraybackslash}X|}}\hline
\cellcolor{lightgray}{} 	& \cellcolor{lightgray}A&\cellcolor{lightgray}B \\ \hline
\cellcolor{lightgray}1	&$n$&$u_n$\\ \hline
\cellcolor{lightgray}2 	&0	&1\\ \hline
\cellcolor{lightgray}3 	&1	&1,75\\ \hline
\cellcolor{lightgray}4 	&2	&\np{2,5625}\\ \hline
\cellcolor{lightgray}5 	&3	&\np{3,421875}\\ \hline
\cellcolor{lightgray}6 	&4	&\np{4,31640625}\\ \hline
\end{tabularx}}

\medskip

\begin{enumerate}[resume]
\item
	\begin{enumerate}
		\item La formule, étirée ensuite vers le bas, que l'on peut écrire dans la cellule B3 de la feuille de calcul pour obtenir les termes successifs de $\left(u_n\right)$ dans la colonne B est:
		
\text{= 3/4 * B2 + 1/4 * A2 +1}.		
		
		\item La suite $\left(u_n\right)$ semble croissante.
	\end{enumerate}
\item
	\begin{enumerate}
		\item Soit $\mathcal{P}_n$ la propriété: $n \leqslant u_n \leqslant n + 1$.
		
\begin{list}{\textbullet}{}
\item \textbf{Initialisation}

Pour $n=0$, $u_0=1$ et $0 \leqslant 1 \leqslant 1$ donc $\mathcal{P}_0$ est vraie.

\item \textbf{Hérédité}

On suppose $\mathcal{P}_n$ vraie, c'est-à-dire: $n \leqslant u_n \leqslant n+1$ (hypothèse de récurrence).

$n \leqslant u_n \leqslant n+1 \iff \dfrac{3}{4}n \leqslant \dfrac{3}{4}u_n \leqslant \dfrac{3}{4}\left (n+1\right )\\[7pt]
 \phantom{n \leqslant u_n \leqslant n+1}
 \iff \dfrac{3}{4}n +\dfrac{1}{4} n \leqslant \dfrac{3}{4}u_n  +\dfrac{1}{4} n  \leqslant \dfrac{3}{4}\left (n+1\right )  +\dfrac{1}{4} n \\[7pt]
 \phantom{n \leqslant u_n \leqslant n+1}
 \iff n \leqslant \dfrac{3}{4}u_n  +\dfrac{1}{4} n  \leqslant  n+\dfrac{3}{4}\\[7pt]
 \phantom{n \leqslant u_n \leqslant n+1}
 \iff n+1 \leqslant \dfrac{3}{4}u_n  +\dfrac{1}{4} n +1  \leqslant  n+\dfrac{3}{4} +1
 \iff n+1 \leqslant u_{n+1} \leqslant n+\dfrac{7}{4}$ 
 
 donc $n+1 \leqslant u_{n+1} \leqslant n+2$.

On a démontré que la propriété était vraie au rang $n+1$.

\item \textbf{Conclusion}

La propriété est vraie au rang 0, et elle est héréditaire pour tout $n\geqslant 0$; d'après le principe de récurrence, la propriété est vraie pour tout $n\geqslant 0$.
\end{list}		

On a donc démontré que, pour tout entier naturel $n$, on a: $n \leqslant u_n \leqslant n + 1$.
		
		\item% En déduire, en justifiant la réponse, le sens de variation et la limite de la suite 		$\left(u_n\right)$.
\begin{list}{\textbullet}{D'après la question précédente:}
\item Pour tout $n$, $n\leqslant u_n \leqslant n+1$ donc $n+1 \leqslant u_{n+1} \leqslant n+2$ donc 

$n \leqslant u_n \leqslant n+1 \leqslant u_{n+1} \leqslant n+2$ d'où on tire $u_n \leqslant u_{n+1}$ ce qui démontre que la suite $(u_n)$ est croissante.
\item Pour tout $n$, $n\leqslant u_n$; or $\ds\lim_{n\to +\infty} n = +\infty$ donc, par comparaison, $\ds\lim_{n\to +\infty} u_n = +\infty$.
\end{list}		
		
		
		\item% Démontrer que : $\displaystyle\lim_{n \to + \infty} \dfrac{u_n}{n} = 1$.
Pour tout $n$, $n\leqslant u_n \leqslant n+1$ donc pour tout $n>0$, on a:
$1 \leqslant \dfrac{u_n}{n} \leqslant \dfrac{n+1}{n}$ c'est-à-dire: 
$1 \leqslant \dfrac{u_n}{n} \leqslant 1+\dfrac{1}{n}$.

$\ds\lim_{n \to +\infty} \dfrac{1}{n}=0$ donc $\ds\lim_{n \to +\infty} 1+\dfrac{1}{n}=1$

Donc, d'après le théorème des gendarmes: $\ds\lim_{n\to +\infty} \dfrac{u_n}{n}=1$.

	\end{enumerate}
\item  On désigne par $\left(v_n\right)$ la suite définie sur $\N$ par $v_n = u_n - n$
	\begin{enumerate}
		\item% Démontrer que la suite $\left(v_n\right)$ est géométrique de raison $\dfrac{3}{4}$.
Pour tout $n$, $v_n=u_n-n$ donc $u_n=v_n+n$.

$v_{n+1}=u_{n+1} - (n+1) = 	\dfrac{3}{4}u_n + \dfrac{1}{4}n+1 -n-1
=\dfrac{3}{4}\left (v_n+n\right ) -\dfrac{3}{4}n
=\dfrac{3}{4}v_n +\dfrac{3}{4}n -\dfrac{3}{4}n
=\dfrac{3}{4}v_n$

$v_0=u_0-0=1$

Donc la suite $(v_n)$ est géométrique de raison $q=\dfrac{3}{4}$ et de premier terme $v_0=1$.
		
		
		\item %En déduire que, pour tout entier naturel $n$,on a : $u_n = \left(\dfrac{3}{4}\right)^n + n$.
On en déduit que, pour tout $n$, $v_n=v_0\times q^n = \left (\dfrac{3}{4}\right )^n$.

Comme $u_n=v_n+n$, on a $u_n=\left (\dfrac{3}{4}\right )^n + n$.		
		
	\end{enumerate}
\end{enumerate}

\bigskip

