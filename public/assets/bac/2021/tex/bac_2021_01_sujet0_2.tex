
\medskip

On considère le cube ABCDEFGH de côté 1, le milieu I de [EF] et J le symétrique de E par rapport à F{}.
\begin{center}

\psset{xunit=4cm,yunit=4cm}
\begin{pspicture}(-0.1,-0.1)(2,1.4)
%\psgrid
\psframe(0,0)(1,1)%ABFE
\psline(1,0)(1.35,0.35)
\psline[linestyle=dashed](0,0)(0.35,0.35)%AD
\psline[linestyle=dashed](0.35,0.35)(1.35,0.35)%DC
\psline[linestyle=dashed](0.35,0.35)(0.35,1.35)%DH
\psline(1.35,0.35)(1.35,1.35)%CG
\psline(0,1)(0.35,1.35)%EH
\psline(0.35,1.35)(1.35,1.35)%HG
\psline(1,1)(1.35,1.35)%FG
\uput[dl](0,0){A}\uput[dr](1,0){B}\uput[r](1.35,0.35){C}
\uput[l](0.35,0.35){D}\uput[l](0,1){E}\uput[ul](1.05,1){F}
\uput[ur](1.35,1.35){G}\uput[ul](0.35,1.35){H}
\uput[u](0.5,1){I}
%\psdots[dotstyle=Bar,dotscale =1.8](0.5,1)(2,1)
\psline[linewidth=0.8pt,linecolor=lightgray] (1,1)(2,1)\rput(2.05,1){J}
\psdots(0,0)(1,1)(0,1)(1,0)(1.35,0.35)(0.35,0.35)(0.5,1)(2,1)(1.35,1.35)(0.35,1.35)%AFEBCDIJGH
\end{pspicture}

\end{center}

\medskip

Dans tout l'exercice, l'espace est rapporté au repère orthonormé $\left(\text{A}~;~\vv{\text{AB}},~\vv{\text{AD}},~\vv{\text{AE}}\right)$.

\medskip

\begin{enumerate}
\item 
	\begin{enumerate}
		\item Par lecture graphique, donner les coordonnées des points I et J.
		\item En déduire les coordonnées des vecteurs $\vv{\text{DJ}}$ , $\vv{\text{BI}}$ et $\vv{\text{BG}}$.
		\item Montrer que $\vv{\text{DJ}}$ est un vecteur normal au plan (BGI).
		\item Montrer qu’une équation cartésienne du plan (BGI) est $2x - y + z - 2 = 0$.
	\end{enumerate}
\item On note $d$ la droite passant par F et orthogonale au plan (BGI).
	\begin{enumerate}
		\item Déterminer une représentation paramétrique de la droite $d$.
		\item On considère le point L de coordonnées $\left(\frac{2}{3}~;~\frac{1}{6}~;~\frac{5}{6}\right)$.

Montrer que L est le point d’intersection de la droite $d$ et du plan (BGI).
	\end{enumerate}
\item On rappelle que le volume $V$ d'une pyramide est donné par la formule
\[V=\dfrac{1}{3}\times \mathcal{B}\times h\]
où $\mathcal{B}$ est l'aire d’une base et $h$ la hauteur associée à cette base.
\begin{enumerate}
\item Calculer le volume de la pyramide FBGI.
\item En déduire l'aire du triangle BGI.
\end{enumerate}
\end{enumerate}

\bigskip

