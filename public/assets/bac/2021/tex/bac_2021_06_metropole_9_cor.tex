
\smallskip

\begin{tabular}{|l|}\hline
Principaux domaines abordés :\\
Fonction logarithme; dérivation\\ \hline
\end{tabular}

\begin{center}\textbf{Partie 1}\end{center}

Soit $h$ la fonction définie sur l'intervalle $]0~;~ +\infty[$ par:
$h(x) = 1 + \dfrac{\ln (x)}{x^2}.$

%On admet que la fonction $h$ est dérivable sur $]0~;~ +\infty[$ et on note $h'$ sa fonction dérivée.

\medskip

\begin{enumerate}
\item %Déterminez les limites de $h$ en $0$ et en $+ \infty$.
\starredbullet~Limite en $0$ : $h(x) = 1 + \dfrac{1}{x} \times \dfrac{\ln (x)}{x}$.

On sait que $\displaystyle\lim_{x \to 0} \dfrac{1}{x} = + \infty$ et que $\displaystyle\lim_{x \to 0} \dfrac{\ln (x)}{x} = - \infty$ ; donc par produit de limites $\displaystyle\lim_{x \to 0} h(x) = - \infty$.

\starredbullet~Limite en $+ \infty$ : on sait que $\displaystyle\lim_{x \to + \infty} \dfrac{1}{x} = 0$ et que $\displaystyle\lim_{x \to 0} \dfrac{\ln x}{x} = 0$, donc par produit et somme de limites : $\displaystyle\lim_{x \to 0} h(x) = 1$. ( La droite d'équation $y = 1$ est asymptote horizontale à la représentation graphique de $h$ en $+\infty$.)
  
\item %Montrer que, pour tout nombre réel $x$ de $]0~;~ +\infty[$,\, $h'(x) = \dfrac{1 - 2\ln (x)}{x^3}$.
La fonction est dérivable (admis) sur $]0~;~ +\infty[$ et sur cet intervalle :

$h'(x) = \dfrac{\frac{1}{x} \times x^2 - 2x\ln (x)}{x^4} = \dfrac{x - 2x\ln (x)}{x^4}  = \dfrac{1 - 2\ln (x)}{x^3}$.
\item %En déduire les variations de la fonction $h$ sur l'intervalle $]0~;~ +\infty[$.
Sur $]0~;~ +\infty[$, \, $x > 0$ donc $x^3 > 0$ : le signe de $h'(x)$ est donc celui du numérateur $1 - 2\ln x$.

\starredbullet~$1 - 2\ln x > 0 \iff 1 > 2\ln x \iff \dfrac{1}{2} > \ln x \iff  \ln x < \dfrac{1}{2}$, soit finalement $x < \e^{\frac{1}{2}}$ \, $\left(\text{ou encore }\, x < \sqrt{\e} \right)$

\item %Montrer que l'équation $h(x) = 0$ admet une solution unique $\alpha$ appartenant à $]0~;~+\infty[$ et vérifier que : $\dfrac{1}{2} < \alpha < 1$.
D'après les résultats précédents, on établit le tableau de variations de $h$ sur $]0~;~+\infty[$.

$h\left ( \e^{\frac{1}{2}}\right ) = 1+\dfrac{\ln\left (\e^{\frac{1}{2}}\right )}{\left (\e^{\frac{1}{2}}\right )^2}= 1+\dfrac{\frac{1}{2}}{\e}= 1+\dfrac{1}{2\e}\approx 1,18$

\begin{center}
{\renewcommand{\arraystretch}{1.5}
\psset{nodesep=3pt,arrowsize=2pt 3}%  paramètres
\def\esp{\hspace*{2.5cm}}% pour modifier la largeur du tableau
\def\hauteur{20pt}% mettre au moins 20pt pour augmenter la hauteur
$\begin{array}{|c|l*4{c}|}
\hline
x & 0  & \esp & \e^{\frac{1}{2}} & \esp & +\infty \\ 
\hline
h'(x) &  \vline\;\vline  &   \pmb{+} & \vline\hspace{-2.7pt}0 & \pmb{-} & \\ 
\hline
 & \vline\;\vline &  &   \Rnode{max}{1+\dfrac{1}{2\e}}  &  &  \rule{0pt}{\hauteur} \\  
h(x) &  \vline\;\vline &     &  &  &  \rule{0pt}{\hauteur} \\ 
 & \vline\;\vline \Rnode{min1}{-\infty} &   &  &  &   \Rnode{min2}{1} \rule{0pt}{\hauteur}    
 \ncline{->}{min1}{max} 
 \ncline{->}{max}{min2} 
 \\ 
\hline
\end{array} $
}
\end{center}	

D'après ce tableau de variations, l'équation $h(x)=0$ admet une solution unique  dans l'intervalle $\left  ] 0~;~\e^{\frac{1}{2}}\right [$.

On appelle $\alpha$ cette solution; $h\left (\dfrac{1}{2}\right ) \approx -1,8<0$ et $h(1)=1>0$ donc $\dfrac{1}{2} < \alpha < 1$.

\item D'après les questions précédentes, on peut établir le tableau  signes de $h(x)$ pour $x$ appartenant à $]0~;~ +\infty[$:

\begin{center}
{\renewcommand{\arraystretch}{1.5}
\psset{nodesep=3pt,arrowsize=2pt 3}%  paramètres
\def\esp{\hspace*{1.5cm}}% pour modifier la largeur du tableau
\def\hauteur{20pt}% mettre au moins 20pt pour augmenter la hauteur
$\begin{array}{|c|l*4{c}|}
\hline
x & 0  & \esp & \alpha & \esp \esp & +\infty \\ 
\hline
h(x) &  \vline\;\vline  &   \pmb{-} & \vline\hspace{-2.7pt}0 & \pmb{+} & \\ 
\hline
\end{array} $
}
\end{center}	

\end{enumerate}

\bigskip

\begin{center}\textbf{Partie 2}\end{center}

On désigne par $f_1$ et $f_2$ les fonctions définies sur $]0~;~ +\infty[$ par :
$f_1(x) = x-1 - \dfrac{\ln (x)}{x^2}\text{et  }f_2(x) = x - 2 - \dfrac{2\ln (x)}{x^2}.$

On note $\mathcal{C}_1$ et $\mathcal{C}_2$ les représentations graphiques respectives de $f_1$ et $f_2$ dans un repère \Oij.

\medskip

\begin{enumerate}
\item Pour tout nombre réel $x$ appartenant à $]0~;~ +\infty[$, on a : 

$f_1(x) -f_2(x) = x-1 - \dfrac{\ln (x)}{x^2} - \left ( x - 2 - \dfrac{2\ln (x)}{x^2}\right )
= x-1 - \dfrac{\ln (x)}{x^2} - x + 2 + \dfrac{2\ln (x)}{x^2}
= 1 + \dfrac{\ln(x)}{x^2}= h(x)$

\item %Déduire des résultats de la Partie 1 la position relative des courbes $\mathcal{C}_1$ et $\mathcal{C}_2$.
%
%On justifiera que leur unique point d'intersection a pour coordonnées $(\alpha~;~\alpha)$.
%
%On rappelle que $\alpha$ est l'unique solution de l'équation $h(x) = 0$.
\begin{list}{\textbullet}{}
\item On a vu que $h(x)<0$ sur $]0~;~\alpha[$, donc sur cet intervalle $f_1(x)<f_2(x)$ donc $\mathcal{C}_1$ est en dessous de $\mathcal{C}_2$.

\item On a vu que $h(x)>0$ sur $]\alpha~;~+\infty[$, donc sur cet intervalle $f_1(x)>f_2(x)$ donc $\mathcal{C}_1$ est au dessus de $\mathcal{C}_2$.

\item $h(\alpha)=0$ donc $f_1(\alpha)=f_2(\alpha)$; donc $\alpha$ est l'abscisse du point d'intersection de $\mathcal{C}_1$ et $\mathcal{C}_2$.

\item L'ordonnée de ce point d'intersection est $f(\alpha) = \alpha - 1 - \dfrac{\ln(\alpha)}{\alpha^2} = \alpha  -\left ( 1+\dfrac{\ln(\alpha)}{\alpha^2}\right )= \alpha -h(\alpha) = \alpha$. 

\item Les deux courbes se coupent donc au point de coordonnées $(\alpha~;~\alpha)$.

\end{list}
\end{enumerate}

\bigskip

