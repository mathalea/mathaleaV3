\textbf{\large\textsc{Exercice 3} \hfill  commun à tous les candidats \hfill 5 points}  

\medskip

Pour préparer l'examen du permis de conduire, on distingue deux types de formation :

\begin{itemize}
\item la formation avec \emph{conduite accompagnée} ;
\item la formation \emph{traditionnelle}.
\end{itemize}

On considère un groupe de 300 personnes venant de réussir l'examen du permis de conduire. Dans ce groupe :

\begin{itemize}
\item 75 personnes ont suivi une formation avec \emph{conduite accompagnée} ; parmi elles, 50 ont réussi l'examen à leur première présentation et les autres ont réussi à leur deuxième présentation.
\item  225 personnes se sont présentées à l'examen suite à une formation \emph{traditionnelle} ; parmi elles, 100 ont réussi l'examen à la première présentation, 75 à la deuxième et 50 à la troisième présentation.
\end{itemize}

On interroge au hasard une personne du groupe considéré.

On considère les évènements suivants :
\begin{itemize}
\item [] $A$ : \og la personne a suivi une formation avec \emph{conduite accompagnée} \fg{} ;
\item [] $R_1$ : \og la personne a réussi l'examen à la première présentation \fg{} ;
\item []$R_2$ : \og la personne a réussi l'examen à la deuxième présentation \fg{} ;
\item []$R_3$ : \og la personne a réussi l'examen à la troisième présentation \fg.
\end{itemize}

\medskip

\begin{enumerate}
\item  On modélise la situation par un arbre pondéré.

\begin{center}
\bigskip
{%\small
\psset{treesep=.75cm,levelsep=3cm,nodesepB=4pt, treesep=10mm}
\pstree[treemode=R,nodesepA=0pt]
       {\TR{}}
       {
       \pstree[nodesepA=4pt]{\TR{$A$}\ncput*{$\frac{75}{300}$}}
	                        {
	                        \TR{$R_1$}\ncput*{$\frac{50}{75}$}
			                \TR{$R_2$}\ncput*{$\frac{25}{75}$}
		                    \TR{$R_3$} \ncput*{$0$}
	                        }
       \pstree[nodesepA=4pt]{\TR{$\overline{A}$}\ncput*{$\frac{225}{300}$}}
                             {
	                        \TR{$R_1$}\ncput*{$\frac{100}{225}$}
			                \TR{$R_2$}\ncput*{$\frac{75}{225}$}
		                    \TR{$R_3$} \ncput*{$\frac{50}{225}$}
	                        }
      }
}% fin du \small
\bigskip
\end{center}

%\emph{Dans les questions suivantes, les probabilités demandées seront données sous forme d'une fraction irréductible.}

\item 
	\begin{enumerate}
		\item La probabilité que la personne interrogée ait suivi une formation avec \emph{conduite accompagnée} et réussi l'examen à sa deuxième présentation est:

$P\left (A\cap R_2\right ) = P(A) \times P_{A}\left (R_2\right ) = \frac{75}{300}\times\frac{25}{75}=\frac{25}{300}=\frac{1}{12}$.
				
		\item La probabilité que la personne interrogée ait réussi l'examen à sa deuxième présentation est égale à $P\left (R_2\right )$..
		
D'après la formule des probabilités totales:

$P\left (R_2\right ) = P\left (A\cap R_2\right ) + P\left (\overline{A}\cap R_2\right )
= \dfrac{25}{300} + \dfrac{125}{300}\times \dfrac{75}{125}
= \dfrac{25}{300} +  \dfrac{75}{300}
=\dfrac{100}{300} = \dfrac{1}{3}$.
		
		\item La personne interrogée a réussi l'examen à sa deuxième présentation. La probabilité qu'elle ait suivi une formation avec \emph{conduite accompagnée} est:
		
$P_{R_2}(A) = \dfrac{P\left (A\cap R_2 \right )}{P\left (R_2\right )}= \dfrac{\frac{1}{12}}{\frac{1}{3}} = \dfrac{3}{12}=\dfrac{1}{4}$.
		
	\end{enumerate}
\item On note $X$ la variable aléatoire qui, à toute personne choisie au hasard dans le groupe, associe le nombre de fois où elle s'est présentée à l'examen jusqu’à sa réussite.

Ainsi, ${X=1}$ correspond à l'évènement $R_1$.
	\begin{enumerate}
		\item La loi de probabilité de la variable aléatoire $X$ est:
		
\begin{center}
\begin{tabularx}{0.7\linewidth}{|c|*{3}{>{\centering \arraybackslash}X|}}
\hline
$x_i$ & 1 & 2 & 3\\
\hline
$p_i=P(X=x_i)$ & $P(R_1)$ & $P(R_2)$ & $P(R_3)$ \\
 \hline
\end{tabularx}
\end{center}
		
\begin{list}{\textbullet}{}
\item $P\left (R_1\right ) =  P\left (A\cap R_1\right ) + P\left (\overline{A}\cap R_1\right )
= \dfrac{75}{300}\times \dfrac{50}{75}+ \dfrac{225}{300}\times \dfrac{100}{225}
= \dfrac{50}{300} + \dfrac{100}{300} = \dfrac{150}{300}=\dfrac{1}{2}$
\item $P\left (R_2\right ) = \dfrac{1}{3}$
\item $P\left (R_3\right ) =  P\left (A\cap R_3\right ) + P\left (\overline{A}\cap R_3\right )
= 0 + \dfrac{225}{300}\times \dfrac{50}{225}
= \dfrac{50}{300} = \dfrac{1}{6}$
\end{list}		
		
Donc  la loi de probabilité de la variable aléatoire $X$ est:		
		
\begin{center}
\begin{tabularx}{0.5\linewidth}{|c|*{3}{>{\centering \arraybackslash}X|}}
\hline
$x_i$ & 1 & 2 & 3\\
\hline
$p_i=P(X=x_i)$ & $\dfrac{1}{2}$ & $\dfrac{1}{3}$ & $\dfrac{1}{6}$\rule[-10pt]{0pt}{28pt} \\
 \hline
\end{tabularx}
\end{center}		
		
		\item L'espérance de cette variable aléatoire est:
		$E(X)= \sum(x_i\times p_i)= 1\times\dfrac{1}{2} + 2 \times \dfrac{1}{3} + 3\times \dfrac{1}{6} = \dfrac{5}{3}\approx 1,67$.		

Cela veut dire que le nombre de passages pour réussir l'examen est en moyenne de $1,67$.		
	\end{enumerate}
\item On choisit, successivement et de façon indépendante, $n$ personnes parmi les 300 du groupe étudié, où $n$ est un entier naturel non nul. On assimile ce choix à un tirage avec remise de $n$ personnes parmi les 300 personnes du groupe.

On admet que la probabilité de l'évènement $R_3$ est égale à $\frac{1}{6}$.
	\begin{enumerate}
		\item  On cherche un évènement dont la probabilité est égale à $1-\left(\frac{5}{6}\right)^n$.

$P(R_3)=\frac{1}{6}$ donc $P\left (\overline{R_3} \right )=1-\frac{1}{6}=\frac{5}{6}$.
Le nombre $\frac{5}{6}$ est donc la probabilité de l'événement \og $R_1$ ou $R_2$ \fg{}, c'est-à-dire la probabilité qu'une personne prise au hasard réussisse l'examen à la première tentative ou à la deuxième.

La probabilité que $n$ personnes réussissent l'examen à la première ou à la deuxième tentative est de $\left (\frac{5}{6}\right )^n$.

L'événement de probabilité $1-\left (\frac{5}{6}\right )^n$ est l'événement contraire du précédent, donc correspond à l'événement \og au moins une personne n'a pas réussi l'examen  à la première ou à la deuxième tentative\fg{}, c'est-à-dire \og au moins une personne a réussi l'examen à la troisième tentative \fg{}.

 \bigskip
 
On considère la fonction Python \textbf{seuil} ci-dessous, où $p$ est un nombre réel appartenant à l'intervalle ]0;1[.
\begin{center}
\begin{tabular}[]{|l|}
\hline
\textbf{def seuil}(p):\\
\hspace{2em}n = 1\\
\hspace{2em}\textbf{while} 1$-$(5/6)**n $<=$ p:\\
\hspace{4.5em}n = n+1\\
\hspace{2em}\textbf{return} n\\
\hline
\end{tabular}
\end{center}

		\item %Quelle est la valeur renvoyée par la commande \textbf{seuil}(0,9) ? Interpréter cette valeur dans le contexte de l'exercice.
La valeur renvoyée par \textbf{seuil}(0.9) est la première valeur de $n$ pour laquelle $1-\left (\frac{5}{6}\right )^n >0,9$. \\
On résout cette inéquation:
		
$1-\left (\frac{5}{6}\right )^n >0,9
\iff 
0,1 > \left (\frac{5}{6}\right )^n
\iff
\ln(0,1) > \ln\left (\left (\frac{5}{6}\right )^n\right )
\iff
\ln(0,1) > n\ln\left (\frac{5}{6}\right )
\iff
\dfrac{\ln(0,1)}{\ln\left (\frac{5}{6}\right )} < n
$		

$\dfrac{\ln(0,1)}{\ln\left (\frac{5}{6}\right )}\approx 12,6$ donc la commande \textbf{seuil}(0.9) renvoie la valeur 13.

Il faut donc prendre $n=13$ personnes sur les 300 pour que la probabilité d'en avoir une qui a réussi l'examen à sa troisième tentative soit supérieure à $0,9$.
		
		
	\end{enumerate}
\end{enumerate}



