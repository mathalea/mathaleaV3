
\medskip

%Un biologiste s'intéresse à l'évolution de la population d'une espèce animale sur une île du Pacifique.
%
%Au début de l'année 2020, cette population comptait $600$ individus. On considère que l'espèce sera menacée d'extinction sur cette île si sa population devient inférieure ou égale à 20 individus.

%\smallskip

%Le biologiste modélise le nombre d'individus par la suite $\left(u_n\right)$ définie par : 

\[\left\{\begin{array}{l c l}
u_0		&=	&0,6\\
u_{n+1}	&=	&0,75 u_n\left(1 - 0,15 u_n\right)
\end{array}\right.\]

où pour tout entier naturel $n$, $u_n$ désigne le nombre d'individus, en milliers, au début de l'année $2020 + n$.

\medskip

\begin{enumerate}
\item ~
$\bullet~~$2021 correspond à $n = 1$, donc $u_1 = 0,75u_0 \times \left(1 - 0,15u_0\right) = 0,75 \times 0,6\times (1 - 0,15 \times 0,6) = 0,45 \times (1 - 0,09) = 0,45 \times0,91 = \np{0,4095}$ soit environ $410$ individus.

$\bullet~~$2022 correspond à $n = 2$, donc $u_2 = 0,75u_1 \times \left(1 - 0,15u_1\right) = 0,75 \times \np{0,4095}\times (1 - 0,15 \times \np{0,4095}) = \np{0,307125} \times (1 - \np{0,061425}) = \np{0,307125} \times \np{0,938575}) \approx \np{0,2282}$ soit environ 228~individus.
%Estimer, selon ce modèle, le nombre d'individus présents sur l'île au début de l'année 2021 puis au début de l'année 2022.
\end{enumerate}

Soit $f$ la fonction définie sur l'intervalle [0~;~1] par 

\[f(x) = 0,75x (1 - 0,15x).\]

\begin{enumerate}[resume]
\item %Montrer que la fonction $f$ est croissante sur l'intervalle [0~;~1] et dresser son tableau de variations.
$f$ est une fonction polynôme dérivable sur $\R$, donc sur [0~;~1] et sur cet intervalle :

$f'(x)  = 0,75(1 - 0,15x) - 0,75x \times 0,15 = 0,75 - \np{0,1125}x - \np{0,1125}x = 0,75 - 0,225x$.

Or $0 \leqslant x \leqslant 1 \Rightarrow 0 \leqslant 0,225x \leqslant 0,225 \Rightarrow - 0,225 \leqslant -0,225x \leqslant 0 \Rightarrow $

$0,75 - 0,225 \leqslant 0,75 - 0,225x \leqslant 0,75$ ou enfin $0,525 \leqslant f'(x) \leqslant 0,75$.

Sur [0~;~1], $f'(x) > 0$, donc $f$est strictement croissante de $f(0) = 0$ à $f(1) = 0,75 \times 0,85 = \np{0,6375}$.
\item %Résoudre dans l'intervalle [0~;~1] l'équation $f(x) = x$.

Sur [0~;~1], \, $f(x) = x \iff 0,75x (1 - 0,15x) = x \iff 0,75x(1 - 0,15x) - x = 0 \iff x[0,75(1 - 0,15x) - 1] = 0 \iff x(0,75 - \np{0,1125}x - 1) = 0 \iff x(- 0,25 - \np{0,1125}x) = 0 \iff \left\{\begin{array}{l c l}
x&=&0\,\text{ou}\\
-0,25 - \np{0,1125}x&=0&
\end{array}\right. \iff \left\{\begin{array}{l c l}
x&=&0\,\text{ou}\\
-0,25 &=&\np{0,1125}x
\end{array}\right. \iff \left\{\begin{array}{l c l}
x&=&0\,\text{ou}\\
-\frac{0,25}{{0,1125}}&=&x
\end{array}\right. $.

Or $-\frac{0,25}{{0,1125}} < 0$ donc dans [0~;~1], \, $S = \{0\}$.
\end{enumerate}

On remarquera pour la suite de l'exercice que, pour tout entier naturel $n$,\, $u_{n+1} = f\left(u_n\right)$.

\begin{enumerate}[resume]
\item 
	\begin{enumerate}
		\item ~%Démontrer par récurrence que pour tout entier naturel $n$,\, $0 \leqslant u_{n+1} \leqslant  u_n \leqslant 1$.
\emph{Initialisation} : on a vu que $0 \leqslant \np{0,4095} \leqslant 0,6 \leqslant 1$, soit $0 \leqslant u_1 \leqslant u_0\leqslant 1$ : la relation est vraie au rang $0$ ;

\emph{Hérédité} : Supposons que pour $n \in \N$, on ait :

$0 \leqslant u_{n+1} \leqslant  u_n \leqslant 1$ ; la fonction $f$ étant strictement croissante sur [0~;~1], on a donc :
$f(0) \leqslant f\left(u_{n+1}\right) \leqslant f\left(u_n\right) \leqslant f(1)$, 

soit puisque $f(0) = 0$ et $f(1) = 0,75 \times (1 - 0,15) = \np{0,6375} \leqslant 1$ :

$0 \leqslant u_{n+2} \leqslant u_{n+1} \leqslant  1$ : la relation est donc vraie au rang $n + 1$.

\textbf{Conclusion :} la relation est vraie au rang 0 et si elle est vraie au rang $n$ naturel quelconque, elle est vraie au rang $n+1$ : d'après le principe de récurrence :

Pour tout entier naturel $n$,\, $0 \leqslant u_{n+1} \leqslant  u_n \leqslant 1$.	
		
		\item %En déduire que la suite $\left(u_n\right)$ est convergente.
La suite $\left(u_n\right)$ est d'après la question précédente décroissante et minorée par $0$ ; elle est donc est convergente.
		\item %Déterminer la limite $\ell$ de la suite $\left(u_n\right)$.
Le résultat précédent montre que la suite $\left(u_n\right)$ converge vers un nombre $\ell \geqslant 0$ et ce nombre $\ell$ vérifie l'équation $f(x) = x$, dont on a vu à la question \textbf{3.} qu'elle n'avait que $0$ comme solution.
		
Conclusion : $\displaystyle\lim_{n \to + \infty} u_n = \ell = 0$.
	\end{enumerate}
\item %Le biologiste a l'intuition que l'espèce sera tôt ou tard menacée d'extinction.
	\begin{enumerate}
		\item %Justifier que, selon ce modèle, le biologiste a raison.
L'étude précédente a montré que le nombre d'individus décroit, donc le biologiste a raison puisque la limite de la suite du nombre d'individus est égale à zéro.
		\item %Le biologiste a programmé en langage Python la fonction \textbf{menace()} ci-dessous:

%\begin{center}
%\fbox{
%\begin{tabular}{l}%\hline
%def menace()\\
%\quad u = 0,6\\
%\quad n = 0\\
%\quad while u > 0,02\\
%\hspace{1cm} u = 0,75*u*(1-0,15*u)\\
%\hspace{1cm} n = n+1\\
%\quad return n\\ %\hline
%\end{tabular}
%}
%\end{center}
%
%Donner la valeur numérique renvoyée lorsqu'on appelle la fonction menace(). 
L'algorithme calcule les termes de la suite tant que ceux-ci sont supérieurs à 0,02

Il s'arrête à $n = 11$ car $u_{10} \approx 0,019$

L'espèce sera donc menacée d'extinction en 2031.
%Interpréter ce résultat dans le contexte de l'exercice.
	\end{enumerate}
\end{enumerate}


