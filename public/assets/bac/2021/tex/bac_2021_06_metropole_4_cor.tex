
\vspace{0.75cm}
\begin{tabular}[]{|l|}
\hline
Principaux domaines abordés :\\

Géométrie de l'espace rapporté  à un repère orthonormé ; orthogonalité dans l'espace\\
\hline
\end{tabular}

\vspace{0.5cm}

\begin{minipage}[]{5.5cm}
Dans un repère orthonormé \Oijk {} on considère

\begin{itemize}
\item [$\bullet$] le point A de coordonnées (1~;~3~;~2),

 \item[$\bullet$]le vecteur $\vv{u}$ de coordonnées $\begin{pmatrix} 1\\1\\0\\\end{pmatrix}$

\item[$\bullet$] la droite $d$ passant par l'origine O du repère et admettant pour vecteur directeur $\vv{u}$. 
\end{itemize}
%
%Le but de cet exercice est de déterminer le point de $d$ le plus proche du point A et d’étudier quelques propriétés de ce point.
%
%On pourra s’appuyer sur la figure ci-contre pour raisonner au fur et à mesure des questions.
\end{minipage}
\begin{minipage}[]{4.5cm}
\psset{coorType=2,unit=1.5cm}
\begin{pspicture}(-2.5,-2.5)(4,3)
\pstThreeDCoor[xMin=0,xMax=3,yMin=-1.5,yMax=3.5,zMin=-1.5,zMax=2.5,IIIDticks]
\pstThreeDDot(1,3,2)\pstThreeDDot(1,3,0)\pstThreeDDot(2.5,2.5,0)\pstThreeDPut(2.5,2,0){$M_0$}
\pstThreeDPut(1,3.1,2.2){A}\pstThreeDPut(1,3,-0.3){A$'$}
\pstThreeDLine[linewidth=1.25pt]{->}(0,0,0)(1,1,0)\pstThreeDLine[linewidth=0.5pt]{-}(-2,-2,0)(4,4,0)
\pstThreeDLine[linewidth=1pt]{-}(2.5,2.5,0)(1,3,2)
\pstThreeDLine[linewidth=1pt]{->}(0,0,0)(1,3,2)
\pstThreeDLine[linewidth=1pt]{->}(0,0,0)(1,0,0)
\pstThreeDLine[linewidth=1pt]{-}(1,3,0)(1,3,2)
\pstThreeDLine[linewidth=1pt]{-}(2.5,2.5,0)(1,3,0)
\pstThreeDLine[linewidth=1pt,linestyle=dashed]{->}(0,0,0)(1,3,0)
\pstThreeDLine[linewidth=1pt]{->}(0,0,0)(0,0,1)
\pstThreeDLine[linewidth=1pt]{->}(0,0,0)(0,1,0)
\pstThreeDLine[linewidth=1pt,linestyle=dashed]{-}(0,0,0)(0,2.5,0)
\pstThreeDPut(1,-0.25,0){$\vv{\imath}$}\pstThreeDPut(0,1,0.15){$\vv{\jmath}$}
\pstThreeDPut(0,-0.2,1){$\vv{k}$}\pstThreeDPut(0,-0.1,-0.15){O}
\pstThreeDPut(0.9,0.65,0){$\vv{u}$}
\end{pspicture}

\end{minipage}


 \begin{enumerate}
\item %Déterminer une représentation paramétrique de la droite $d$.
$M(x~;~y~;~z) \in (d) \iff \vect{\text{O}M} = t\vect{u}, \, \text{avec } t \in \R$, soit :
$\left\{\begin{array}{l c l}
x&=&t\\
y&=&t\\
z&=&0
\end{array}\right., \, t \in \R$.

\item %Soit $t$ un nombre réel quelconque, et $M$ un point de la droite $d$, le point $M$ ayant pour coordonnées $(t~;~t~;~0)$. 
\begin{enumerate}
\item  %On note $AM$ la distance entre les points $A$ et $M$. Démontrer que: 
De $\vect{\text{A}M}\begin{pmatrix}t - 1\\t - 3\\0 - 2\end{pmatrix}$, on calcule :

A$M^2 = (t - 1)^2 + (t - 3)^2 + (- 2)^2 = t^2 + 1 - 2t + t^2 + 9 - 6t + 4 = 2t^2 - 8t+ 14.$
%\[AM^2 = 2t^2 - 8t+ 14.\] 

\item %Démontrer que le point $M_0$ de coordonnées $(2~;~2~;~0)$ est le point de la droite $d$ pour lequel la distance $AM$ est minimale. 

%On admettra que la distance $AM$ est minimale lorsque son carré $AM^2$ est minimal.
$2t^2 - 8t+ 14 = 2\left(t^2 - 4t + 7\right) = 2\left[(t - 2)^2 -  4 + 7 \right] = 2\left[(t - 2)^2 + 3\right]$.

La plus petite valeur de ce trinôme est obtenue quand le carré est nul, soit pour $t = 2$.

On a: $2t^2 - 8t+ 14 \geqslant 6$, soit $\text{A}M^2 \geqslant 6 \Rightarrow \text{A}M \geqslant \sqrt{6}$.

La plus petite distance est A$M_0 = \sqrt{6}$ avec $M_0(2~;~2~;~0)$.
\end{enumerate}
\item %Démontrer que les droites $(AM_0)$ et $d$ sont orthogonales.
On a: $\vect{\text{A}M_0}\begin{pmatrix}1\\- 1\\0\end{pmatrix}$ et $\vect{u}\begin{pmatrix}1\\ 1\\0\end{pmatrix}$ est un vecteur directeur de $(d)$.

On a: $\vect{\text{A}M_0} \cdot \vect{u} = 1 - 1 + 0 = 0$ : les vecteurs sont orthogonaux donc les droites $\left(\text{A}M_0\right)$ et $d$ sont orthogonales.
\item %On appelle $A'$ le projeté orthogonal du point $A$ sur le plan d'équation cartésienne $z = 0$. Le point $A'$ admet donc pour coordonnées $(1~;~3~;~0)$. 

%Démontrer que le point $M_0$ est le point du plan $\left(\text{AA}'M_0\right)$ le plus proche du point O, origine du repère.
$\vect{u}$ est orthogonal au plan horizontal d'équation $z = 0$. Comme A$'$ et $M_0$ appartiennent à ce plan le vecteur $\vect{u}$ est orthogonal au vecteur $\vect{\text{A}'M_0}$.

Donc le vecteur $\vect{u}$ est orthogonal à deux vecteurs non colinéaires du plan $\left(\text{AA}'M_0\right)$,donc la droite $(d)$ est orthogonale au plan 
$\left(\text{AA}'M_0\right)$. Le point $M_0$ est donc le projeté orthogonal de O sur le plan $\left(\text{AA}'M_0\right)$, donc O$M_0$ est la distance la plus courte du point O au plan $\left(\text{AA}'M_0\right)$.

\item %Calculer le volume de la pyramide $\text{O}M_0\text{A}'\text{A}$. 

%On rappelle que le volume d'une pyramide est donné par: $V = \frac{1}{3}\mathcal{B}h$, où $\mathcal{B}$ est l'aire  

%d'une base et $h$ est la hauteur de la pyramide correspondant à cette base.
Aire de la base AA$'M_0$ : on a AA$' = 2$ et A$'M_0^2 = (2 - 1)^2 + (2 - 3)^2 + 0^2 = 1 + 1 = 2$. D'où A$'M_0 = \sqrt{2}$.

On a donc $\mathcal{A}\left(\text{AA}'M_0\right) = \dfrac{2 \times \sqrt{2}}{2} = \sqrt{2}$.

D'autre part : O$M_0^2 = 2^2 + 2^2 = 8$, d'où O$M_0 = \sqrt{8} = \sqrt{4 \times 2} = 2\sqrt{2} = h$.

Finalement $V  = \dfrac{\sqrt{2} \times 2\sqrt{2}}{3} = \dfrac{4}{3}$.
\end{enumerate}

\vspace{0.75cm}

