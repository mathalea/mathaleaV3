
\medskip

Soit $f$ la fonction définie sur l'intervalle $\left]-\dfrac{1}{3}~;~+\infty\right[$ par:

\[ f(x) = \dfrac{4x}{1 + 3x}\]

On considère la suite $\left(u_n\right)$ définie par : $u_0 = \dfrac{1}{2}$ et, pour tout entier naturel $n$,\, $u_{n+1} = f\left(u_n\right)$.

\medskip

\begin{enumerate}
\item Calculer $u_1$.
\item On admet que la fonction $f$ est croissante sur l'intervalle $\left]-\dfrac{1}{3}~;~+\infty\right[$.
	\begin{enumerate}
		\item Montrer par récurrence que, pour tout entier naturel $n$, on a : $\dfrac{1}{2} \leqslant u_n \leqslant u_{n+1} \leqslant 2$.
		\item En déduire que la suite $\left(u_n\right)$ est convergente.
		\item On appelle $\ell$ la limite de la suite $\left(u_n\right)$. Déterminer la valeur de $\ell$.
	\end{enumerate}
\item 
	\begin{enumerate}
		\item Recopier et compléter la fonction Python ci-dessous qui, pour tout réel positif $E$, détermine la plus petite valeur $P$ tel que : $1 - u_{P} < E$.
		
\begin{center}
\begin{tabularx}{0.4\linewidth}{|X|}\hline
def seuil($E$):\\
\quad u = 0,5\\
\quad n = 0 \\
\quad  while \dotfill\\
\quad \quad u = \dotfill\\
\quad \quad n = n + 1\\
\quad return n\\ \hline
\end{tabularx}
\end{center}
		\item Donner la valeur renvoyée par ce programme dans le cas où $E= 10^{-4}$.
	\end{enumerate}
\item On considère la suite $\left(v_n\right)$ définie, pour tout entier naturel $n$, par :

\[v_n  = \dfrac{u_n}{1 - u_n}\]

	\begin{enumerate}
		\item Montrer que la suite $\left(v_n\right)$ est géométrique de raison 4.
		
En déduire, pour tout entier naturel $n$, l'expression de $v_n$ en fonction de $n$.
		\item Démontrer que, pour tout entier naturel $n$, on a : $u_n = \dfrac{v_n}{v_n + 1}$. 
		\item Montrer alors que, pour tout entier naturel $n$ , on a :
		
		\[u_n = \dfrac{1}{1 + 0,25^n}.\]

Retrouver par le calcul la limite de la suite $\left(u_n\right)$.
	\end{enumerate}
\end{enumerate}

\bigskip

