\textbf{\large Exercice 2 \hfill 5 points}

\bigskip

Un rameur est une machine d'exercice physique simulant les mouvements d'une personne qui fait de l'aviron.
Il est souvent utilisé pour l'entraînement sportif afin d'améliorer sa condition physique.
La courbe ci-dessous représente la puissance (en Watt) en fonction du temps (en dixième de seconde) développée par un rameur débutant.

\medskip

\textbf{Partie A} :  lecture graphique

\medskip

\begin{enumerate}
\item La puissance maximale atteinte par ce rameur est d'environ 160 Watts.
\item La puissance développée reste au-dessus de 100 Watts entre environ $1,7$ et $3,7$ dixièmes de seconde, soit pendant 2 dixièmes de seconde.
\end{enumerate}

\begin{center}
\psset{xunit=2cm, yunit=0.05cm,labelFontSize=\scriptstyle}
\def\xmin {0}   \def\xmax {4.6}
\def\ymin {-10}   \def\ymax {170}
\begin{pspicture*}(-0.28,\ymin)(\xmax,\ymax)
\psgrid[unit=1cm,subgriddiv=1,gridlabels=0,gridcolor=lightgray](0,-1)(10,9)
\psaxes[arrowsize=3pt 2, ticksize=-2pt 2pt,Dx=0.5,Dy=20]{->}(0,0)(\xmin,\ymin)(\xmax,\ymax) 
\def\f{-8 x mul 32 add 2.7183 x exp mul} % définition de la fonction
\psplot[linecolor=blue,plotpoints=2000,linewidth=1.25pt]{0.2}{4}{\f}
%%%
\psline[linecolor=red, linestyle=dashed](0,160)(4.5,160) \uput*[l](0,160){\red 160}
\psline[linecolor=red, linestyle=dashed](0,100)(4.5,100) \uput*[l](0,100){\red 100}
\psline[linecolor=red,linewidth=1.6pt]{<->}(1.69,0)(3.69,0)
\uput[u](2.7,0){\red \bf 2 dixièmes}
\psline[linecolor=red, linestyle=dashed](1.69,100)(1.69,0)
\psline[linecolor=red, linestyle=dashed](3.69,100)(3.69,0)
\uput{8pt}[d](1.69,0){\red $1,7$} \uput{8pt}[d](3.69,0){\red $3,7$} 
\end{pspicture*}
\end{center}

\medskip

\textbf{Partie B} : Modélisation par une fonction

\medskip

On suppose que la courbe est la courbe représentative de la fonction $f$ définie sur l'intervalle $[0,2~;~4]$ par :
$f(x)=(-8x+32)\e^{x}.$

On note $f'$ la fonction dérivée de $f$. 

On admet que pour tout réel $x$ de l'intervalle $[0,2~;~4]$,
$f'(x)=(-8x+24)\e^{x}.$

\begin{enumerate}
\item% Étudier le signe de $f'(x)$  puis en déduire les variations de $f$ sur $[0,2~;~4]$.
Pour tout réel $x$, $\e^{x}>0$ donc $f'(x)$ est du signe de $(-8x+24)$ qui s'annule et change de signe pour $x=3$.

\begin{center}
{\renewcommand{\arraystretch}{1.3}
\psset{nodesep=3pt,arrowsize=2pt 3}  % paramètres
\def\esp{\hspace*{2.5cm}}% pour modifier la largeur du tableau
\def\hauteur{0pt}% mettre au moins 20pt pour augmenter la hauteur
$\begin{array}{|c| *4{c} c|}
\hline
 x & 0,2  & \esp & 3 & \esp & 4 \\
  \hline
-8x+24 &  &  \pmb{+} & \vline\hspace{-2.7pt}0 & \pmb{-} & \\  
 \hline
f'(x) &  &  \pmb{+} & \vline\hspace{-2.7pt}0 & \pmb{-} & \\  
\hline
 && f \text{ est croissante} & \vline\hspace{-2.7pt}{\phantom 0} & f \text{ est décroissante} & \\
\hline
\end{array}$
}
\end{center}


\item %Déterminer la valeur exacte du maximum de la fonction $f$.
Le maximum de la fonction $f$ est $f(3)= (-8\times 3+32)\e^{3}=8\e^{3}$.

On suppose que le sportif améliore sa meilleure performance de 5\,\% tous les mois. On cherche combien de mois d'entrainement sont nécessaires pour qu'il dépasse les 200~W.

Ajouter 5\,\%, c'est multiplier par $1+\frac{5}{100} = 1,05$. il faut donc compter le nombre de fois qu'il faut multiplier $8\e^{3}$ par $1,05$ pour que le résultat dépasse $200$.

$8\e^{3} \times 1,05^4 \approx 195,3 < 200$ et $8\e^{3} \times 1,05^5 \approx 205,1 > 200$ 

Il faudra donc 5 mois d'entraînement pour que le sportif dépasse les 200~W.

\end{enumerate}

\vspace{0.5cm}


