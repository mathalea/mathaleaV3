
\textbf{Commun à tous les candidats}

\medskip


\emph{Cet exercice est un questionnaire à choix multiples.\\
 Pour chacune des questions suivantes, une seule des quatre réponses proposées est exacte.\\
 Une réponse exacte rapporte un point. Une réponse fausse, une réponse multiple ou l'absence de réponse à une question ne rapporte ni
n'enlève de point.\\
 Pour répondre, indiquer sur la copie le numéro de la question et la lettre de
la réponse choisie.\\ Aucune justification n'est demandée.}

\vspace{0.7cm}

Soit $f$ la fonction définie pour tout nombre réel $x$ de l'intervalle $]0~;~ +\infty[$ par :
\[ f(x) =\dfrac{\e^{2x}}{x}\]

On donne l'expression de la dérivée seconde $f''$ de $f$, définie sur l'intervalle $]0~;~ +\infty[$ par:

\[f''(x)=\dfrac {2 \e^{2x} (2x^2-2x+1)}{x^3}.\]


\begin{enumerate}
\item La fonction $f'$, dérivée de $f$, est définie sur l'intervalle $]0~;~+\infty[$ par :

\begin{tabularx}{\linewidth}{*{2}{X}}
\textbf{a.~~} $f'(x) = 2\e^{2x}$ 					&\textbf{b.~~} $f'(x)=\dfrac{\e^{2x}(x-1)}{x^2} $\\[0.35cm]
\textbf{c.~~}$ f'(x) = \dfrac{\e^{2x}(2x - 1)}{x^2}$& \textbf{d.~~} $f'(x)=\dfrac{\e^{2x}(1 + 2x)}{x^2} $.\\
\end{tabularx}

\item La fonction $f$ :

\begin{tabularx}{\linewidth}{*{2}{X}}
\textbf{a.~~} est décroissante sur $]0~;~+\infty[ $ &\textbf{b.~~} est monotone sur $]0~;~+\infty[$\\
\textbf{c.~~}admet un minimum en $\dfrac{1}{2}$		& \textbf{d.~~}admet un maximum en $\dfrac{1}{2}$.
\end{tabularx}
\item  La fonction $f$ admet pour limite en $+ \infty$ :

\begin{tabularx}{\linewidth}{*{4}{X}}
\textbf{a.~~} $+\infty $ &\textbf{b.~~} $0$&\textbf{c.~~}$1$& \textbf{d.~~} $\e^{2x}$.
\end{tabularx}
\item  La fonction $f$ :

\begin{tabularx}{\linewidth}{*{4}{X}}
\textbf{a.~~} est concave sur $]0~;~+\infty[$ &\textbf{b.~~} est convexe  $]0~;~+\infty[$\\
\textbf{c.~~} est concave sur $\left]0~;~\frac{1}{2}\right] $& \textbf{d.~~} est représentée par une courbe
admettant un point d'inflexion.\\
\end{tabularx}
 
\end{enumerate}

\vspace{0,5cm}

