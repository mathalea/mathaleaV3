\begin{enumerate}
    \item 
    \begin{enumerate}
        \item % Comparer les durées moyennes de course.
        La moyenne de course de la s\oe ur de Célia est $31$ min et $13$ secondes. 

        Je calcule la moyenne de course de Célia:

        $\text{Moyenne} = \dfrac{33+32+40+27+30+26+29}{7}~\text{min et}~\dfrac{12+4+25+11+38+1}{7}~\text{secondes}=31~\text{min et}~13~\text{secondes}$.

        Les deux s\oe urs ont donc une moyenne de course identique.


        \item La médiane de course de la s\o eur de Célia est $30$ min.
    
        Je détermine la médiane de course de Célia, pour cela je range les durées dans l’ordre croissant :

        $26$ min et $1$ secondes ; $27$ min et $11$ secondes ; $29$ min et $1$ seconde ; $30$ min ; $32$ min et $4$ secondes ; $33$ min et $12$ secondes ; $40$ min et $25$ secondes.

        La médiane partage la série statistique en deux séries de même effectif. C’est donc la $4^e$ valeur soit $30$ min.

        Les deux médianes sont donc identiques. 
        \item Cette réponse est vraie. L'étendue pour la série statistique de la s\oe ur de Célia est de $3$ min.
        
        Si elle avait couru en $28$ min minimum le maximum aurait été de $31$ min ce qui est contradictoire avec une moyenne de $31$ min $13$ s. 
        \item L'étendue pour Célia est $40$ min et $25$ secondes $- 26$ min et $38$ secondes = $13$ min $47$ s.
        
        La médiane et la moyenne sont les mêmes pour les deux s\oe urs avec une étendue beaucoup plus importante pour Célia.
        
        \medskip
        Donc sa s\oe ur a bien été plus régulière.
    \end{enumerate}
    \item   
    \begin{enumerate}
        \item On veut représenter le parcours à une échelle $\dfrac{1}{20~000}$. Donc sur le plan :
        
        $AB = \dfrac{1}{20~000}\times 2~300~\text{m}=0,115~\text{m}=11,5~\text{cm}$

        \medskip
        On obtient donc :

        \begin{tikzpicture}[x=1cm,y=1cm]            
            \draw (0,0) arc (180:0:4cm);
            \draw (0,0) -- (8,0);            
            \draw (-0.2,-0.2) node {$A$};            
            \draw (8.2,-0.2) node {$B$};
            \draw (4.1,-0.4) node {$11,5~\text{cm}$};            
        \end{tikzpicture}

        \item Le diamètre $D=2~300~\text{m}$ et le rayon $R=1~150~\text{m}$
        
        Dictance parcourue = $\pi\times R+D = \pi\times1~150+2~300\approx5~913$ m à l'unité près.
        
        \item La durée de course est de $33$ minutes et $36$ secondes, $33~\text{min et}~36~\text{s}=\dfrac{33}{60}+\dfrac{36}{3~600}~\text{h}=0,56~\text{h}$
        
        La distance parcourue est d'environ $5~913~\text{m}$ soit $5,913~\text{km}$

        La vitesse moyenne de course est donc : $v = \dfrac{5,913~\text{km}}{0,56~\text{h}\approx10,6~\text{km/h}}$ au dixième près.
        \item 
        \begin{spacing}{2}
            Le quart du parcours : $\dfrac{\pi\times 1~1150+2~300}{4}\approx1~478,21~\text{m}$.

            La moitié vaut donc $2~956,42~\text{m}$.

            Et les trois-quarts $4~434,62~\text{m}$.
        \end{spacing}

        La longueur du demi-cercle est de $\pi\times 1~150 \approx 3~612,83~\text{m}$.

        Je calcule l'angle au degré près pour le quart du parcours et la moitié du parcours en utilisant le tableau de proportionnalité.

        \medskip
        \noindent\begin{tabularx}{0.9\linewidth}{|>{\hsize=4cm}X|*{3}{>{\centering \arraybackslash}X|}}
            \hline 
            \textbf{Angle en \degree} & $180$ & $74$ & $147$ \\ 
            \hline 		
            \textbf{Distance parcourue} & $3~612,83$ & $1~478,21$ & $2~956,42$ \\
            \hline 	
        \end{tabularx}

        \medskip
        Pour les trois quarts du parcours, Célia fait le tour complet du demi-disque puis il reste à parcourir
        environ : $5~913 - 4~434,62 = 1~478,38~\text{m}$ soit à l’échelle $7,4$ cm.

        \begin{center}
            \begin{tikzpicture}[x=1cm,y=1cm]            
                \draw (0,0) arc (180:0:5.75cm);
                \draw (0,0) -- (11.5,0); 
                \draw (5.75,0.1) -- (5.75,-0.1);
                \draw (5.75,0) node (O) {$ $};                       
                \draw (5.75,-0.3) node {$O$};                       
                \draw (0,0) node (A) {$ $};            
                \draw (-0.2,-0.3) node {$A$};            
                \draw (11.7,-0.3) node {$B$};
                \draw (10.57,3.13) node (M) {$ $};
                \draw (10.8,3.4) node {$M$};
                \draw (5.75,0) -- (10.57,3.13);
                \draw (4.16,5.53) node (L) {$ $};
                \draw (4,5.8) node {$L$};
                \draw (5.75,0) -- (4.16,5.53);
                \draw[<->] (0,-1.2) -- (11.5,-1.2) node [midway, fill = white] {$11,5~\text{cm}$};
                \draw (4.1,0.1) -- (4.1,-0.1);
                \draw (4.1,-0.3) node {$N$};                            
                \draw[<->] (4.1,-0.8) -- (11.5,-0.8) node [midway, fill = white] {$7,4~\text{cm}$};
                \pic [draw,blue, -, "$74 \degree$", angle eccentricity=1.5] {angle = L--O--A};
                \pic [draw,red, -, "$147 \degree$", angle eccentricity=1.1, angle radius = 1.3cm,pic text options={shift={(-16pt,0pt)}}] {angle = M--O--A};
            \end{tikzpicture}
        \end{center}
        

    \end{enumerate}
\end{enumerate}