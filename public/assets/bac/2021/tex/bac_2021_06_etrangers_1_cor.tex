\textbf{\large\textsc{Exercice 1} \hfill 5 points}

\textbf{Commun à tous les candidats}

\medskip

%\emph{Ceci est un questionnaire à choix multiples (QCM).\\
% Pour chacune des questions, une seule des quatre affirmations est exacte.\\
%  Le candidat recopiera sur sa copie le numéro de la question et la réponse correspondante.\\
%Aucune justification n'est demandée.\\
%Une réponse exacte rapporte un point, une réponse fausse ou une absence de réponse ne rapporte ni n'enlève aucun point.}
%
%\medskip

\begin{enumerate}
\item On considère la fonction définie sur $\R$ par 
$f(x) = x\e^{-2x}.$

On note $f''$ la dérivée seconde de la fonction $f$.

Quel que soit le réel $x$, \,$f''(x)$ est égal à :
\begin{center}
\begin{tabularx}{\linewidth}{*{4}{X}}
\textbf{a.~~} $(1 - 2x)\e^{-2x}$& \textbf{b.~~} $4(x - 1)\e^{-2x}$&\textbf{c.~~} $4\e^{-2x}$&\textbf{d.~~}
$ (x+2)\e^{-2x}$
\end{tabularx}
\end{center}

\medskip

\begin{tabular}{@{\hspace*{0.05\linewidth}}|p{0.92\linewidth}}
$f(x) = x\e^{-2x}$ donc $f'(x)=\e^{-2x}+x\times (-2)\e^{-2x}= \left ( 1-2x\right )\e^{-2x}$
et donc

$f''(x)= -2\e^{-2x}+ (1-2x)\times (-2)\e^{-2x} = \left (-2-2+4x\right )\e^{-2x}
= 4\left (x-1\right )\e^{-2x}$

\smallskip

\textbf{Réponse b.} 
\end{tabular}

\bigskip

\item Un élève de première générale choisit trois spécialités parmi les douze proposées. \\
Le nombre de combinaisons possibles est:

\begin{center}
\begin{tabularx}{\linewidth}{*{4}{X}}
\textbf{a.~~} \np{1728}&\textbf{b.~~} \np{1320}&\textbf{c.~~} \np{220}&\textbf{d.~~} \np{33}
\end{tabularx}
\end{center}

\medskip

\begin{tabular}{@{\hspace*{0.05\linewidth}}|p{0.92\linewidth}}
$\ds\binom{12}{3} =\dfrac{12\,!}{3\,! (12-3)\,!}=\dfrac{12\times 11\times 10}{3\times 2 \times 1}=220$

\smallskip

\textbf{Réponse c.}
\end{tabular}

\bigskip

\item On donne ci-dessous la représentation graphique de $f'$ fonction dérivée d'une fonction $f$ définie sur [0~;~7].

\begin{center}
\psset{unit=1cm}
\begin{pspicture*}(-1,-4.5)(7.5,1.1)
\psgrid[gridlabels=0pt,gridwidth=0.3pt,subgriddiv=5,subgridwidth=0.1pt,gridcolor=gray](-1,-4.5)(7.5,1)
\psaxes[linewidth=1.25pt,labelFontSize=\scriptstyle](0,0)(-0.95,-4.5)(7.5,1)
\psplot[plotpoints=2000,linewidth=1.25pt,linecolor=blue]{0}{7}{2 x sub x 5 sub mul 0.4 mul 2.71828 x 0.25 mul exp div}
\end{pspicture*}
\end{center}

Le tableau de variation de $f$  sur l'intervalle [0~;~7] est :

\begin{center}
\begin{tabularx}{\linewidth}{*{2}{X}}
\textbf{a.~~}&\textbf{b.~~}\\
{%\renewcommand{\arraystretch}{1.3}
\psset{nodesep=3pt,arrowsize=2pt 3}%  paramètres
\def\esp{\hspace*{1cm}}% pour modifier la largeur du tableau
\def\hauteur{0pt}% mettre au moins 20pt pour augmenter la hauteur
$\begin{array}{|c|*5{c}|}
\hline
x & 0  & \esp & 3,25 & \esp & 7 \\ 
%\hline
%f'(x) &  &   \pmb{+} & \vline\hspace{-2.7pt}0 & \pmb{-} & \\ 
\hline
 & &  &   \Rnode{max}{\phantom{8}}  &  &   \\  
f(x) & &     &  &  &  \rule{0pt}{\hauteur} \\ 
 & \Rnode{min1}{\phantom{8}} &   &  &  &   \Rnode{min2}{\phantom{8}} \rule{0pt}{\hauteur}    
 \ncline{->}{min1}{max} 
 \ncline{->}{max}{min2} 
 \\ 
\hline
\end{array} $
}
&
{%\renewcommand{\arraystretch}{1.3}
\psset{nodesep=3pt,arrowsize=2pt 3}  % paramètres
\def\esp{\hspace*{0.5cm}}% pour modifier la largeur du tableau
\def\hauteur{0pt}% mettre au moins 20pt pour augmenter la hauteur
$\begin{array}{|c| *6{c} c|}
\hline
 x & 0 & \esp & 2 & \esp & 5 & \esp & 7 \\
% \hline
%f'(x) &  &  \pmb{-} & \vline\hspace{-2.7pt}0 & \pmb{+} & \vline\hspace{-2.7pt}0 & \pmb{-} & \\  
\hline
  &   \Rnode{max1}{\phantom{8}} & &  & & \Rnode{max2}{\phantom{8}} & & \\
f (x) & &  & & & & & \rule{0pt}{\hauteur} \\
 & & & \Rnode{min1}{\phantom{8}} & & & & \Rnode{min2}{\phantom{8}}  \rule{0pt}{\hauteur}
\ncline{->}{max1}{min1} 
\ncline{->}{min1}{max2}
\ncline{->}{max2}{min2} \\
\hline
\end{array}$
}
\\\\
\textbf{c.~~}&\textbf{d.~~}\\
{%\renewcommand{\arraystretch}{1.3}
\psset{nodesep=3pt,arrowsize=2pt 3}  % paramètres
\def\esp{\hspace*{0.5cm}}% pour modifier la largeur du tableau
\def\hauteur{0pt}% mettre au moins 20pt pour augmenter la hauteur
$\begin{array}{|c| *6{c} c|}
\hline
 x & 0 & \esp & 2 & \esp & 5 & \esp & 7 \\
% \hline
%f'(x) &  &  \pmb{+} & \vline\hspace{-2.7pt}0 & \pmb{-} & \vline\hspace{-2.7pt}0 & \pmb{+} & \\  
\hline
  &  &  & \Rnode{max1}{\phantom{8}} & & & &  \Rnode{max2}{\phantom{8}} \\
f (x) & &  & & & & & \rule{0pt}{\hauteur}\\
 & \Rnode{min1}{\phantom{8}} & & & & \Rnode{min2}{\phantom{8}} & & \rule{0pt}{\hauteur} 
\ncline{->}{min1}{max1} 
\ncline{->}{max1}{min2}
\ncline{->}{min2}{max2} \\
\hline
\end{array}$
}
&
{%\renewcommand{\arraystretch}{1.3}
\psset{nodesep=3pt,arrowsize=2pt 3}%  paramètres
\def\esp{\hspace*{1cm}}% pour modifier la largeur du tableau
\def\hauteur{0pt}% mettre au moins 20pt pour augmenter la hauteur
$\begin{array}{|c|*5{c}|}
\hline
x & 0  & \esp & 2 & \esp & 7 \\ 
%\hline
%f'(x) &  &   \pmb{+} & \vline\hspace{-2.7pt}0 & \pmb{-} & \\ 
\hline
 & &  &   \Rnode{max}{\phantom{81}}  &  &   \\  
f(x) & &     &  &  &  \rule{0pt}{\hauteur} \\ 
 & \Rnode{min1}{\phantom{8}} &   &  &  &   \Rnode{min2}{\phantom{8}} \rule{0pt}{\hauteur}    
 \ncline{->}{min1}{max} 
 \ncline{->}{max}{min2} 
 \\ 
\hline
\end{array} $
}
\\
\end{tabularx}
\end{center}

\medskip

\begin{tabular}{@{\hspace*{0.05\linewidth}}|p{0.92\linewidth}}
$f'$ est négative ou nulle sur $[0~,~2]$ donc la fonction $f$ est décroissante sur $[0~,~2]$.

$f'$ est positive  ou nulle sur $[2~,~5]$ donc la fonction $f$ est croissante sur $[2~,~5]$.

$f'$ est négative  ou nulle sur $[5~,~7]$ donc la fonction $f$ est décroissante sur $[5~,~7]$.

\smallskip

\textbf{Réponse b.}
\end{tabular}

\bigskip

\item Une entreprise fabrique des cartes à puces. Chaque puce peut présenter deux défauts notés A et B.
Une étude statistique montre que 2,8\,\% des puces ont le défaut A, 2,2\,\% des puces ont le défaut B et, heureusement, 95,4\,\% des puces n'ont aucun des deux défauts.

La probabilité qu'une puce prélevée au hasard ait les deux défauts est:

\begin{center}
\begin{tabularx}{\linewidth}{*{4}{X}}
\textbf{a.~~} 0,05&\textbf{b.~~} 0,004&\textbf{c.~~} 0,046&\textbf{d.~~} On ne peut pas le savoir
\end{tabularx}
\end{center}

\medskip

\begin{tabular}{@{\hspace*{0.05\linewidth}}|p{0.92\linewidth}}
On appelle $A$ l'événement \og la puce a le défaut A \fg{} et $B$ l'événement \og la puce a le défaut B \fg{}.

D'après le texte, on a: $P(A)=0,028$ et $P(B)=0,022$. 

On cherche la probabilité qu'une puce ait les deux défauts, c'est-à-dire $P(A\cap B)$.

On sait que  $95,4$\,\% des puces n'ont aucun des deux défauts donc il y a $100-95,4=4,6$\,\% des puces qui ont au moins un des deux défauts, donc $P(A\cup B)=0,046$.

Or $P(A\cup B)=P(A)+P(B)-P(A\cap B)$ donc 

$P(A\cap B) = P(A)+P(B) - P(A\cup B) = 0,028+0,022-0,046=0,004$

\smallskip

\textbf{Réponse b.}
\end{tabular}

\bigskip

\item On se donne une fonction $f$, supposée dérivable sur $\R$, et on note $f’$ sa fonction dérivée.

On donne ci-dessous le tableau de variation de $f$ :

\begin{center}
{%\renewcommand{\arraystretch}{1.3}
\psset{nodesep=3pt,arrowsize=2pt 3}%  paramètres
\def\esp{\hspace*{0.5cm}}% pour modifier la largeur du tableau
\def\hauteur{0pt}% mettre au moins 20pt pour augmenter la hauteur
$\begin{array}{|c|*5{c}|}
\hline
x & -\infty  & \esp & -1 & \esp & +\infty\\ 
\hline
%f'(x) &  &   \pmb{+} & \vline\hspace{-2.7pt}0 & \pmb{-} & \\ 
%\hline
 & &  &   \Rnode{max}{0}  &  &   \\  
f(x) & &     &  &  &  \rule{0pt}{\hauteur} \\ 
 & \Rnode{min1}{-\infty} &   &  &  &   \Rnode{min2}{-\infty} \rule{0pt}{\hauteur}    
 \ncline{->}{min1}{max} 
 \ncline{->}{max}{min2} 
 \\ 
\hline
\end{array} $
}
\end{center}

D'après ce tableau de variation :

\textbf{a.~~} $f’$ est positive sur $\R$.
\hfill
\textbf{b.~~} $f’$ est positive sur $] - \infty~;~ - 1]$. 

\textbf{c.~~} $f’$ est négative sur $\R$.
\hfill
\textbf{d.~~} $f’$ est positive sur $[- 1~;~ +\infty[$.

\medskip

\begin{tabular}{@{\hspace*{0.05\linewidth}}|p{0.92\linewidth}}
La fonction $f$ est croissante sur $] - \infty~;~ - 1]$ donc $f’$ est positive sur $] - \infty~;~ - 1]$. 

\smallskip

\textbf{Réponse b.}
\end{tabular}

\bigskip

\end{enumerate}



\bigskip

