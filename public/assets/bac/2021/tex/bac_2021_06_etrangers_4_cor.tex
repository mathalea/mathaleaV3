
\medskip

Dans un repère orthonormé de l'espace, on considère les points suivants : \hfill{}$\text{A}(2~;~-1~;~0)$, $\text{B}(3~;~-1~;~2)$, $\text{C}(0~;~4~;~1)$ et $\text{S} (0~;~1~;~4)$.\hfill{}

\smallskip

\begin{enumerate}
\item %Montrer que le triangle ABC est rectangle en A.
$\vectt{AB}\,:\,
\begin{pmatrix}
3-2 \\-1-(-1)\\2-0
\end{pmatrix}
=
\begin{pmatrix}
1\\0\\2
\end{pmatrix}$
et
$\vectt{AC}\,:\,
\begin{pmatrix}
0-2 \\4-(-1)\\1-0
\end{pmatrix}
=
\begin{pmatrix}
-2 \\5\\1
\end{pmatrix}
$
donc

$\vectt{AB}\cdot \vectt{AC} = 1\times (-2) + 0\times 5 + 1 \times 2 = 0$
donc
$\vectt{AB}\perp \vectt{AC}$.

Le triangle ABC est donc rectangle en A.

\item  
	\begin{enumerate}
		\item Soit le vecteur $\vect{n}\begin{pmatrix}2\\1\\- 1\end{pmatrix}$.

Les vecteurs $\vectt{AB}$ et $\vectt{AC}$ ne sont pas colinéaires donc ce sont deux vecteurs directeurs du plan (ABC).
		
\begin{list}{\textbullet}{}
\item $\vect{n}\cdot \vectt{AB} = 2\times 1 + 1\times 0 + (-1)\times 2 = 0$ donc $\vect{n}\perp \vectt{AB}$ ;
\item $\vect{n}\cdot \vectt{AC} = 2\times (-2) + 1\times 5 + (-1)\times 1 = -4+5-1 = 0$ donc $\vect{n}\perp \vectt{AC}$.
\end{list}		
		
Le vecteur $\vect{n}$ est orthogonal aux deux vecteurs $\vectt{AB}$ et $\vectt{AC}$, donc il est orthogonal au plan (ABC).
				
		\item %En déduire une équation cartésienne du plan (ABC).
Le vecteur $\vect n$ est un vecteur normal au plan (ABC) donc le plan (ABC) a une équation cartésienne de la forme: $2x + 1y + (-1)z + d=0$ soit $2x+y-z=d=0$ où $d\in\R$.

$\text A \in \text{(ABC)}$ donc 	$2x_{\text A}+y_{\text A}-z_{\text A}+d=0$, c'est-à-dire $4 -1 +0 +d=0$, donc $d=-3$.
		
Le plan (ABC) a pour équation: $2x+y-z-3=0$.
		
		\item %Montrer que les points A, B, C et S ne sont pas coplanaires.
$2x_{\text S}+y_{\text S}-z_{\text S} - 3 = 0 + 1 -4 -3 = -6 \neq 0$ donc $\text S \not\in \text{(ABC)}$ 	donc  les points A, B, C et S ne sont pas coplanaires.	
		
	\end{enumerate}
	
\item  Soit $(d)$ la droite orthogonale au plan (ABC) passant par S. Elle coupe le plan (ABC) en H.
	\begin{enumerate}
		\item %Déterminer une représentation paramétrique de la droite $(d)$.
La droite $(d)$ est orthogonale au plan (ABC) donc elle a pour vecteur directeur le vecteur $\vect n$. De plus elle contient le point S\,$(0~;~1~;~4)$. \\
Donc elle a pour représentation paramétrique:

$\left \lbrace
\begin{array}{l !{=} l !{+} l l}
x & 0 & 2\times t\\
y & 1& 1\times t & t\in\R\\
z & 4 &(-1)\times t
\end{array}
\right .$
soit 
$\left \lbrace
\begin{array}{l !{=} r l}
x & 2t\\
y & 1+t & t\in\R\\
z & 4 - t
\end{array}
\right .$		
		
		\item %Montrer que les coordonnées du point H sont H(2~;~2~;~3).
Le point H est l'intersection de la droite $(d)$ et du plan (ABC), donc ses coordonnées vérifient le système:

$\left \lbrace
\begin{array}{l !{=} r}
x & 2t\\
y & 1+t \\
z & 4 - t\\
2x+y-z-3 & 0
\end{array}
\right .$			
		
Donc: $2\left (2t\right ) + \left (1+t\right ) -\left (4-t\right )-3=0$, c'est-à-dire
$4t +1+t - 4 +t-3 =0$ soit $t=1$.

Pour $t=1$, on aura $x=2\times 1 = 2$,	$y=1+1=2$ et $z=4-1=3$.

Les coordonnées du point H sont donc $(2~;~2~;~3)$.
		
	\end{enumerate}
\item  On rappelle que le volume $\mathcal V$ d'un tétraèdre est $\mathcal V = \dfrac{\small \text{aire de la base} \times \text{hauteur}}{3}$.

\begin{list}{\textbullet}{}
\item La base est le triangle ABC rectangle en A dont l'aire vaut 
$\mathcal{A} = \dfrac{\text{AB}\times \text{AC}}{2}$.

$\text{AB}^2 = 1^2+ 0^2 + 2^2 = 5$ donc $\text{AB} = \sqrt{5}$

$\text{AC}^2 = (-2)^2+ 5^2 + 1^2 = 30$ donc $\text{AC}=\sqrt{30}$

$\mathcal{A} = \dfrac{\sqrt{5}\times \sqrt{30}}{2}=\dfrac{\sqrt{150}}{2}=\dfrac{5\sqrt{6}}{2}$

\item La hauteur est SH : $\text{SH}^2 = (2-0)^2+(2-1)^2+3-4)^2=6$ donc $\text{SH}=\sqrt{6}$

\item $\mathcal{V} = \dfrac{1}{3} \times \mathcal{A}\times \text{SH} = \dfrac{1}{3}\times \dfrac{5\sqrt{6}}{2}\times \sqrt{6}=5$
\end{list}
% Calculer le volume du tétraèdre SABC.

\item  
	\begin{enumerate}
		\item %Calculer la longueur SA.
		
		$\vectt{SA}\,:\,
\begin{pmatrix}
2-0 \\-1-1\\0-4
\end{pmatrix}
=
\begin{pmatrix}
2 \\-2\\4
\end{pmatrix}
$
donc
$\text{SA}^2 = 2^2+(-2)^2+(-4)^2=24$ donc $\text{SA}=\sqrt{24}=2\sqrt{6}$
		
		\item On indique que SB $= \sqrt{17}$. $\vectt{SB}\,:\,
\begin{pmatrix}
3-0 \\1-(-1)\\4-2
\end{pmatrix}
=
\begin{pmatrix}
3 \\2\\2
\end{pmatrix}$
donc
$\vectt{SA}\cdot \vectt{SB} = 2\times 2 + (-2)\times (-2) + (-4)\times (-2)=18$

Or $\vectt{SA}\cdot \vectt{SB} = \text{SA}\times \text{SB}\times \cos \left ( \widehat{\text{ASB}}\right )$

Donc $18 = 2\sqrt{6} \times \sqrt{17} \times \cos \left ( \widehat{\text{ASB}}\right )$
et donc
$\cos \left ( \widehat{\text{ASB}}\right ) = \dfrac{18}{2\sqrt{6} \times \sqrt{17}}
= \dfrac{9}{\sqrt{102}}$

On en déduit que $\widehat{\text{ASB}} \approx 27,0\degres$.

	\end{enumerate}
\end{enumerate}

\bigskip

