
\medskip

Une entreprise qui fabrique des aiguilles dispose de deux sites de production, le site A et le site B.

Le site A produit les trois-quarts des aiguilles, le site B l'autre quart. 

Certaines aiguilles peuvent présenter un défaut. Une étude de contrôle de qualité a révélé que :

\setlength\parindent{9mm}
\begin{itemize}
\item[$\bullet~~$] 2\,\% des aiguilles du site A sont défectueuses ;
\item[$\bullet~~$] 4\,\% des aiguilles du site B sont défectueuses.
\end{itemize}
\setlength\parindent{0mm}
\medskip

Les aiguilles provenant des deux sites sont mélangées et vendues ensemble par lots. 

On choisit une aiguille au hasard dans un lot et on considère les évènements suivants:

\setlength\parindent{9mm}
\begin{itemize}
\item[$\bullet~~$]$A$ : l'aiguille provient du site A ; 
\item[$\bullet~~$]$B$ : l'aiguille provient du site B ;
\item[$\bullet~~$]$D$ : l'aiguille présente un défaut.
\end{itemize}
\setlength\parindent{0mm}

L'évènement contraire de $D$ est noté $\overline{D}$.

\medskip

\begin{enumerate}
\item D'après les données de l'énoncé, donner la valeur de la
probabilité de l'évènement $A$ que l'on notera $P(A)$.
\item Recopier et compléter sur la copie l'arbre de probabilités ci-
dessous en indiquant les probabilités sur les branches.
\item Quelle est la probabilité que l'aiguille ait un défaut et provienne du site A ?
\item Montrer que $P(D) =0,025$.
\item Après inspection, l'aiguille choisie se révèle défectueuse. 

Quelle est la probabilité qu'elle ait été produite sur le site A ?
\end{enumerate}

\begin{center}
\pstree[treemode=R,nodesepB=3pt,levelsep=2.75cm]{\TR{}}
{\pstree{\TR{$A$~~}}
	{\TR{$D$}
	\TR{$\overline{D}$}
	}
\pstree{\TR{$\overline{A}$~~}}
	{\TR{$D$}
	\TR{$\overline{D}$}	
	}
}
\end{center}
