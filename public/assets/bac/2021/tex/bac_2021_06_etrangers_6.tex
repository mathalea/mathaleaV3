
\textbf{Commun à tous les candidats}

\medskip

Cet exercice est un questionnaire à choix multiples. 

Pour chacune des cinq questions, quatre
réponses sont proposées; une seule de ces réponses est exacte.

\smallskip

\textbf{Indiquer sur la copie le numéro de la question et recopier la réponse exacte sans justifier le choix effectué.}

\textbf{Barème :} une bonne réponse rapporte un point. Une réponse inexacte ou une absence de réponse n'apporte ni n'enlève aucun point.

\bigskip

\textbf{Question 1 :}

On considère la fonction $g$ définie sur $]0~;~+\infty[$  par $g(x)= x^2+ 2x - \dfrac{3}{x}$.

Une équation de la tangente à la courbe représentative de $g$ au point d'abscisse 1 est: 
\begin{center}
\begin{tabularx}{\linewidth}{|*{4}{X|}}\hline
\textbf{a.~~} $y=7(x - 1)$&\textbf{b.~~} $y = x - 1$&\textbf{c.~~} $y = 7x + 7$&\textbf{d.~~}$ y = x +1$\\ \hline
\end{tabularx}
\end{center}

\medskip

\textbf{Question 2 :}

On considère la suite $\left(v_n\right)$ définie sur $\N$ par $v_n = \dfrac{3n}{n + 2}$. On cherche à déterminer la limite de $v_n$ lorsque $n$ tend vers $+\infty$.

\begin{center}
\begin{tabularx}{\linewidth}{|*{4}{X|}}\hline
\textbf{a.~~}$\displaystyle\lim_{n \to + \infty}v_n = 1$&
\textbf{b.~~} $\displaystyle\lim_{n \to + \infty}v_n = 3$ &  \textbf{c.~~} $\displaystyle\lim_{n \to + \infty}v_n = \dfrac{3}{2}$ &\textbf{d.~~} On ne peut pas la déterminer\\ \hline
\end{tabularx}
\end{center}

\medskip

\textbf{Question 3 :}
Dans une urne il y a 6 boules noires et 4 boules rouges. On effectue successivement 10 tirages aléatoires avec remise. Quelle est la probabilité (à $10^{-4}$ près) d'avoir 4 boules noires et 6 boules rouges?

\begin{center}
\begin{tabularx}{\linewidth}{|*{4}{X|}}\hline
\textbf{a.~~}\np{0,1662}&\textbf{b.~~} 0,4&\textbf{c.~~} \np{0,1115}&\textbf{d.~~} \np{0,8886}\\ \hline
\end{tabularx}
\end{center}

\medskip

\textbf{Question 4 :}

On considère la fonction $f$ définie sur $\R$ par $f(x) = 3\e^x - x$.

\begin{center}
\begin{tabularx}{\linewidth}{|*{4}{>{\small}X|}}\hline
\textbf{a.~~}$\displaystyle\lim_{x \to + \infty} f(x) = 3$
&\textbf{b.~~}$\displaystyle\lim_{x \to + \infty} f(x) = +\infty $&\textbf{c.~~} $\displaystyle\lim_{x \to + \infty} f(x) =  -\infty$&\textbf{d.~~} On ne peut pas déterminer la limite de la fonction $f$ lorsque $x$ tend vers $+\infty$\\ \hline
\end{tabularx}
\end{center}

\medskip

\textbf{Question 5 :}

Un code inconnu est constitué de 8 signes. 

Chaque signe peut être une lettre ou un chiffre. Il y a
donc 36 signes utilisables pour chacune des positions.

Un logiciel de cassage de code teste environ cent millions de codes par seconde. En combien de temps au maximum le logiciel peut-il découvrir le code ?

\begin{center}
\begin{tabularx}{\linewidth}{|*{4}{X|}}\hline
\textbf{a.~~}environ 0,3 seconde&\textbf{b.~~}environ 8 heures&\textbf{c.~~}environ 3 heures&
\textbf{d.~~}environ 470 heures\\ \hline
\end{tabularx}
\end{center}

\bigskip

