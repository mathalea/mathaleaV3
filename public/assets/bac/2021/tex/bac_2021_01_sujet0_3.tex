
\medskip

Pour préparer l’examen du permis de conduire, on distingue deux types de formation :

\begin{itemize}
\item la formation avec \emph{conduite accompagnée} ;
\item la formation \emph{traditionnelle}.
\end{itemize}

On considère un groupe de 300 personnes venant de réussir l’examen du permis de conduire. Dans ce groupe :

\begin{itemize}
\item 75 personnes ont suivi une formation avec \emph{conduite accompagnée} ; parmi elles, 50 ont réussi l’examen à leur première présentation et les autres ont réussi à leur deuxième présentation.
\item  225 personnes se sont présentées à l’examen suite à une formation \emph{traditionnelle} ; parmi elles, 100 ont réussi l’examen à la première présentation, 75 à la deuxième et 50 à la troisième présentation.
\end{itemize}

On interroge au hasard une personne du groupe considéré.

On considère les évènements suivants :
\begin{itemize}
\item [] $A$ : \og la personne a suivi une formation avec \emph{conduite accompagnée} \fg{} ;
\item [] $R_1$ : \og la personne a réussi l’examen à la première présentation \fg{} ;
\item []$R_2$ : \og la personne a réussi l’examen à la deuxième présentation \fg{} ;
\item []$R_3$ : \og la personne a réussi l’examen à la troisième présentation \fg.
\end{itemize}

\medskip

\begin{enumerate}
\item  Modéliser la situation par un arbre pondéré.

\emph{Dans les questions suivantes, les probabilités demandées seront données sous forme d’une fraction irréductible.}

\item 
	\begin{enumerate}
		\item Calculer la probabilité que la personne interrogée ait suivi une formation avec \emph{conduite accompagnée} et réussi l’examen à sa deuxième présentation.
		\item Montrer que la probabilité que la personne interrogée ait réussi l’examen à sa deuxième présentation est égale à $\frac{1}{3}$.
		\item La personne interrogée a réussi l’examen à sa deuxième présentation. Quelle est la probabilité qu’elle ait suivi une formation avec \emph{conduite accompagnée}?
	\end{enumerate}
\item On note $X$ la variable aléatoire qui, à toute personne choisie au hasard dans le groupe, associe le nombre de fois où elle s’est présentée à l’examen jusqu’à sa réussite.

Ainsi, ${X=1}$ correspond à l’évènement $R_1$.
	\begin{enumerate}
		\item  Déterminer la loi de probabilité de la variable aléatoire $X$.
		\item Calculer l’espérance de cette variable aléatoire. Interpréter cette valeur dans le contexte de l’exercice.
	\end{enumerate}
\item On choisit, successivement et de façon indépendante, $n$ personnes parmi les 300 du groupe étudié, où $n$ est un entier naturel non nul. On assimile ce choix à un tirage avec remise de $n$ personnes parmi les 300 personnes du groupe.

On admet que la probabilité de l’évènement $R_3$ est égale à $\frac{1}{6}$.
	\begin{enumerate}
		\item  Dans le contexte de cette question, préciser un évènement dont la probabilité est égale à $1-\left(\dfrac{5}{6}\right)^n$.

On considère la fonction Python \textbf{seuil} ci-dessous, où $p$ est un nombre réel appartenant à l’intervalle ]0;1[.
\begin{center}
\begin{tabular}[]{|l|}
\hline
\textbf{def seuil}(p):\\
\hspace{2em}n = 1\\
\hspace{2em}\textbf{while} 1$-$(5/6)**n $< =$ p:\\
\hspace{4.5em}n = n+1\\
\hspace{2em}\textbf{return} n\\
\hline
\end{tabular}
\end{center}

		\item Quelle est la valeur renvoyée par la commande \textbf{seuil}(0,9) ? Interpréter cette valeur dans le contexte de l’exercice.
	\end{enumerate}
\end{enumerate}

\bigskip


