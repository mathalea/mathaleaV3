\textbf{\large Exercice 2 \hfill Commun à  tous les candidats \hfill 6 points}

\medskip

On considère les suites $\left(u_n\right)$ et $\left(v_n\right)$ définies pour tout entier naturel $n$ par : 

\[\left\{\begin{array}{l !{=} l}
u_0		& v_0 = 1\\
u_{n+1}	& u_n +v_n\\
v_{n+1}	& 2u_n + v_n
\end{array}\right.\]

Dans toute la suite de l'exercice, on \textbf{admet} que les suites $\left(u_n\right)$ et $\left(v_n\right)$ \textbf{sont strictement positives}.

\medskip

\begin{enumerate}
\item 
	\begin{enumerate}
		\item $u_1=u_0+v_0=1+1=2$ et $v_1=2\times u_0+v_0 = 2\times 1+1=3$.
		
		\item %Démontrer que la suite $\left(v_n\right)$ est strictement croissante, puis en déduire que, pour tout entier naturel $n$,\: $v_n \geqslant 1$.
Pour tout $n$, $v_{n+1}= 2u_n+v_n$ donc $v_{n+1}-v_n=2u_n$.

On a admis que la suite $(u_n)$ était strictement positive donc, pour tout $n$, $u_n>0$; on en déduit que, pour tout $n$, $v_{n+1}-v_n>0$ donc que la suite $(v_n)$ est strictement croissante.

La suite $(v_n)$ est strictement croissante	 donc, pour tout $n$, $v_n\geqslant v_0$ donc $v_n\geqslant 1$.
		
		\item Soit $\mathcal P_n$ la propriété: $u_n \geqslant  n + 1$.
		
\begin{list}{\textbullet}{On démontre cette propriété par récurrence.}
\item \textbf{Initialisation}

Pour $n=0$, $u_n=u_0=1$ et $n+1=1$ donc $u_n\geqslant n+1$; $\mathcal{P}_0$ est vraie.

\item \textbf{Hérédité}

On suppose que $\mathcal{P}_n$ est vraie (hypothèse de récurrence) et on va démontrer que $\mathcal{P}_{n+1}$ est vraie.

$\mathcal{P}_n$ vraie équivaut à  $u_n\geqslant n+1$.

$u_{n+1}=u_n+v_n$; or $u_n\geqslant n+1$ et, d'après la question 1.b, $v_n\geqslant 1$. On en déduit que $u_{n+1} \geqslant n+2$ et donc que la propriété est vraie au rang $n+1$.

\item \textbf{Conclusion}

La propriété $\mathcal{P}_n$ est vraie au rang 0, et elle est héréditaire pour tout $n\geqslant 0$; d'après le principe de récurrence, la propriété est vraie pour tout $n\geqslant 0$.
\end{list}		
		
On a donc démontré que, pour tout entier naturel $n$, $u_n\geqslant n+1$.		
		
		\item Pour tout $n$, $u_n\geqslant n+1$; or $\ds\lim_{n\to +\infty} n+1=+\infty$, donc par comparaison, $\ds\lim_{n\to +\infty} u_n=+\infty$.
%		En déduire la limite de la suite $\left(u_n\right)$.
	\end{enumerate}
	\item On pose, pour tout entier naturel $n$: $r_n = \dfrac{v_n}{u_n}.$
On admet que: $r_n^2 = 2 + \dfrac{(- 1)^{n+1}}{u_n^2}$.

	\begin{enumerate}
		\item %Démontrer que pour tout entier naturel $n$ :
$(-1)^{n+1}$ vaut soit $-1$, soit $1$ selon la parité de $n$; donc $-1\leqslant (-1)^{n+1} \leqslant 1$.

On sait que $u_n>0$ donc $u_n^2>0$.

On divise par $u_n^2$ et on obtient:
$- \dfrac{1}{u_n^2} \leqslant \dfrac{(- 1)^{n+1}}{u_n^2} \leqslant \dfrac{1}{u_n^2}.$

		\item% En déduire :
On sait que $\ds\lim_{n\to +\infty} u_n=+\infty$ donc $\ds\lim_{n\to +\infty} u_n^2=+\infty$. 

On en déduit que $\ds\lim_{n\to +\infty} -\dfrac{1}{u_n^2} = \ds\lim_{n\to +\infty} \dfrac{1}{u_n^2} = 0$.

On sait de plus que, pour tout $n$:  $- \dfrac{1}{u_n^2} \leqslant \dfrac{(- 1)^{n+1}}{u_n^2} \leqslant \dfrac{1}{u_n^2}$.
		
Donc, d'après le théorème des gendarmes, on peut dire que:
$\displaystyle\lim_{n \to + \infty} \dfrac{(- 1)^{n+1}}{u_n^2}=0$.

		\item %Déterminer la limite de la suite $\left(r_n^2\right)$ et en déduire que $\left(r_n\right)$ converge vers $\sqrt{2}$.
$r_n^2 = 2 + \dfrac{(- 1)^{n+1}}{u_n^2}$ et $\displaystyle\lim_{n \to + \infty} \dfrac{(- 1)^{n+1}}{u_n^2}=0$, donc: $\ds\lim_{n \to +\infty} r_n^2=2$.

On peut en déduire que la suite $(r_n)$ converge vers $\sqrt{2}$
		
		\item  Pour tout entier naturel $n$,
		
$r_{n+1}=\dfrac{v_{n+1}}{u_{n+1}} = \dfrac{2u_n + v_n}{u_n+v_n}
=\dfrac{u_n\left (2+\dfrac{v_n}{u_n}\right )}{u_n \left ( 1+\dfrac{v_n}{u_n}\right )}		
=\dfrac{2+\dfrac{v_n}{u_n}}{ 1+\dfrac{v_n}{u_n}}		
 = \dfrac{2 + r_n}{1 + r_n}$.

		\item On considère le programme suivant écrit en langage Python :
		
\begin{center}
\fbox{
\begin{tabularx}{0.5\linewidth}{X}
\textbf{def seuil()}:\\
\qquad  n = 0\\
\qquad  r = 1\\
\qquad \textbf{while} abs(r-sqrt(2)) > 10**(-4) :\\
\quad \qquad r = (2+r)/(1+r)\\
\quad \qquad n = n+1\\
\qquad \textbf{return} n\\ 
\end{tabularx}
}
\end{center}

%\smallskip
%
%(abs désigne la valeur absolue, sqrt la racine carrée et 18** (-4) représente $10^{-4}$).

La valeur de $n$ renvoyée par ce programme est 5.

% à€ quoi correspond-elle ?

Elle correspond à  la plus petite valeur de $n$ pour laquelle la distance entre $r_n$ et $\sqrt{2}$ est inférieure ou égale à  $10^{-4}$.

	\end{enumerate}
\end{enumerate}

\bigskip

