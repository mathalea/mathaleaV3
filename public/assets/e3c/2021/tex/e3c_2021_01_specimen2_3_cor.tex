\textbf{\large\textsc{Exercice 3 \hfill 5 points}}

\bigskip

\begin{enumerate}
\item On lance deux dés cubiques équilibrés \og classiques \fg{} et on note les numéros apparaissant sur la face supérieure de chaque dé.

Soit $X$ la variable aléatoire égale au produit des numéros apparaissant sur les deux faces.

Le jeu est gagné si le produit des numéros apparaissant sur les faces supérieures des deux
dés lancés est strictement inférieur à 10.

\begin{enumerate}
\item On détermine tous les résultats possibles:

\begin{center}
{\renewcommand{\arraystretch}{1.5}
\begin{tabular}{|*7{>{\centering\arraybackslash}p{0.3cm}|}}
\cline{2-7}
\multicolumn{1}{c|}{$\times$} & 1 & 2 & 3 & 4 & 5 & 6\\
 \hline
 1 & $\blue 1$ & $\blue 2$ & $\blue 3$ & $\blue 4$ & $\blue 5$ & $\blue 6$\\
 \hline
 2 & $\blue 2$ & $\blue 4$ & $\blue 6$ & $\blue 8$ & $\red 10$ & $\red 12$\\
 \hline 
3 & $\blue 3$ & $\blue 6$ & $\blue 9$ & $\red 12$ & $\red 15$ & $\red 18$\\
 \hline
 4  & $\blue 4$ & $\blue 8$ & $\red 12$ & $\red 16$ & $\red 20$ & $\red 24$\\
 \hline
 5 & $\blue 5$ & $\red 10$ & $\red 15$ & $\red 20$ & $\red 25$ & $\red 30$\\
 \hline 
 6 & $\blue 6$ & $\red 12$ & $\red 18$ & $\red 24$ & $\red 30$ & $\red 36$\\
 \hline   
\end{tabular}
}
\end{center}

 La variable aléatoire $X$ prend donc les valeurs 
1, 2, 3, 4, 5, 6, 8, 9, 10, 12, 15, 16, 18, 20, 24, 25, 30 et 36.

\item Dans le tableau, il y a 36 résultats qui sont équiprobables; donc, par exemple, la probabilité d'obtenir 4 est $\dfrac{3}{36}$ car le 4 apparaît 3 fois dans le tableau.

On détermine alors la loi de probabilité de   $X$:

\begin{center}
{\renewcommand{\arraystretch}{1.5}
\begin{tabular}{|c|*{18}{>{\centering\arraybackslash}p{0.2cm}|}}
\hline
$x_i$ & 1 & 2 & 3 & 4 & 5 & 6 & 8 & 9 & 10 & 12 & 15 & 16 & 18 & 20 & 24 & 25 & 30 & 36 \\
\hline
$p_i$ & $\frac{1}{36}$  & $\frac{2}{36}$  & $\frac{2}{36}$  & $\frac{3}{36}$  & $\frac{2}{36}$  & $\frac{4}{36}$  & $\frac{2}{36}$  & $\frac{1}{36}$  & $\frac{2}{36}$  & $\frac{4}{36}$  & $\frac{2}{36}$  & $\frac{1}{36}$  & $\frac{2}{36}$  & $\frac{2}{36}$  & $\frac{2}{36}$  & $\frac{1}{36}$  & $\frac{2}{36}$  & $\frac{1}{36}$\\
\hline
\end{tabular}}
\end{center}

\item La probabilité de gagner est:

$P\left ( \left \lbrace 1~,~2~,~3~,~4~,~5~,~6~,~8~,~9\strut \right \rbrace\right )
= \dfrac{1}{36} + \dfrac{2}{36} + \dfrac{2}{36} + \dfrac{3}{36} + \dfrac{2}{36} + \dfrac{4}{36} + \dfrac{2}{36} + \dfrac{1}{36} = \dfrac{17}{36}$.

On peut dire aussi qu'il y a 17 cas favorables (en bleu) sur 36 cas possibles.

\end{enumerate}

\item On lance à présent deux dés spéciaux : ce sont des dés cubiques parfaitement équilibrés dont les faces sont numérotées différemment des dés classiques.

\begin{list}{\textbullet}{}
\item Les faces du premier dé sont numérotées avec les chiffres : 1, 2, 2, 3, 3, 4.
\item Les faces du deuxième dé sont numérotées avec les chiffres : 1, 3, 4, 5, 6, 8.
\end{list}

On note $Y$ la variable aléatoire égale au produit des numéros apparaissant sur les deux faces après lancer de ces deux dés spéciaux.

On détermine tous les résultats possibles:

\begin{center}
{\renewcommand{\arraystretch}{1.5}
\begin{tabular}{|*7{>{\centering\arraybackslash}p{0.3cm}|}}
\cline{2-7}
\multicolumn{1}{c|}{} & 1 & 2 & 2 & 3 & 3 & 4\\
 \hline
 1& $\blue 1$ & $\blue 2$ & $\blue 2$ & $\blue 3$ & $\blue 3$ & $\blue 4$\\
 \hline
3 & $\blue 3$ & $\blue 6$ & $\blue 6$ & $\blue 9$ & $\blue 9$ & $\red 12$\\
 \hline 
4 & $\blue 4$ & $\blue 8$ & $\blue 8$ & $\red 12$ & $\red 12$ & $\red 16$\\
 \hline
5  & $\blue 5$ & $\red 10$ & $\red 10$ & $\red 15$ & $\red 15$ & $\red 20$\\
 \hline
6 & $\blue 6$ & $\red 12$ & $\red 12$ & $\red 18$ & $\red 18$ & $\red 24$\\
 \hline 
8 & $\blue 8$ & $\red 16$ & $\red 16$ & $\red 24$ & $\red 24$ & $\red 32$\\
 \hline   
\end{tabular}
}
\end{center}

L'évènement $Y<10$ comporte 17 cas favorables (en bleu dans le tableau). Il y a équiprobabilité donc  $P(Y < 10)=\dfrac{17}{36}$.

\item Les deux probabilités de gagner étant égales, il est équivalent de jouer au jeu de la question 1 avec des dés classiques ou avec des dés spéciaux.
\end{enumerate}

\vspace{0.5cm}

