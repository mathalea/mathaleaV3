\textbf{\large Exercice 1 \hfill 5 points}

\bigskip

\emph{Ce questionnaire à choix multiples (QCM) comprend cinq questions indépendantes.\\
Pour chacune des questions, une seule des quatre réponses proposées est correcte.\\
Pour chaque question, indiquer sur la copie le numéro de la question et la lettre de la réponse choisie. Aucune justification n’est demandée.\\
Chaque réponse correcte rapporte $1$ point. Une réponse incorrecte ou l'absence de réponse ne rapporte ni ne retire de point.}

\bigskip

\textbf{Question 1}

\medskip

Soit $f$ la fonction définie sur $\R$ par $f(x) = x^2+x+1$.

Cette fonction est dérivable sur $\R$. Sa fonction dérivée $f'$ est donnée sur $\R$ par:

\begin{center}
{\renewcommand{\arraystretch}{1.5}
\begin{tabularx}{\linewidth}{XXXX}
\textbf{a.}\quad $f'(x)=x+1$ & \textbf{b.}\quad $f'(x)=2x+1$ 
& \textbf{c.}\quad $f'(x)=2x$   & \textbf{d.}\quad $f'(x)=2x^2+x$\\
\end{tabularx}}
\end{center}

\medskip

\textbf{Question 2}

\medskip

La somme $1+2+2^2 + 2^3 + \cdots + 2^{10}$ est égale à:

\begin{center}
{\renewcommand{\arraystretch}{1.5}
\begin{tabularx}{0.9\linewidth}{XXXX}
\textbf{a.}\quad $2^{10}-1$ & \textbf{b.}\quad $2^{10}$ 
& \textbf{c.}\quad $2^{11}-1$   & \textbf{d.}\quad $2^{11}$\\
\end{tabularx}}
\end{center}

\medskip

\textbf{Question 3}

\medskip

On considère l'équation $x^2+2x-8=0$.

On note $S$ la somme des racines de cette équation et $P$ leur produit.

Laquelle des affirmations suivantes est vraie ?

\begin{center}
{\renewcommand{\arraystretch}{1.5}
\begin{tabularx}{\linewidth}{*{4}{X}}
\textbf{a.}\quad $S=2$ et $P=-8$ & \textbf{b.}\quad $S=-2$ et $P=-8$ 
& \textbf{c.}\quad $S=-2$ et $P=8$   & \textbf{d.}\quad $S=2$ et $P=8$\\
\end{tabularx}}
\end{center}

\medskip

\textbf{Question 4}

\medskip

On désigne par $\mathcal C$ le cercle trigonométrique.

Soit $x$ un réel strictement positif et M le point de $\mathcal{C}$ associé au réel $x$.

\begin{center}
\psset{unit=1.5cm,}
\def\xmin {-1.2}   \def\xmax {1.2}
\def\ymin {-1.2}   \def\ymax {1.2}
\begin{pspicture}(\xmin,\ymin)(\xmax,\ymax)
%\psgrid[subgriddiv=1,  gridlabels=0, gridcolor=lightgray] 
\psaxes[ ticksize=0pt 0pt, labels=none](0,0)(\xmin,\ymin)(\xmax,\ymax) 
\uput[ul](0,0){O}
\pscircle(0,0){1} \uput[135](1;135){$\mathcal{C}$}
\psset{linecolor=blue}
{\blue
\psline(1;25)(1;205)
\psdots(1,0)(0,1)(1;25)(1;205)
\uput[ur](1,0){I}  \uput[25](1;25){M} 
\uput[ul](0,1){J}  \uput[205](1;205){M$'$} 
}
\end{pspicture}
\end{center}

Alors le point M$'$, symétrique de M par rapport à O, est associé au réel:

\begin{center}
{\renewcommand{\arraystretch}{1.5}
\begin{tabularx}{0.9\linewidth}{XXXX}
\textbf{a.}\quad $-x$ & \textbf{b.}\quad $\pi + x$ 
& \textbf{c.}\quad $\pi - x$   & \textbf{d.}\quad $-\pi - x$\\
\end{tabularx}}
\end{center}

\bigskip

\textbf{Question 4}

\medskip

Parmi les égalités suivantes, laquelle est vraie pour tout réel $x$?

\begin{center}
{\renewcommand{\arraystretch}{1.5}
\begin{tabularx}{0.8\linewidth}{XX}
\textbf{a.}\quad $\cos(x+2\pi)=\cos(x)$ & \textbf{b.}\quad $\sin(-x)=\sin(x)$ \\
\textbf{c.}\quad $\cos(-x)=-\cos(x)$   & \textbf{d.}\quad $\cos^2(x)+\sin^2(x)=2$\\
\end{tabularx}}
\end{center}

\vspace{0,5cm}

