
\textbf{Commun à tous les candidats}

\medskip

On considère la suite $\left(u_n\right)$ définie par $u_0 = \np{10000}$ et pour tout entier naturel $n$ : 

\[u_{n+1} = 0,95u_n + 200.\]

\begin{enumerate}
\item $\bullet~~u_1 = 0,95 \times u_0 + 200 = 0,95 \times \np{10000} + 200 = \np{9500} + 200 = \np{9700}$.%Calculer $u_1$ et vérifier que $u_2 = \np{9415}$. 

$\bullet~~u_2 = 0,95 \times u_1 + 200 = 0,95 \times \np{9700} + 200 = \np{9215} + 200 = \np{9415}$. 
\item 
	\begin{enumerate}
		\item On démontre  par récurrence, que pour tout entier naturel $n$ : $u_n > \np{4000}.$
		
\emph{Initialisation} : $u_0 = \np{10000} > \np{4000}$ : l'inégalité est vraie au rang $0$ ;

\emph{Hérédité} : supposons que pour $n \in \N$, on ait $u_n > \np{4000}$, alors par produit par $0,95 > 0$,\, on a $0,95u_n > 0,95 \times \np{4000}$, soit :

$0,95u_n > \np{3800}$, et en ajoutant 200 à chaque membre :

$0,95u_n  + 200 > \np{3800} + 200$, c'est-à-dire $u_{n+1} > \np{4000}$ : la relation est vraie au rang $n + 1$.

Conclusion : l'inégalité est vraie au rang $0$ et si elle est vraie au rang $n \in \N$, elle est vraie au rang $n + 1$ : d'après le principe de récurrence $u_n > \np{4000}$ quel que soit le naturel $n$.
		\item %On admet que la suite $\left(u_n\right)$ est décroissante. Justifier qu'elle converge.
On sait que si la suite est décroissante et minorée par \np{4000}, elle converge vers une limite $\ell$, avec $\ell \geqslant \np{4000}$.
	\end{enumerate}
\item  %Pour tout entier naturel $n$, on considère la suite $\left(v_n\right)$ définie par: $v_n = u_n - \np{4000}$.
	\begin{enumerate}
		\item %Calculer $v_0$.
Pour $n = 0$, on a $v_0 = u_0 -  \np{4000} =  \np{10000} -  \np{4000} =  \np{6000}$.
		\item %Démontrer que la suite $\left(v_n\right)$ est géométrique de raison égale à $0,95$.
Au choix :
		
\emph{Méthode} 1 : pour $n \in \N$, on a :

$v_{n+1} = u_{n+1} - \np{4000} = 0,95u_n + 200 - \np{4000} = 0,95u_n  - \np{3800} = 0,95\left(u_n - \dfrac{\np{3800}}{0,95}\right) = 0,95\left(u_n - \np{4000}\right) = 0,95v_n$.

L'égalité $v_{n+1} = 0,95v_n$ vraie quel que soit $n \in \N$ montre que la suite $\left(v_n\right)$ est géométrique de raison égale à $0,95$.
		
\emph{Méthode} 2 : pour $n \in \N$, on a vu que $u_n > \np{4000}$, donc $v_n = u_n - 4000 > 0$.

On peut donc calculer : $\dfrac{v_{n+1}}{v_n} = \dfrac{u_{n+1} - \np{4000}}{u_n - \np{4000}} = \dfrac{0,95u_n + 200 - \np{4000}}{u_n - \np{4000}} =$

$ \dfrac{0,95u_n - \np{3800}}{u_n - \np{4000}} = \dfrac{0,95u_n - \np{3800}}{u_n - \np{4000}} = \frac{0,95\left(u_n - \frac{\np{3800}}{0,95} \right)}{u_n - \np{4000}} = \dfrac{0,95\left(u_n - \np{4000} \right)}{u_n - \np{4000}} = 0,95$.

Cette égalité vraie pour tout naturel $n$, montre que la suite $\left(v_n\right)$ est géométrique de raison égale à $0,95$.
		\item %En déduire que pour tout entier naturel $n$ :

%\[u_n = \np{4000} + \np{6000} \times 0,95^n.\]
D'après le résultat précédent, on sait que que pour tout $n \in \N$, \, 

$v_n = v_0 \times 0,95^n = \np{6000} \times 0,95^n$.

Or $v_n = u_n - \np{4000} \iff u_n = v_n + \np{4000} = \np{6000} \times 0,95^n + \np{4000}$.
		\item %Quelle est la limite de la suite $\left(v_n\right)$ ? Justifier la réponse.
		Comme $0 < 0,95 < 1$, on sait que $\displaystyle\lim_{n \to + \infty} 0,95^n = 0$, donc $\displaystyle\lim_{n \to + \infty} \np{6000} \times 0,95^n = 0$, d'où $\displaystyle\lim_{n \to + \infty}u_n = \np{4000}$ (par somme de limites).
	\end{enumerate}
\item %En 2020, une espèce animale comptait \np{10000} individus. L'évolution observée les années précédentes conduit à estimer qu'à partir de l'année 2021, cette population baissera de 5\,\% chaque début d'année.

%Pour ralentir cette baisse, il a été décidé de réintroduire $200$ individus à la fin de chaque année, à partir de 2021.
%
%Une responsable d'une association soutenant cette stratégie affirme que : \og l'espèce ne devrait pas s'éteindre, mais malheureusement, nous n'empêcherons pas une disparition de plus de la moitié de la population \fg.
%
%Que pensez-vous de cette affirmation ? Justifier la réponse.
$u_n$ est égal au nombre d'individus de l'espèce animale au rang $n$ ; d'après le résultat précédent ce nombre va diminuer et se rapprocher de \np{4000} soit moins de la moitié de la population initiale : le responsable a raison.
\end{enumerate}

\bigskip

