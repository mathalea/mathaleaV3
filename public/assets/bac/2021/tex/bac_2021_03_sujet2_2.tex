
\medskip

On considère les suites $\left(u_n\right)$ et $\left(v_n\right)$ définies pour tout entier naturel $n$ par : 

\[\left\{\begin{array}{l !{=} l}
u_0		& v_0 = 1\\
u_{n+1}	&  u_n +v_n\\
v_{n+1}	&  2u_n + v_n
\end{array}\right.\]

Dans toute la suite de l'exercice, on \textbf{admet} que les suites $\left(u_n\right)$ et $\left(v_n\right)$ \textbf{sont strictement positives}.

\medskip

\begin{enumerate}
\item 
	\begin{enumerate}
		\item Calculez $u_1$ et $v_1$.
		\item Démontrer que la suite $\left(v_n\right)$ est strictement croissante, puis en déduire que, pour tout entier naturel $n$,\: $v_n \geqslant 1$.
		\item Démontrer par récurrence que, pour tout entier naturel $n$, on a : $u_n \geqslant  n + 1$.
		\item En déduire la limite de la suite $\left(u_n\right)$.
	\end{enumerate}
\item On pose, pour tout entier naturel $n$ :

\[r_n = \dfrac{v_n}{u_n}.\]

On admet que:

\[r_n^2 = 2 + \dfrac{(- 1)^{n+1}}{u_n^2}\]

	\begin{enumerate}
		\item Démontrer que pour tout entier naturel $n$ :

\[- \dfrac{1}{u_n^2} \leqslant \dfrac{(- 1)^{n+1}}{u_n^2} \leqslant \dfrac{1}{u_n^2}.\]
		\item En déduire :

\[\displaystyle\lim_{n \to + \infty} \dfrac{(- 1)^{n+1}}{u_n^2}.\]

		\item Déterminer la limite de la suite $\left(r_n^2\right)$ et en déduire que $\left(r_n\right)$ converge vers $\sqrt{2}$.
		\item  Démontrer que pour tout entier naturel $n$,

\[r_{n+1} = \dfrac{2 + r_n}{1 + r_n}.\]

		\item On considère le programme suivant écrit en langage Python :
		
\begin{center}
\fbox{
\begin{tabularx}{0.5\linewidth}{X}
\textbf{def seuil()}:\\
\qquad  n = 0\\
\qquad  r = 1\\
\qquad \textbf{while} abs(r-sqrt(2)) > 10**(-4) :\\
\quad \qquad r = (2+r)/(1+r)\\
\quad \qquad n = n+1\\
\qquad \textbf{return} n\\ 
\end{tabularx}
}
\end{center}

\smallskip

(abs désigne la valeur absolue, sqrt la racine carrée et 10** (-4) représente $10^{-4}$).

La valeur de $n$ renvoyée par ce programme est 5.

 À quoi correspond-elle ?
	\end{enumerate}
\end{enumerate}

\bigskip

