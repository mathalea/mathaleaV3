\textbf{\large Exercice 4 \hfill 5 points}

\bigskip

Dans un repère orthonormé \Oij{}, on considère les points A\,$(3~;~1)$, B\,$(-3~;~3)$ et C\,$(2~;~4)$.

\begin{enumerate}
\item %Montrer que l'équation $x+3y-6=0$ est une équation cartésienne de la droite (AB).
On regarde si les coordonnées de A et de B vérifient l'équation $x+3y-6=0$:

\begin{list}{\textbullet}{}
\item $x_{\text A}+3y_{\text A}-6= 3+3\times 1 - 6 = 0$ donc les coordonnées de A  vérifient l'équation.
\item $x_{\text B}+3y_{\text B}-6= -3+3\times 3 - 6 = 0$ donc les coordonnées de B  vérifient l'équation.
\end{list}

Donc  l'équation $x+3y-6=0$ est une équation cartésienne de la droite (AB).

\item Soit $d$ la droite perpendiculaire à la droite (AB) et passant par le point C.

On sait qu'une droite d'équation $ax+by+c=0$ a pour vecteur directeur le vecteur de coordonnées $(-b~;~a)$. Donc la droite (AB) a le vecteur $\vect{v}\left (-3~;~1\right )$ pour vecteur directeur.

La droite $d$ est perpendiculaire à la droite (AB) donc le vecteur directeur $\vect{v}$ de la droite (AB) est un vecteur normal à la droite $d$.
Donc $d$ a une équation de la forme $-3x+y+c=0$.

Le point C appartient à la droite $d$ donc les coordonnées de C vérifient l'équation de $d$:

$-3x_{\text C} + y_{\text C}+c=0
\iff
-3\times 2 + 4\times 1 +c=0
\iff c=2$

La droite $d$ a pour équation $-3x + y +2=0$.

\item Le point $K$, projeté orthogonal du point C sur la droite (AB), est le point d'intersection des droites (AB) et $d$; ses coordonnées vérifient donc le système:
$\left \lbrace
\begin{array}{r !{=} l}
x+3y-6 & 0\\
-3x+y+2 & 0
\end{array}
\right .$

$\left \lbrace
\begin{array}{r !{=} l}
x+3y-6 & 0\\
-3x+y+2 & 0
\end{array}
\right .
\iff
\left \lbrace
\begin{array}{r !{=} ll}
3x+9y-18 & 0 & (L_1 \gets 3L_1)\\
-3x+y+2 & 0
\end{array}
\right .\\
\hspace*{1cm}
\iff
\left \lbrace
\begin{array}{r !{=} ll}
x & -3y+6\\
10y-16 & 0 & (L_2 \gets L_1+L_2)
\end{array}
\right .
\iff
\left \lbrace
\begin{array}{r !{=} l}
x & 1,2\\
y& 1,6
\end{array}
\right .$

Donc le point K a pour coordonnées $(1,2~;~1,6)$.

\item %Calculer la distance AB et déterminer les coordonnées du milieu M du segment [AB].
\begin{list}{\textbullet}{}
\item On est placé dans un repère orthonormé donc:

$\text{AB} = \ds\sqrt{\left ( x_{\text B} - x_{\text A}\right )^2 + \left (y_{\text B} - y_{\text A} \right )^2 }
=\ds\sqrt{ (-3-3)^2 + (3-1)^2}
=\ds\sqrt{36+4}
=\ds\sqrt{40} = 2\ds\sqrt{10}$
\item Le milieu M de [AB] a pour coordonnées les moyennes des coordonnées de A et de B:

$x_{\text M} = \dfrac{x_{\text A}+ x_{\text B}}{2} = \dfrac{3-3}{2} = 0$ et
$y_{\text M} = \dfrac{y_{\text A}+ y_{\text B}}{2} = \dfrac{1+3}{2}=2$

Donc M a pour coordonnées $(0~;~2)$.
\end{list}

\item Le cercle de diamètre [AB] a pour centre M\,$(0~;~2)$ et a pour rayon $R=\dfrac{\text{AB}}{2}=\ds\sqrt{10}$, donc il a pour équation:

$\left ( x-x_{\text M}\right )^2 + \left ( y-y_{\text M} \right )^2 = R^2$
c'est-à-dire 
$\left (x - 0 \right )^2 + \left ( y-2 \right )^2 =  \left (\sqrt{10}\right )^2$
soit
$x^2 + \left (y-2\right )^2=10$.
\end{enumerate}

\begin{center}
	\large \textbf{Figures (non demandées)}
	\end{center}
	
	\bigskip
	
	\begin{center}
	\psset{unit=1cm,labelFontSize=\scriptstyle}
	\def\xmin {-4}   \def\xmax {4}
	\def\ymin {-2}   \def\ymax {6}
	\begin{pspicture*}(\xmin,\ymin)(\xmax,\ymax)
	%\psset{yMaxValue=\ymax,yMinValue=\ymin}
	\psgrid[subgriddiv=1,  gridlabels=0, gridcolor=lightgray]
	\psaxes[linewidth=1.8pt]{->}(0,0)(1,1)[$\vect{\imath}$,d][$\vect{\jmath}$,180]
	\psaxes[arrowsize=3pt 2, ticksize=-2pt 2pt,labels=none](0,0)(-3.99,-1.99)(\xmax,\ymax) 
	\uput[dl](0,0){O}
	\psset{linecolor=blue}
	\pscircle(0,2){3.162}
	\psplot{\xmin}{\xmax}{x neg 6 add 3 div}
	\psplot{\xmin}{\xmax}{3 x mul 2 sub}
	\psdots(3,1)(-3,3)(2,4)(1.2,1.6)(0,2)
	\blue
	\uput[ur](3,1){A} \uput[100](-3,3){B} \uput[l](2,4){C} 
	\uput[20](1.2,1.6){K} \uput[ur](0,2){M}
	\end{pspicture*}
	\end{center}

