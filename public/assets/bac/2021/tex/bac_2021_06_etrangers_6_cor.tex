
\textbf{Commun à tous les candidats}

\medskip

%Cet exercice est un questionnaire à choix multiples. 
%
%Pour chacune des cinq questions, quatre
%réponses sont proposées; une seule de ces réponses est exacte.
%
%\smallskip
%
%\textbf{Indiquer sur la copie le numéro de la question et recopier la réponse exacte sans justifier le choix effectué.}
%
%\textbf{Barème :} une bonne réponse rapporte un point. Une réponse inexacte ou une absence de réponse n'apporte ni n'enlève aucun point.
%
%\bigskip

\textbf{Question 1 :}

On considère la fonction $g$ définie sur $]0~;~+\infty[$  par $g(x)= x^2+ 2x - \dfrac{3}{x}$.

%Une équation de la tangente à la courbe représentative de $g$ au point d'abscisse 1 est: 
%\begin{center}
%\begin{tabularx}{\linewidth}{|*{4}{X|}}\hline
%\textbf{a.~~} $y=7(x - 1)$&\textbf{b.~~} $y = x - 1$&\textbf{c.~~} $y = 7x + 7$&\textbf{d.~~}$ y = x +1$\\ \hline
%\end{tabularx}
%\end{center}
Si $T$ est la tangente à la courbe représentative de $g$ au point d'abscisse 1, on sait que :

$M(x~;~y) \in T \iff y - g(1) = g'(1)(x - 1)$.

\starredbullet $g$ est dérivable sur $]0~;~+\infty[$ et sur cet intervalle : $g'(x) = 2x + 2 + \dfrac{3}{x^2}$

On a donc $g'(1) = 2 + 2 + \dfrac{3}{1} = 7$ ;

\starredbullet D'autre part $g(1) = 1^2 + 2 \times 1 - \dfrac{3}{1} = 0$.

On a donc : $M(x~;~y) \in T \iff y - 0 = 7(x - 1) \iff y = 7x - 7$. Réponse \textbf{a.}

\medskip

\textbf{Question 2 :}

%On considère la suite $\left(v_n\right)$ définie sur $\N$ par $v_n = \dfrac{3n}{n + 2}$. On cherche à déterminer la limite de $v_n$ lorsque $n$ tend vers $+\infty$.

%\begin{center}
%\begin{tabularx}{\linewidth}{|*{4}{X|}}\hline
%\textbf{a.~~}$\displaystyle\lim_{n \to + \infty}v_n = 1$&
%\textbf{b.~~} $\displaystyle\lim_{n \to + \infty}v_n = 3$ &  \textbf{c.~~} $\displaystyle\lim_{n \to + \infty}v_n = \dfrac{3}{2}$ &\textbf{d.~~} On ne peut pas la déterminer\\ \hline
%\end{tabularx}
%\end{center}
Pour $n$ assez grand, on a $n \ne 0$, donc $v_n = \dfrac{3}{1 + \frac{2}{n}}$

Comme $\displaystyle\lim_{n \to + \infty} \dfrac{3}{n} = 0$, on a $\displaystyle\lim_{n \to + \infty} = 3$. Réponse \textbf{b.}

\medskip

\textbf{Question 3 :}

%Dans une urne il y a 6 boules noires et 4 boules rouges. On effectue successivement 10 tirages aléatoires avec remise. Quelle est la probabilité (à $10^{-4}$ près) d'avoir 4 boules noires et 6 boules rouges?
%
%\begin{center}
%\begin{tabularx}{\linewidth}{|*{4}{X|}}\hline
%\textbf{a.~~}\np{0,1662}&\textbf{b.~~} 0,4&\textbf{c.~~} \np{0,1115}&\textbf{d.~~} \np{0,8886}\\ \hline
%\end{tabularx}
%\end{center}
Si $X$ est la variable aléatoire correspondant au nombre de boules noires tirées, l'expérience correspond à une loi de Bernoulli  et $X$ suit une loi binomiale de paramètres $n = 10$ et $p = 0,6$.

On sait que $p(X = 4) = \binom{10}{4} \times 0,6^4 \times \left(1 - 0,4^6\right) \approx \np{0,11147} \approx \np{0,1115}$ à $10^{-4}$ près. Réponse \textbf{c.~~}
\medskip

\textbf{Question 4 :}

%On considère la fonction $f$ définie sur $\R$ par $f(x) = 3\e^x - x$.

%\begin{center}
%\begin{tabularx}{\linewidth}{|*{4}{>{\small}X|}}\hline
%\textbf{a.~~}$\displaystyle\lim_{x \to + \infty} f(x) = 3$
%&\textbf{b.~~}$\displaystyle\lim_{x \to + \infty} f(x) = +\infty $&\textbf{c.~~} $\displaystyle\lim_{x \to + \infty} f(x) =  -\infty$&\textbf{d.~~} On ne peut pas déterminer la limite de la fonction $f$ lorsque $x$ tend vers $+\infty$\\ \hline
%\end{tabularx}
%\end{center}
On a pour $x \ne 0$, \, $f(x) = x\left(3\dfrac{\e^x}{x}  - 1\right)$.

Or on sait que $\displaystyle\lim_{x \to + \infty} \dfrac{\text{e}^x}{x} = + \infty$, donc 
$\displaystyle\lim_{x \to + \infty} \dfrac{\text{e}^x}{x} - 1 = + \infty$ et par produit de limites
 
$\displaystyle\lim_{x \to + \infty} f(x) = + \infty$ Réponse \textbf{b.}
\medskip

\textbf{Question 5 :}

%Un code inconnu est constitué de 8 signes. 
%
%Chaque signe peut être une lettre ou un chiffre. Il y a
%donc 36 signes utilisables pour chacune des positions.
%
%Un logiciel de cassage de code teste environ cent millions de codes par seconde. En combien de temps au maximum le logiciel peut-il découvrir le code ?
%
%\begin{center}
%\begin{tabularx}{\linewidth}{|*{4}{X|}}\hline
%\textbf{a.~~}environ 0,3 seconde&\textbf{b.~~}environ 8 heures&\textbf{c.~~}environ 3 heures&
%\textbf{d.~~}environ 470 heures\\ \hline
%\end{tabularx}
%\end{center}
Le nombre de codes différents est donc $36^8 = \np{2821109907456}$.

Il faut donc $\dfrac{\np{2821109907456}}{10^8} \approx \np{28211}$~(s) ou $\approx 478,2$~(min) ou $\approx 7,8$~(h). Réponse \textbf{b.}

\bigskip

