
\textbf{Commun à tous les candidats}

%\medskip
%
%
%Les probabilités demandées dans cet exercice seront arrondies à $10^{-3}$.
%
%\medskip
%
%Un laboratoire pharmaceutique vient d'élaborer un nouveau test anti-dopage. 

\bigskip

\textbf{Partie A}

\medskip

%Une étude sur ce nouveau test donne les résultats suivants:
%
%\setlength\parindent{9mm}
%\begin{itemize}
%\item[$\bullet~~$] si un athlète est dopé, la probabilité que le résultat du test soit positif est $0,98$ (sensibilité du test) ;
%\item[$\bullet~~$]si un athlète n'est pas dopé, la probabilité que le résultat du test soit négatif est $0,995$ (spécificité du test).
%\end{itemize}
%\setlength\parindent{0mm}
%
%\smallskip
%
%On fait subir le test à un athlète sélectionné au hasard au sein des participants à une compétition d'athlétisme. 
%
%On note $D$ l'évènement \og l'athlète est dopé \fg{} et $T$ l'évènement \og le test est positif \fg. 
%
%On admet que la probabilité de l'évènement $D$ est égale à 0,08.
%
%\medskip

\begin{enumerate}
\item ~%Traduire la situation sous la forme d'un arbre pondéré.
\begin{center}
\pstree[treemode=R,nodesepB=3pt,levelsep=2.75cm]{\TR{}}
{\pstree{\TR{$D$~~}\taput{0,08}}
	{\TR{$T$}\taput{0,98}
	\Tr{$\overline{T}$}\tbput{0,02}
	}
\pstree{\TR{$\overline{D}$~~}\tbput{0,92}}
	{\TR{$T$}\taput{0,005}
	\Tr{$\overline{T}$}\tbput{0,995}
	}	
}
\end{center}

\item %Démontrer que $P(T) = 0,083$.
D'après la loi des probabilités totales :

$P(T) = P(D \cap T) + P\left(\overline{D} \cap T \right)$ : or

$P(D \cap T) = P(D) \times P_D(T) =  0,08 \times 0,98 = \np{0,0784}$ et 

$P\left(\overline{D} \cap T \right) = P\left(\overline{D}  \right) \times P_{\overline{D}}(T) = 0,92 \times 0,005 = \np{0,00460}$. Donc :

$P(T)  = \np{0,0784} + \np{0,0046} = 0,083$.
\item 
	\begin{enumerate}
		\item %Sachant qu'un athlète présente un test positif, quelle est la probabilité qu'il soit dopé ?
La probabilité conditionnelle $P_T(D) = \dfrac{P(T \cap D)}{P(T)} = \dfrac{P(D \cap T)}{P(T)} = \dfrac{\np{0,0784}}{0,083} \approx \np{0,9445}$, soit 0,945 au millième près.
		\item %Le laboratoire décide de commercialiser le test si la probabilité de l'évènement \og un athlète présentant un test positif est dopé \fg{} est supérieure ou égale à $0,95$.
D'après la question précédente $0,945 < 0,95$, donc le test ne sera pas commercialisé.

%Le test proposé par le laboratoire sera-t-il commercialisé ? Justifier.
	\end{enumerate}
\end{enumerate}

\bigskip

\textbf{Partie B}

\medskip

%Dans une compétition sportive, on admet que la probabilité qu'un athlète contrôlé présente un test positif est $0,103$.

\medskip

\begin{enumerate}
\item %Dans cette question \textbf{1.} on suppose que les organisateurs décident de contrôler $5$ ~athlètes au hasard parmi les athlètes de cette compétition. 

%On note $X$ la variable aléatoire égale au nombre d'athlètes présentant un test positif parmi les $5$ athlètes contrôlés.
	\begin{enumerate}
		\item %Donner la loi suivie par la variable aléatoire $X$. Préciser ses paramètres.
		Quel que soit l'athlète choisi la probabilité que cet  athlète présente un test positif est $0,103$. La variable aléatoire $X$ suit donc une loi binomiale de paramètres $n = 5$ et $p = 0,103$.
		\item %Calculer l'espérance $E(X)$ et interpréter le résultat dans le contexte de l'exercice.
On sait que $E = n \times p = 5 \times 0,103 = 0,515$ : ceci montre que sur un grand nombre de contrôles, il y aura à peu près 1 athlète sur 10 contrôlé positif.
		\item %Quelle est la probabilité qu'au moins un des 5 athlètes contrôlés présente un test positif ?
La probabilité qu'aucun athlète ne soit contrôlé positif est :

$0,103^0 \times (1 - 0,103)^5 = 0,897^5 \approx \np{0,5807}$ soit environ 0,581 au millième près.

Donc la probabilité qu'au moins un des 5 athlètes contrôlés présente un test positif est :

$1 - 0,581$, soit 0,419 au millième près.
	\end{enumerate}
\item %Combien d'athlètes faut-il contrôler au minimum pour que la probabilité de l'évènement \og au moins un athlète contrôlé présente un test positif\fg{} soit supérieure ou égale à $0,75$ ? Justifier.
On a pour $n$ athlètes contrôlés, $P(X = 0) = 0,103^0 \times 0,897^n = 0,897^n$.

Il faut donc trouver $n$ tel que :

$1 - 0,897^n \geqslant 0,75 \iff 1 - 0,75 \geqslant 0,897^n \iff 0,25 \geqslant 0,897^n$.

La calculatrice donne le plus petit $n \in \N$ vérifiant cette inéquation : pour $n = 13$, on a $0,897^{13} \approx 0,243$.

Il faut donc contrôler 13 athlètes en moyenne pour en trouver un positif.
\end{enumerate}
\vspace{0,5cm}
\bigskip

