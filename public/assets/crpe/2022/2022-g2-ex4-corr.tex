\begin{enumerate}
\item 
\begin{enumerate}
	\item Si le nombre choisi au départ est $3$ : 
	
	$\text{Valeur 1} = 2 \times 3 = 6$
	
	$\text{Valeur 2} = 6 + 3 =9$
	
	$\text{Valeur 3} = 3 - 2 = 1$
	
	$\text{Résultat} = 9 \times 1 = 9$ 
	\item Si le nombre choisi au départ est $2,4$ : 
	
	$\text{Valeur 1} = 2 \times 2,4 = 4,8$
	
	$\text{Valeur 2} = 4,8 + 3 = 7,8$
	
	$\text{Valeur 3} = 2,4 - 2 = 0,4$
	
	$\text{Résultat} = 7,8 \times 0,4 = 3,12$
	
	\item Si le nombre choisi au départ est $x$ : 
	
	$\text{Valeur 1} = 2 \times x = 2x$
	
	$\text{Valeur 2} = 2x+3$
	
	$\text{Valeur 3} = x-2$
	
	$\text{Résultat} = (2x+3)\times (x-2) = 2x^2-4x+3x-6 = 2x^2-x-6$
\end{enumerate}
\item 
\begin{enumerate}
	\item Si le nombre choisi au départ est $3$ :
	
	Je l’élève au carré : $3^2 = 9$
	
	Je soustrais $3$ : $9 - 3 = 6$
	
	Je multiplie par $2$ : $6 \times 2 = 12$
	
	Je soustrais le nombre de départ : $12 - 3 = 9$
	\item Si le nombre choisi au départ est $\dfrac{7}{3}$ :
	
	Je l’élève au carré : $\left(\dfrac{7}{3}\right)^2=\dfrac{49}{9}$
	
	Je soustrais $3$ : $ \dfrac{49}{9}-3 = \dfrac{49}{9}-\dfrac{27}{9} = \dfrac{22}{9}$
	
	Je multiplie par $2$ : $\dfrac{22}{9}\times 2 = \dfrac{44}{9}$
	
	Je soustrais le nombre de départ : $\dfrac{44}{9}-\dfrac{7}{3}=\dfrac{44}{9}-\dfrac{21}{9}=\dfrac{23}{9}$
\end{enumerate}
\item Avec le programme de Pauline, si le nombre choisi au départ est $x$ : 

	Je l’élève au carré : $x^2$
	
	Je soustrais $3$ : $x^2-3$
	
	Je multiplie par $2$ : $2\times (x^2-3) = 2x^2-6$
	
	Je soustrais le nombre de départ : $2x^2-6-x= 2x^2-x-6$
	
Donc pour un même nombre choisi au départ les programmes d’Adam et de Pauline donnent le même résultat.

\item On cherche $x$ tel que $(2x+3)(x-2)=0$

Un produit de facteurs est nul si un au moins des facteurs est nul.

Donc $(2x+3)(x-2)=0$ si :
\begin{minipage}{4cm}
\begin{align*}
2x+3 &= 0\\
2x &=-3\\
x &= \dfrac{-3}{2}
\end{align*}
\end{minipage}
\begin{minipage}{1cm}
ou
\end{minipage}
\begin{minipage}{4cm}
\begin{align*}
x-2 &=0\\
x &= 2
\end{align*}
\end{minipage}


\item En B2 il doit saisir : $\mathbf{= 2*A2*A2 - A2 - 6}$.
\end{enumerate}