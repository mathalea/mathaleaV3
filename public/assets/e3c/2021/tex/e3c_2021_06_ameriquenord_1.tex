
\medskip

Ce QCM comprend 5 questions.

\emph{Pour chacune des questions, une seule des quatre réponses proposées est correcte.
Les questions sont indépendantes.\\
Pour chaque question, indiquer le numéro de la question et recopier sur la copie la lettre correspondante à la réponse choisie.\\
Aucune justification n'est demandée mais il peut être nécessaire d'effectuer des recherches au
brouillon pour aider à déterminer votre réponse.\\
Chaque réponse correcte rapporte $1$ point. Une réponse incorrecte ou une question sans réponse n'apporte ni ne retire de point.}

\medskip

\begin{tabularx}{\linewidth}{|c|X|X|X|X|}\hline
\textbf{1.} &\multicolumn{4}{|l|}{Pour tout réel $x$,\, $\text{e}^{2x} + \text{e}^{4x}$ est égal à}
\\ \hline
 &\textbf{a.} $\text{e}^{6x}$ &\textbf{b.} $\text{e}^{2x}\left(1 + \text{e}^{2}\right)$ &\textbf{c.} $\text{e}^{3x}\left(\text{e}^{x} + \text{e}^{-x} \right)$ &\textbf{d.} $\text{e}^{8x^2}$
 \\ \hline
\textbf{2.} &\multicolumn{4}{l|}{Dans le plan muni d'un repère orthonormé \Oij, on considère les }\\
& \multicolumn{4}{l|}{vecteurs $\vect{u}(-5~;~2)$ et $\vect{v}(4~;~10)$  et la droite $(d)$ d'équation : $5x + 2y + 3 = 0$.}
\\ \hline
&\textbf{a.} $\vect{u}$ et $\vect{v}$ sont\newline colinéaires&\textbf{b.}  $\vect{u}$ est un\newline  vecteur normal à la droite $(d)$ &\textbf{c.} $\vect{u}$ et $\vect{v}$ sont\newline   orthogonaux&\textbf{d.} $\vect{u}$ est un\newline  vecteur directeur de $(d)$\\ 
\hline
\textbf{3.} & \multicolumn{4}{p{8cm}|}{La dérivée $f'$ de la fonction $f$ définie sur $\R$ par $f(x) = (2x - 1)\e^{-x}$ est :} \\ \hline
&\textbf{a.} $2x\e^{-x}$ &\textbf{b.} $- 2\e^{-x}$ &\textbf{c.} $(- 2x + 3)\e^{- x}$ &\textbf{d.} $2\e^{-x} + (2x - 1)\e^{-x}$\\ 
\hline
\textbf{4.} &\multicolumn{4}{l|}{Pour tout réel $x$, on a $\sin(\pi + x) =$}\\ \hline
&\textbf{a.} $-\sin (x)$&\textbf{b.} $\cos (x)$ &\textbf{c.} $ \sin (x)$&\textbf{d.} $-\cos (x)$
\\ \hline
\begin{pspicture*}(0,0)(0.4,5.25) \uput[r](-0.1,4.8){\textbf{5.}} \end{pspicture*}
&\multicolumn{2}{p{5cm}|}{
\vspace{-4cm}Soit $f$ une fonction définie et dérivable sur $\R$ dont la courbe représentative est donnée ci-contre.
La tangente à la courbe au point A est la droite $T$.}&
\multicolumn{2}{|l|}{
\psset{unit=0.75cm}
\begin{pspicture*}(-1,-2.5)(6,4.5)
\psgrid[subgriddiv=1,gridcolor=lightgray](-1,-3)(6,4)
\psaxes[linewidth=1.25pt,labelFontSize=\scriptstyle]{->}(0,0)(-0.99,-3)(6,4)
\psplot[plotpoints=500]{-0.2}{1.5}{3 x 5 mul sub}
\psplot[plotpoints=2000,linewidth=1.25pt,linecolor=blue]{-0.2}{6}{0.9 x  sub 3.3333 mul 2.71828  x 0.45 mul  exp div}
\uput[l](1,-2){$(T)$} \uput[u](3.5,-1.9){$\blue \mathcal{C}_f$} 
\end{pspicture*}
}
\\ \hline
&\textbf{a.} $f'(0) = 3$ &\textbf{b.} $f'(0) =\dfrac{1}{5}$\rule[-12pt]{0pt}{30pt} &\textbf{c.} $f'(0) = 5$ &\textbf{d.} 
$f'(0) = - 5$\\ \hline
\end{tabularx}

\bigskip

