
\smallskip

\begin{tabular}{|l|}\hline
Principaux domaines abordés :\\
\hspace{1.25cm}$\bullet~~$Fonction exponentielle\\
\hspace{1.25cm}$\bullet~~$Convexité\\ \hline
\end{tabular}

\medskip

Pour chacune des affirmations suivantes, indiquer si elle est vraie ou fausse. 

On justifiera chaque réponse. 

\medskip

\textbf{Affirmation 1 :} Pour tous réels $a$ et $b$,\, $\left(\text{e}^{a+b}\right)^2 = \text{e}^{2a} + \text{e}^{2b}$.

\smallskip
\textbf{Affirmation 2 :}  Dans le plan muni d'un repère, la tangente au point A d'abscisse 0 à la courbe représentative de la fonction $f$ définie sur $\R$ par $f(x) = - 2 + (3 - x)\text{e}^x$ admet pour équation réduite $y = 2x + 1$.

\smallskip
\textbf{Affirmation 3 :} $\displaystyle\lim_{x \to + \infty} \text{e}^{2x} - \text{e}^{x} + \dfrac{3}{x}= 0$.

\smallskip
\textbf{Affirmation 4 :} L'équation $1 - x + \text{e}^{-x} = 0$ admet une seule solution appartenant à l'intervalle [0~;~2].

\smallskip
\textbf{Affirmation 5 :} La fonction $g$ définie sur $\R$ par $g(x) = x^2 - 5x + \text{e}^x$ est convexe.

\bigskip

