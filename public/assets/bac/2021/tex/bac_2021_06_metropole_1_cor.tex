
\textbf{Commun à tous les candidats}

\medskip


%\emph{Cet exercice est un questionnaire à choix multiples.\\
%Pour chacune des questions suivantes, une seule des quatre réponses proposées est exacte.\\
%Une réponse exacte rapporte un point. Une réponse fausse, une réponse multiple ou l'absence de réponse à une question ne rapporte ni
%n'enlève de point.\\
%Pour répondre, indiquer sur la copie le numéro de la question et la lettre de
%la réponse choisie.\\ Aucune justification n'est demandée.}
%
%\vspace{0.7cm}

Soit $f$ la fonction définie pour tout $x\in\R$ de l'intervalle $]0~;~ +\infty[$ par :
$ f(x) =\dfrac{\e^{2x}}{x}$.

On donne $f''$, définie sur l'intervalle $]0~;~ +\infty[$ par:
$f''(x)=\dfrac {2 \e^{2x} (2x^2-2x+1)}{x^3}$.

\smallskip

\begin{enumerate}
\item %La fonction $f'$, dérivée de $f$, est définie sur l'intervalle $]0~;~+\infty[$ par :
$f$ est dérivable comme fonction quotient de fonctions dérivables, le dénominateur étant non nul sur l'intervalle $]0~;~+\infty[$.

On a: $f'(x) = \dfrac{2\e^{2x} \times x - 1 \times \e^{2x}}{x^2} = \dfrac{\e^{2x}(2x - 1)}{x^2}$. \hfill{}Réponse \textbf{c.}
%\begin{tabularx}{\linewidth}{*{2}{X}}
%\textbf{a.~~} $f'(x) = 2\e^{2x}$ 					&\textbf{b.~~} $f'(x)=\dfrac{\e^{2x}(x-1)}{x^2} $\\[0.35cm]
%\textbf{c.~~}$ f'(x) = \dfrac{\e^{2x}(2x - 1)}{x^2}$& \textbf{d.~~} $f'(x)=\dfrac{\e^{2x}(1 + 2x)}{x^2} $.\\
%\end{tabularx}

\item Comme sur l'intervalle $]0~;~+\infty[$, \, $x^2 > 0$ et $\e^{2x} > 0$, le signe de $f'(x)$ est celui de $2x - 1$.

\starredbullet~$f'(x)  = 0 \iff 2x - 1 = 0 \iff x = \dfrac{1}{2}$ ;

\starredbullet~$f'(x)  < 0 \iff 2x - 1 < 0 \iff x < \dfrac{1}{2}$ : la fonction $f$ est décroissante sur $\left]0~;~\frac{1}{2}\right[$ ;

\starredbullet~$f'(x)  > 0 \iff 2x - 1 > 0 \iff x > \dfrac{1}{2}$ : la fonction $f$ est croissante sur $\left]\frac{1}{2}~;~+ \infty\right[$ ;

Conclusion : $f\left(\frac{1}{2}\right)$ est le minimum de la fonction sur $]0~;~+ \infty[$.  
\hfill{}Réponse \textbf{c.}

%\begin{tabularx}{\linewidth}{*{2}{X}}
%\textbf{a.~~} est décroissante sur $]0~;~+\infty[ $ &\textbf{b.~~} est monotone sur $]0~;~+\infty[$\\
%\textbf{c.~~}admet un minimum en $\dfrac{1}{2}$		& \textbf{d.~~}admet un maximum en $\dfrac{1}{2}$.
%\end{tabularx}
\item   On a: $f(x) = 2 \times \dfrac{\e^{2x}}{2x}$.

En posant $X = 2x$, on a $\displaystyle\lim_{x \to + \infty} 2x = \displaystyle\lim_{X \to + \infty} X = + \infty$.

On sait que $\displaystyle\lim_{X \to + \infty} \dfrac{\e^X}{X} = + \infty$, donc $\displaystyle\lim_{x \to + \infty} f(x) = + \infty$. \hfill{}Réponse \textbf{a.}
%La fonction $f$ admet pour limite en $+ \infty$ :

%\begin{tabularx}{\linewidth}{*{4}{X}}
%\textbf{a.~~} $+\infty $ &\textbf{b.~~} $0$&\textbf{c.~~}$1$& \textbf{d.~~} $\e^{2x}$.
%\end{tabularx}
\item  Sur $]0~;~+\infty[$, \, $x^3 > 0$ et $2\e^{2x} > 0$, donc le signe de $f''(x)$ est celui du trinôme $2x^2 - 2x +  1$.

Or $2x^2 - 2x +  1 = 2\left(x^2 - x + \frac{1}{2}\right) = 2\left[ \left (x - \frac{1}{2}\right)^2 - \frac{1}{4} + \frac{1}{2}\right ] = 2\left(x - \frac{1}{2}\right)^2 + \frac{1}{2}$. 

Donc $f''(x)$ somme de deux nombres positifs est positive sur $]0~;~+ \infty$. La fonction est donc convexe sur $]0~;~+\infty[$. \hfill{}Réponse \textbf{b.}

%La fonction $f$ :

%\begin{tabularx}{\linewidth}{*{4}{X}}
%\textbf{a.~~} est concave sur $]0~;~+\infty[$ &\textbf{b.~~} est convexe  $]0~;~+\infty[$\\
%\textbf{c.~~} est concave sur $\left]0~;~\frac{1}{2}\right] $& \textbf{d.~~} est représentée par une courbe
%admettant un point d'inflexion.\\
%\end{tabularx}
 
\end{enumerate}

\vspace{0,5cm}

