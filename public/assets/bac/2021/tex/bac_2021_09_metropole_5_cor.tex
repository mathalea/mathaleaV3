
\medskip

\begin{tabular}{|l|}\hline
Principaux domaines abordés:\\
Fonction logarithme.\\ \hline
\end{tabular}

\bigskip

\textbf{Partie I}

\medskip

On considère la fonction $h$ définie sur l'intervalle $]0~;~ +\infty[$ par:
$h(x) = 1 + \dfrac{\ln (x)}{x}.$

\smallskip

\begin{enumerate}
\item %Déterminer la limite de la fonction $h$ en $0$.
$\displaystyle\lim_{x \to 0 \atop x>0} \ln x = - \infty$, et $\displaystyle\lim_{x \to 0 \atop x>0} \dfrac{1}{x} = + \infty$.

Par produit on déduit que $\displaystyle\lim_{x \to 0 \atop x>0} \dfrac{\ln x}{x} = - \infty$ et donc que $\displaystyle\lim_{x \to 0 \atop x>0} h(x) = - \infty$.

\item %Déterminer la limite de la fonction $h$ en $+\infty$.
On sait que $\displaystyle\lim_{x \to + \infty}\dfrac{\ln x}{x} = 0$, donc $\displaystyle\lim_{x \to + \infty} h(x) = 1$
\item %On note $h'$ la fonction dérivée de $h$. Démontrer que, pour tout nombre réel $x$ de $]0~;~ +\infty[$, on a:
La fonction $h$ est dérivable comme quotient de fonctions dérivables sur $]0~;~ +\infty[$ et sur cet intervalle :
$h'(x) = \dfrac{\frac{1}{x} \times x - 1 \times \ln x}{x^2} = \dfrac{1 - \ln x}{x^2}$.
%\[h'(x) = \dfrac{1 - \ln (x)}{x^2}.\]

\smallskip

\item %Dresser le tableau de variations de la fonction $h$ sur l'intervalle $]0~;~ +\infty[$.
Comme $x^2 > 0$ pour $x \in ]0~;~+ \infty[$ le signe de $h'(x)$ est celui du numérateur $1 - \ln x$ :

\begin{list}{\textbullet}{}
\item $1 - \ln x > 0 \iff 1 > \ln x \iff \text{e} > x$ : \\
la fonction $h$ est donc strictement croissante sur $]0~;~\text{e}[$ ;

\item $1 - \ln x < 0 \iff 1 < \ln x \iff \text{e} < x$ :\\
 la fonction $h$ est donc strictement décroissante sur $]\text{e}~;~+ \infty[$ ;

\item $1 - \ln x = 0 \iff 1 = \ln x \iff \text{e} = x$ : \\
la fonction $h$ a un maximum $f(\text{e}) = 1 + \dfrac{\ln \text{e}}{\text{e}} = 1 + \dfrac{1}{\text{e}}$.
\end{list}

D'où le tableau de variations de $h$ :

\begin{center}
{\renewcommand{\arraystretch}{1.3}
\psset{nodesep=3pt,arrowsize=2pt 3}%  paramètres
\def\esp{\hspace*{2.5cm}}% pour modifier la largeur du tableau
\def\hauteur{0pt}% mettre au moins 20pt pour augmenter la hauteur
$\begin{array}{|c|l*4{c}|}
\hline
x & 0  & \esp & \e\, & \esp & +\infty \\ 
\hline
h'(x) & \vline\;\vline &   \pmb{+} & \vline\hspace{-2.7pt}0 & \pmb{-} & \\ 
\hline
 &  \vline\;\vline & &   \Rnode{max}{1+\frac{1}{\e}}  &  &   \\  
h &\vline\;\vline &     &  &  &  \rule{0pt}{\hauteur} \\ 
 &\vline\;\vline \Rnode{min1}{~-\infty} &   &  &  &   \Rnode{min2}{1} \rule{0pt}{\hauteur}    
 \ncline{->}{min1}{max} 
 \ncline{->}{max}{min2} 
 \\ 
\hline
\end{array} $
}
\end{center}	

\item %Démontrer que l'équation $h(x) = 0$ admet une unique solution $\alpha$ dans $]0~;~ +\infty[$.
Comme $1 + \frac{1}{\text{e}} > 1 > 0$, le tableau de variations montre que l'équation $h(x) = 0$ admet une solution unique $\alpha \in ]0~;~\text{e}[$.

On a $f(1) = 1 + \dfrac{0}{1} = 1$, donc $0 < \alpha < 1$ ;

La calculatrice donne :
$f(0,5) \approx - 0,4$ et $f(0,6) \approx 0,15$, donc $0,5 < \alpha < 0,6$.
%Justifier que l'on a : $0,5 < \alpha < 0,6$.
\end{enumerate}

\bigskip

\textbf{Partie II}

\medskip

Soit les fonctions $f$ et $g$ définies sur $]0~;~ +\infty[$ par
$f(x) = x \ln (x) - x$ et $g(x) = \ln (x).$

%On note $\mathcal{C}_f$ et $\mathcal{C}_g$ les courbes représentant respectivement les fonctions $f$ et $g$ dans un repère orthonormé \Oij.
%
%Pout tout nombre réel $a$ strictement positif, on appelle:
%
%\setlength\parindent{1cm}
%\begin{itemize}
%\item[$\bullet~~$] $T_a$ la tangente à $\mathcal{C}_f$ en son point d'abscisse $a$ ;
%\item[$\bullet~~$] $D_a$ la tangente à $\mathcal{C}_g$ en son point d'abscisse $a$.
%\end{itemize}
%\setlength\parindent{0cm}
%
%Les courbes  $\mathcal{C}_f$ et $\mathcal{C}_g$ ainsi que deux tangentes $T_a$ et $D_a$ sont représentées ci-dessous.

\begin{center}
\psset{unit=1cm}
\begin{pspicture*}(-0.6,-2)(7,6)
\psgrid[gridlabels=0pt,subgriddiv=1,gridwidth=0.06pt]
\psaxes[linewidth=1.25pt,labelFontSize=\scriptstyle](0,0)(0,-1.95)(6.9,5.9)
\psaxes[linewidth=1.25pt,labelFontSize=\scriptstyle]{->}(0,0)(1,1)
\psplot[plotpoints=2000,linewidth=1.25pt,linecolor=red]{0.01}{7}{x ln x mul x sub}
\psplot[plotpoints=2000,linewidth=1.25pt,linecolor=blue]{0.01}{7}{x ln}
\psplot[plotpoints=2000,linewidth=1.25pt,linecolor=red,linestyle=dashed]{0.01}{7}{1.833 x mul 6.254 sub}
\psplot[plotpoints=2000,linewidth=1.25pt,linecolor=blue,linestyle=dashed]{0.01}{7}{0.162 x mul 0.833 add}
\uput[u](1,1){\blue $D_a$}\uput[r](2.9,-1){\red $T_a$}\uput[l](4.3,2){\red $\mathcal{C}_f$ }
\uput[r](0.2,-1.8){\blue $\mathcal{C}_g$}
\psplotTangent[linecolor=red]{0.566}{4}{x ln x mul x sub}
\psplotTangent[linecolor=blue]{0.566}{4}{x ln}
\psline[linestyle=dotted,linewidth=1.25pt](6.25,0)(6.25,5.204)
\uput[d](6.25,0){$a$}
\end{pspicture*}
\end{center}

%On recherche d'éventuelles valeurs de $a$ pour lesquelles les droites $T_a$ et $D_a$ sont perpendiculaires. 
%
%Soit $a$ un nombre réel appartenant à l'intervalle $]0~;~ +\infty[$.


\medskip

\begin{enumerate}
\item %Justifier que la droite $D_a$ a pour coefficient directeur $\dfrac{1}{a}$.
La fonction $f$ est dérivable sur $]0~;~ +\infty[$  et sur cet intervalle :

$f'(x) = \ln (x) + x \times \dfrac{1}{x} - 1 = \ln (x) + 1 - 1  = \ln (x)$

Donc le coefficient directeur de la tangente à $\mathcal{C}_f$ au point de la courbe d'abscisse $a$ est égal à $f'(a) = \ln (a)$.

\item %Justifier que la droite $T_a$ a pour coefficient directeur $\ln (a)$.
La fonction $g$ est dérivable sur $]0~;~ +\infty[$  et sur cet intervalle :
$g'(x) = \dfrac{1}{x}$.

Donc le coefficient directeur de la tangente à $\mathcal{C}_g$ au point de la courbe d'abscisse $a$ est égal à $g'(a) = \dfrac{1}{a}$.
\end{enumerate}

%On rappelle que dans un repère orthonormé, deux droites de coefficients directeurs respectifs $m$ et $m'$sont perpendiculaires si et seulement si $mm' = -1$.

\begin{enumerate}[resume]
\item %Démontrer qu'il existe une unique valeur de $a$, que l'on identifiera, pour laquelle les droites $T_a$ et $D_a$ sont perpendiculaires.
Le produit des coefficients directeurs est égal à $- 1$, soit :

$\ln (a) \times \dfrac{1}{a} = - 1 \iff \dfrac{\ln (a)}{a} = - 1 \iff 1 + \dfrac{\ln (a)}{a} = 0 \iff h(a) = 0$

 et on a vu à la fin de la partie I que cette équation n'avait qu'une solution $a = \alpha$ : il existe une seule valeur de $a$ telle que les droites $T_a$ et $D_a$ sont perpendiculaires : $a = \alpha$. Voir la figure.
\end{enumerate}


%%%%%%%%%%%%%%%%%%%% SUJET J2 %%%%%%%%%%%%%%%%%%%ù
